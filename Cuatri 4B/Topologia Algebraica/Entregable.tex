\documentclass[leqno]{article}
\usepackage{verbatim}
\usepackage{array}
\usepackage{listings}
\usepackage{fancyvrb}
\usepackage{enumitem}

\usepackage[utf8]{inputenc}
\usepackage[T1]{fontenc}
\usepackage{textcomp}
\usepackage{multicol} \usepackage{mathtools}
\usepackage{amsmath}
\usepackage{wrapfig}
\usepackage{amssymb}
\usepackage{amsmath,amsfonts,amssymb,amsthm,epsfig,epstopdf,titling,url,array}
\usepackage{hyperref}
\usepackage{eso-pic}
\usepackage{pgf}
\usepackage{tikz}
\usepackage{tikz-cd}
\usepackage{graphicx}

% figure support
\usepackage{import}
\usepackage{xifthen}
\pdfminorversion=7
\usepackage{pdfpages}
\usepackage{transparent}
\usepackage{xcolor}

% geometry
\usepackage{geometry}
\geometry{a4paper, margin=1in}

% paragraph length
\setlength{\parindent}{0em}
\setlength{\parskip}{1em}

\newtheorem*{theorem}{Theorem}
\newtheorem*{lemma}{Lemma}
\newtheorem*{proposition}{Proposition}
\newtheorem*{definition}{Definition}
\newtheorem*{observation}{Observation}

\DeclareMathOperator{\Hom}{Hom}
\DeclareMathOperator{\Poset}{Poset}
\DeclareMathOperator{\im}{Im}
\DeclareMathOperator{\coker}{Coker}
\DeclareMathOperator{\coim}{Coim}

\newcommand{\com}[1]{\textcolor{red}{#1}}
\newcommand{\incfig}[1]{%
\center
\def\svgwidth{0.9\columnwidth}
\import{./figures/}{#1.pdf_tex}
}
\newcommand{\incimg}[1]{%
\center
\includegraphics[width=0.9\columnwidth]{images/#1}
}
\pdfsuppresswarningpagegroup=1

\title{Entregable Topología Algebraica}
\author{Abel Doñate Muñoz}
\date{}

\begin{document}
\maketitle
\tableofcontents
\newpage

\section{Homología en la categoría de Poset}

Para ver que $C_{\bullet}(P)$ es, en efecto, un complejo, debemos probar que la aplicación $d$ es la aplicación borde, es decir  $d^2 = 0$.
 \begin{align*}
 &d\circ d (x_0\le \cdots \le x_n) = \sum_{i=0}^n (-1)^{i} d(x_0\le  \cdots \le  \hat{x_i} \le \cdots\le x_n) = \\
 &=\sum_{i=0}^n \left( \sum_{j=0}^{i-1} (-1)^{i+j}(x_0\le \cdots\le \hat{x_j}\le \cdots \le \hat{x_i}\le \cdots \le x_n)  - \sum_{j=i+1}^{n} (-1)^{i+j} (x_0\le \cdots\le \hat{x_i}\le \cdots \le \hat{x_j}\le \cdots \le x_n) \right) 
\end{align*}
Pero esta expresión es cero, ya que si fijamos dos indices diferentes $k, l$, vemos que pasa una vez por  $(i, j)=(k, l)$ y otra por  $(j, i) = (k, l)$ con signo cambiado anulándose. Ya que toda cadena es combinación lineal de cadenas con coeficientes en $\mathbb{Z}$, al extender la cadena por linealidad del morfismo $d$ tenemos que  $d^2$ sigue dando  $0$ para toda cadena. Esto prueba que 
% https://q.uiver.app/#q=WzAsNSxbMSwwLCJDX3tuKzF9Il0sWzIsMCwiQ19uIl0sWzMsMCwiQ197bi0xfSJdLFswLDAsIlxcY2RvdHMiXSxbNCwwLCJcXGNkb3RzIl0sWzAsMSwiZCJdLFsxLDIsImQiXSxbMywwLCJkIl0sWzIsNCwiZCJdXQ==
\[\begin{tikzcd}
	\cdots & {C_{n+1}(P)} & {C_n(P)} & {C_{n-1}(P)} & \cdots
	\arrow["d_{n+1}", from=1-2, to=1-3]
	\arrow["d_{n}", from=1-3, to=1-4]
	\arrow["d_{n+2}", from=1-1, to=1-2]
	\arrow["d_{n-1}", from=1-4, to=1-5]
\end{tikzcd}\]
es en efecto un complejo. Al ser un complejo en una categoría abeliana, podemos definir la homología como
\[
H_n(P) = \ker(d_n) / \im (d_{n+1}) := Z_n / B_{n+1}
\]
que esta bien definida gracias a que es un complejo y la imagen queda dentro del núcleo.

Para los ejemplos empezaremos por calcular $H_0$. Observamos que $d_1(x)=0\ \forall x$, por lo que dado el conjunto del poset con cardinal $n$ tenemos que todo el grupo abeliano $\sum a_n x_n$ esta en el nucleo de  $d_1$, por lo que  $Z_0 \cong  \mathbb{Z}^n$. De la misma forma si consideramos la imagen $B_1$, esta esta formada por todas las cadenas de la forma $x-y$ si  $x\le y$, es decir, si están en la misma componente conexa del grafo de Hasse (no importa la dirección, pues si está al reves haremos $y-x$). Por tanto al pasar por el cociente estamos identificando todas las componentes conexas y $H_0 \cong \mathbb{Z}^{\# \{\text{comp. con.}\}}$



\end{document}
