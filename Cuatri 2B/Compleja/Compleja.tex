\documentclass[leqno]{article}
\usepackage{verbatim}
\usepackage{array}
\usepackage{listings}
\usepackage{fancyvrb}
\usepackage{enumitem}

\usepackage[utf8]{inputenc}
\usepackage[T1]{fontenc}
\usepackage{textcomp}
\usepackage{multicol}
\usepackage{mathtools}
\usepackage{amsmath}
\usepackage{wrapfig}
\usepackage{amssymb}
\usepackage{amsmath,amsfonts,amssymb,amsthm,epsfig,epstopdf,titling,url,array}
\usepackage{hyperref}
\usepackage{eso-pic}
\usepackage{pgf}
\usepackage{tikz}
\usepackage{graphicx}

% figure support
\usepackage{import}
\usepackage{xifthen}
\pdfminorversion=7
\usepackage{pdfpages}
\usepackage{transparent}
\usepackage{xcolor}

\setlength{\parindent}{0em}
\setlength{\parskip}{1em}

\newtheorem*{definition}{Definition}
\newtheorem*{theorem}{Theorem}
\newtheorem*{proposition}{Proposition}

\newcommand{\incfig}[1]{%
\center
\def\svgwidth{0.9\columnwidth}
\import{./figures/}{#1.pdf_tex}
}


\newcommand{\incimg}[1]{%
\center
\includegraphics[width=0.9\columnwidth]{images/#1}
}


\pdfsuppresswarningpagegroup=1

\title{Funciones de Varible Compleja}

\begin{document}
\maketitle
\tableofcontents
\newpage

\section{Funciones Holomorfas}
\begin{definition}
$f$ es \textbf{Holomorfa} ($f\in \mathcal{H}$) si
\[
\exists f'(z_0) = \lim_{z\to z_0} \frac{f(z)-f(z_0)}{z-z_0}
\] 
\end{definition}

De la definición se deduce que si $f$ es holomorfa  $\implies \exists \ f_x, f_y$.

\begin{definition}
$f$ es  \textbf{Analítica} en $U$ si  $\ \forall z_0\in U$ existe una serie de potencias con radio de convergencia positivo.
\end{definition}

En variable compleja se cumple holomorfa  $\iff$ analítica

\begin{theorem}
Ecuaciones de Cauchy-Riemann (CR en adelante) (condiciones necesarias para que $f$ sea holomorfa)
\[
f_y(z_0) = if_x(z_0) 
\iff 
\begin{cases}
u_x = v_y \\
u_y = -v_x
\end{cases}
\iff
\begin{cases}
f_z = f'\\
f_{\overline{z}} = 0
\end{cases}
\] 
\end{theorem}

\begin{theorem}
Condiciones suficientes para que $f$ sea holomorfa
\[
\begin{cases}
  u, v\in \mathcal{C}(\Omega) \\
  CR
\end{cases}
\implies
f\in \mathcal{H}(\Omega)
\] 
\end{theorem}
Si las funciones $f, g$ son holomorfas, entonces son holomorfas
\[
\lambda f+\mu g , \quad fg, \quad \frac{f}{g}, \quad f\circ g
\] 
y sus derivadas coinciden con las derivadas en el caso $z\in \mathbb{R}$

\begin{definition}
Una función $L$ es lineal $\iff L(z) = \mu z = re^{i\varphi }z$ 
\[
  L(z) = L(x + iy) = \begin{pmatrix} a & -c \\ c & a \end{pmatrix} \begin{pmatrix} x \\ y \end{pmatrix} 
\] 
Observamos que se trata de una rotación + homotecia
\end{definition}

\begin{proposition}
$L$ preserva ángulos y orientación $\implies L $ es $\mathbb{C}-$lineal
\end{proposition}

\begin{definition}
$f$ es conforme en $\Omega \iff f$ preserva ángulos y orientación
\end{definition}

Se cumple $f$ conforme  $\iff f$ lineal

\begin{definition}
Una función $u$ es armónica si $\Delta u = u_{x x} + u_{yy} = 0$ 

Si $f = u + iv$ holomorfa  $\implies u, v$ armónicas
\end{definition}

\begin{definition}
Sea $a_n$ una sucesión, definimos la serie de potencias centrada en $a$
\[
\sum_{n\ge 0} a_n(z-a)^n
\] 
y su radio de convergencia como
\[
R = \frac{1}{\limsup_{n\to \infty} \sqrt[n]{|a_n|} }
\] 
\end{definition}

\begin{theorem}
Cauchy-Hadamard. Sea $R$ el radio de convergencia de  $S = \sum_{n\ge 0} a_n (z-a)^n$.
\begin{enumerate}[topsep=0pt, itemsep=0pt]
  \item $S$ es absolutamente convergente en  $|z-a|<R$ 
  \item $S$ es divergente en  $|z-a|>R$ 
  \item $S$ es uniformemente convergente en $|z-a|\le r$ con $r<R$
\end{enumerate}
\end{theorem}

\begin{theorem}
Criterio M de Weierstrass. Sea $f_n$ una sucesión de funciones en $S$ tal que existe una sucesión de reales  $M_n$ tal que  $|f_n(z)|<M_m \ \forall z\in S$  y $\sum_{n\ge 0} M_n<\infty$, entonces
\[
\sum_{n\ge 0} f_n(z) \text{ es absolutamente y uniformemente convergente en } S
\] 
\end{theorem}

Observamos que no podemos saber nada de la convergencia en el borde del disco $|z-a|=R$ con Cauchy-Hadamard

\begin{theorem}
Picard. 
\[
\text{si } \begin{cases}
  c_j\ge 0, c_j\in \mathbb{R} \\
  c_j\ge c_{j+1} \\
  \lim c_j = 0
\end{cases}
\implies \sum_{n\ge 0} c_nz^n \text{ es convergente } \ \forall |z|=1 \text{ excepto quizá para } z=1
\] 
\end{theorem}

\begin{theorem}
Sea $f(z) = \sum a_n(z-a)^n$ con radio de convergencia $R$S. Se cumple
 \begin{enumerate}[topsep=-6pt, itemsep=0pt]
  \item $f\in H(D(a;R))$
  \item $f\in \mathcal{C}^\infty(D(a;R))$ 
  \item $f^{(k)}$ tiene radio de convergencia $R$
\end{enumerate}
\end{theorem}

\section{Teoría local de Cauchy}
\begin{definition}
La integral de $f$ sobre un camino $\Gamma$ con parametrización $\gamma(t):[a, b]\to \mathbb{C}$ se define como
\[
\int_\Gamma f(z)dz = \int_a^b f(\gamma(t))\gamma'(t)dt
\] 
\end{definition}

Si integramos una función derivada, tenemos un campo potencial $\int_\Gamma f'(z)dz = f(\gamma(b))-f(\gamma(a)) $

\begin{theorem}
Teorema de Green
\[
  \iint_R \frac{\partial F}{\partial \overline{z}}dxdy = \frac{1}{2i}\int_\gamma f(z)dz
\] 
\end{theorem}

\begin{theorem}[Cauchy-Goursat]
$f\in C(\Omega )\cap H(\Omega -\{w\} \implies \displaystyle \int_{\partial R^+} f(z)dz=0$
\end{theorem}

\begin{theorem}[Existencia de primitivas]
$f\in C(D)\cap H(D-\{w\}) \implies \ \exists F\in H(D): F'(z) = f(z)$
\end{theorem}

\begin{theorem}[Cauchy]
\[
f\in H(D)\cap \mathcal{C}(\overline{D}) \implies f(w) = \frac{1}{2\pi i}\int \frac{f(z)}{z-w}dz
\] 
\end{theorem}

\begin{theorem}[Representación en serie de potencias] $f\in H(\Omega )$
\[
f(z) = \sum c_n(z-a)^n \implies c_n = \frac{f^{(n)}(a)}{n!} = \frac{1}{2\pi i} \int_{\partial D(a, r)+} \frac{f(z)}{(z-a)^{n+1}}dz
\] 
\end{theorem}

\begin{theorem}
Extensión analítica. $f\in H(\Omega)$. Si $\ \exists U$ abierto $: f|_U=0 \implies f=0$ en todo $\Omega$
\end{theorem}

\begin{theorem}
Si $f\in H(\Omega) \implies f$ el conjunto de los ceros de $f$ es numerable
\end{theorem}

\begin{theorem} Sea $f\in H(\Omega), f\neq 0$. Sea $E = \{z\in \Omega : f(z) = 0\}$ el conjunto de ceros. Entonces
  \[
  \ \forall a\in \Omega \ \exists ! \ m\ge 1 : f(z) = (z-a)^mg(z) \quad \text{con} \quad g\in H(\Omega),\ g(a)\neq 0
  \] 

\end{theorem}

\begin{theorem}[Desigualdad de Cauchy]
Sea $f\in H(D(a;R))$, definimos $M(r) = \sup_{z\in \partial D(a, r)} \{|f(z)|\}$. Entonces
\[
|f^{(n)}(a)|\le n! M(r) r^{-n}
\] 
\end{theorem}

Como consecuencia tenemos los dos siguientes teoremas

\begin{theorem}[Liouville]
Sea $f$ entera y acotada. Entonces $f(z) = c \ \forall z\in \mathbb{C}$ 
\end{theorem}

\begin{theorem}[Fundamental del Álgebra]
Sea $p(z) = a_0 + a_1z + \ldots + a_nz^n$ un polinomio de grado $n\ge 1$. Entonces 
\begin{enumerate}[topsep=-6pt, itemsep=0pt]
  \item $P$ tiene  $n$ raíces en  $\mathbb{C}$ (contando multiplicidad) 
  \item Factoriza como  $p(z)= a_n(z-\alpha _1) \cdots (z-\alpha _n)$
\end{enumerate}
\end{theorem}

\begin{proposition}
Si $f$ entera y $\ \exists r, M, \lambda: |f(z)|\le M|z|^\lambda \ \forall |z|\ge r$, entonces $f$ es un polinomio de grado  $\le \lambda$
\end{proposition}

\begin{theorem}[Valor medio]
Sea $f\in H(\Omega)$, el valor de $f$ en el centro de un disco es el promedio del de la frontera
\[
f(a) = \frac{1}{2\pi}\int_0^{2\pi}f(a+ re^{i \theta}) d\theta
\] 
\end{theorem}

\begin{theorem}[Principio del máximo fuerte]
Sea $f\in H(\Omega)$. Si $|f|$ alcanza el supremo en $\Omega^{\circ} \implies f(z) = c$
\end{theorem}

\begin{theorem}[Principio del mínimo fuerte]
Sea $f\in H(\Omega)$, $f(z)\neq 0 \ \forall z\in \Omega$. Si $|f|$ alcanza el ínfimo en  $\Omega^\circ \implies f(z) = c$
\end{theorem}

\begin{theorem}[Teorema de la función inversa] Sea $f\in H(\Omega), f'(a)\neq 0$, entonces $f$ tiene una inversa local holomorfa tal que  $f^{-1}(f(a))=a$. Es decir, $\ \exists r>0$ tal que
  \begin{enumerate}[topsep=-6pt, itemsep=0pt]
    \item $D(a;r) \subseteq  \Omega, f(D(a;r))$ abierto
	\item $f: D(a;r) \to f(D(a;r)$ biyectiva
	\item $(f^{-1})'(f(w))=\frac{1}{f'(w)}$
  \end{enumerate}
\end{theorem}

\begin{theorem} Sea $f\in H(\Omega), f\neq c$. Sea $m\ge 1$ el orden del cero de $f(z)-f(a)$ en  $z = a$. Entonces  $\ \exists U\subseteq \Omega$ entorno de $a$, $\varphi \in H(U)$ y $r>0$ tal que
   \begin{enumerate}[topsep=-6pt, itemsep=0pt]
    \item $f(z) = f(a) + (\varphi (z))^m \ \forall z\in U$
	\item $\varphi '(z) \neq 0 \ \forall z\in U$ y $\varphi : U \to D(0;r)$ es biyectiva
  \end{enumerate}
\end{theorem}

\begin{theorem}[Teorema de la aplicación abierta]
Sea $f\in H(\Omega), f\neq c$,  $f$ es abierta (envía abiertos a abiertos)
\end{theorem}

\begin{theorem}[Weierstrass]
Sea $f_i \in H(\Omega) \in i\ge 1$. $\ \forall K\subseteq \Omega$ compacto $f_n|_K \to f|_K$ es uniformemente convergente a la función $f:\Omega \to \mathbb{C}$. Entonces $f\in H(\Omega)$
\end{theorem}

\section{Teoría global de Cauchy}
\begin{definition}[Indice respecto a un punto] sea $\Gamma$ un camino cerrado y orientado en $\mathbb{C}$, definimos el indice de $\Gamma$ respecto al punto $a$ como
  \[
  n(\Gamma, a) := \frac{1}{2\pi i} \int_{\Gamma} \frac{dz}{z-a}
  \] 
\end{definition}

\begin{theorem}
Sea $\Gamma$ un camino cerrado y orientado fijado. Entonces $n(\Gamma, \cdot): \mathbb{C}-\Gamma \to \mathbb{Z}$
es constante en cada una de sus componentes conexas y $0$ en la no fitada
\end{theorem}

\begin{definition}[Ciclos homólogos]
$\Gamma$ es homólogo a $0$ en $\Omega (\Gamma \sim 0) \iff \ \forall a \not\in \Omega \ n(\Gamma, a) = 0$ 

Dos ciclos son homólogos en $\Omega (\Gamma_1 \sim \Gamma_2) \iff \Gamma_1-\Gamma_2 \sim 0$ 
\end{definition}

\begin{definition}[Homotopía de caminos]
Dos caminos son homótopos $(\Gamma_1 \simeq \Gamma_2)$ si se pueden deformar de forma continua en $\Omega$. Es decir
\[
\ \exists F: [0, 1]\times [0, 1] \to \Omega, F\in \mathcal{C}^0 : \begin{cases}
  F(0, t) = \gamma_1(t) \\
  F(1, t) = \gamma_2(t) \\
  F(s, 1) = F(s, 0)
\end{cases}
\] 
\end{definition}

Observamos que $\Gamma_1 \simeq \Gamma_2 \implies \Gamma_1 \sim \Gamma_2$ pero $\Gamma_1 \sim \Gamma_2 \not\implies \Gamma_1 \simeq \Gamma_2$


\begin{theorem}[Teorema de Cauchy Global]
Sea $f\in H(\Omega)$ y $\Gamma$ ciclo tal que $\Gamma \sim 0$ en $\Omega$
\[
\int_\Gamma f(z)dz = 0, \qquad n(\Gamma, a)f^{(k)}(a) = \frac{k!}{2\pi i}\int_\Gamma \frac{f(z)}{(z-a)^{k+1}}dz  \quad \ \forall a\in \Omega-\Gamma
\] 
\end{theorem}

\begin{definition}[Singularidades] .
\begin{enumerate}[topsep=-6pt, itemsep=0pt]
  \item \textbf{Evitable} si $\ \exists \lim_{z\to a}f(z)=L\in \mathbb{C}$ $\implies$ prolongación holomorfa con $f(a) = L$ 
  \item \textbf{Polo} si $\ \exists \lim_{z\to a}|f(z)|=\infty$ $\implies f(z) = (z-a)^{-m}h(z)$ con $h$ holomorfa
  \item \textbf{Esencial} si $\not \exists \lim_{z\to a}f(z)$ $\implies f(D(a;\varepsilon )-\{a\})$ es denso en $\mathbb{C}$ (Casorati-Weierstrass)
\end{enumerate}
\end{definition}

\begin{theorem}[Series de Laurent]
Sea $f\in H(A(R_1, R_2))$, entonces f admite un desarrollo
\[
f(z) = \sum_{n\in \mathbb{Z}} a_n(z-a)^n, \qquad a_n = \frac{1}{2\pi i} \int_{\partial D(a;r)^+} \frac{f(z)}{(z-a)^{n+1}}dz, \qquad R_1<r<R_2
\] 
\end{theorem}

\begin{definition}[Residuo]
Sea $f\in H(\Omega)$ y $a$ una singularidad aislada de $f$ 
\[
Res(f, a) = \frac{1}{2\pi i}\int_{\partial D(a; \varepsilon) } f(z)dz \qquad \varepsilon >0 : D(a;\varepsilon )-\{a\} \subseteq \Omega
\] 
\end{definition}
Observamos que si $\displaystyle  f(z) = \sum_{n\in \mathbb{Z}} a_n(z-a)^n  \implies Res(f, a) = a_{-1} =  \frac{1}{(m-1)!} \lim_{z\to a} \frac{d^{m-1}}{dz^{m-1}}\left((z-a)^mf(z)\right)$ 

\begin{theorem}[Residuos]
Sea $f\in H(\Omega-\{a_j\}$ con $\{a_j\}$ conjunto de singularidades aisladas finitas o numerables y $\Gamma \sim 0$ un ciclo  $a_j\not\in \Gamma$.
\[
\int_\Gamma f(z)dz = 2\pi i \sum_j n(\Gamma, a_j)Res(f, a_j)
\] 
\end{theorem}

\begin{definition}[Función meromorfa]
Sea $f\in H(\Omega)$ con $\{a_j\}$ conjunto de singularidades aisladas finitas o numerables.  $f$ es meromorfa si todas las $a_j$ son polos.
\end{definition}

\begin{theorem}[Principio del argumento]
Sea $f$ meromorfa en $\Omega$ con ceros $\{a_j\}$ y polos  $\{b_j\}$. Sea $\Gamma \sim 0$ un ciclo tal que $a_j, b_j \not \in \Gamma $ :
\[
n(f(\Gamma ), 0) = \frac{1}{2\pi i} \int_\Gamma \frac{f'(z)}{f(z)}dz = \sum_j n(\Gamma , a_j) - \sum_k n(\Gamma , b_k)
\]
\end{theorem}

\begin{theorem}[Rouche]
Sea $f\sim 0: \ \forall z\in \Omega -\Gamma , \ n(\Gamma , z)= 0 \text{ ó } 1, \quad f, g \in H(\Omega ), \quad |f(z)-g(z)|< |f(z)| \ \forall z\in \Gamma  \implies \#z_{f, \Gamma } = \#z_{g, \Gamma }$

\end{theorem}

\begin{proposition} Son equivalentes:
\begin{enumerate}[topsep=-6pt, itemsep=0pt]
  \item $\Omega $ simplemente conexo
  \item Para todo ciclo $\Gamma \subseteq \Omega $ se cumple $\Gamma \sim 0$ 
  \item Sea $f\in H(\Omega )$ existe $F\in H(\Omega )$ primitiva holomorfa tal que $F' = f$
  \item $f\in H(\Omega ), \Gamma \subseteq \Omega$ se tiene $\int_\Gamma f(z)dz = 0$ 
  \item Sea  $f \in H(\Omega ), f(z)\neq 0 \ \forall z\in \Omega $. Existe $g \in H(\Omega ): f = e^g$ ($g=\log f$ determinación holomorfa del logaritmo)
\end{enumerate}
\end{proposition}

\begin{theorem}[Gauss-Lucas]
Sea $p(z)$ un polinomio. Las raíces de $p'(z)$ están en la envoltura convexa de las raíces de  $p(z)$
\end{theorem}








\end{document}
