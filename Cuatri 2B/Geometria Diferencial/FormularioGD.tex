\documentclass[leqno]{article}
\usepackage{verbatim}
\usepackage{array}
\usepackage{listings}
\usepackage{fancyvrb}
\usepackage{enumitem}
\usepackage{multicol}

\usepackage[utf8]{inputenc}
\usepackage[T1]{fontenc}
\usepackage{textcomp}
\usepackage{multicol}
\usepackage{mathtools}
\usepackage{amsmath}
\usepackage{wrapfig}
\usepackage{amssymb}
\usepackage{amsmath,amsfonts,amssymb,amsthm,epsfig,epstopdf,titling,url,array}
\usepackage{hyperref}
\usepackage{eso-pic}
\usepackage{pgf}
\usepackage{tikz}
\usepackage{graphicx}

% figure support
\usepackage{import}
\usepackage{xifthen}
\pdfminorversion=7
\usepackage{pdfpages}
\usepackage{transparent}
\usepackage{xcolor}

\setlength{\parindent}{0em}
\setlength{\parskip}{1em}

\newtheorem*{definition}{Definition}
\newtheorem*{theorem}{Theorem}
\newtheorem*{proposition}{Proposition}

\newcommand{\incfig}[1]{%
\center
\def\svgwidth{0.9\columnwidth}
\import{./figures/}{#1.pdf_tex}
}


\newcommand{\incimg}[1]{%
\center
\includegraphics[width=0.9\columnwidth]{images/#1}
}
\pdfsuppresswarningpagegroup=1


\begin{document}

\begin{multicols}{2}[\columnsep2em]

Parámetro arco 
\[
s(t) = \int_{t_0}^t \|\beta' (u)\| du
\]
Triedro de Frenet (bon)
 \[
   T(t) = \frac{\beta'(t)}{\|\beta'(t)\|} \quad N(t) =
\frac{T'(t)}{\|T'(t)\|} \quad
B = T\times N
\] 
Curvatura y torsión
 \[
k(s) = \|T'(s)\|= \langle T'(s) , N(s) \rangle
\]
\[
  \tau(s) = \langle N'(s) , B(s) \rangle 
\]
Para parámetro arbitrario
\[
k = \frac{\|\gamma' \times \gamma''\|}{\|\gamma'\|^3}, \quad \tau = \frac{\det\left( \gamma', \gamma'', \gamma^{3)} \right) }{\|\gamma'\times \gamma''\|^2}
\]
Si la curva es bidimensional
\[
k = \frac{\det(\gamma',\gamma'')}{\|\gamma'\|^3}
\] 

Fórmulas de Frenet
\begin{align*}
  T' &= kN \\
  N' &= -kT + \tau B \\
  B' &= -\tau N 
\end{align*}

Superficie parametrizada
\[
\sigma(u, v) = \begin{pmatrix} x(u, v) \\ y(u, v) \\ z(u, v) \end{pmatrix}, \quad D\sigma = \begin{pmatrix} x_u & x_v \\ y_u & y_v \\ z_u & z_v \end{pmatrix}
\]
Primera forma fundamental
\[
P =   \begin{pmatrix} E & F \\ F & G \end{pmatrix} = \begin{pmatrix} \langle \sigma_u , \sigma_u \rangle & \langle \sigma_u , \sigma_v \rangle \\ \langle \sigma_u , \sigma_v \rangle & \langle \sigma_v , \sigma_v \rangle   \end{pmatrix} 
\]
Cálculo de longitudes:
\[
\begin{pmatrix} u(t)  \\ v(t) \end{pmatrix} \implies   L = \int_a^b \sqrt{Eu'^2 + 2Fu'v' + Gv'^2}dt
\] 

Cálculo de áreas
\[
A = \iint_X\|\sigma _u\times \sigma _v\|dudv=  \iint_X \sqrt{EG-F^2}dudv
\]
Cálculo de ángulos
\[
\cos \alpha  = \frac{\langle u, v \rangle }{\|u\|\|v\|} = \frac{u^TPv}{\sqrt{u^TPu}\sqrt{v^TPv}  }
\] 

Aplicación de Gauss
\[
N: S\to \mathbb{S}^2 \quad N(q) = \frac{\sigma _u\times \sigma _v}{\|\sigma _u\times \sigma _v\|}
\]
Aplicación de Weingarten.
\[
  DN \text{ matriz } A = \begin{pmatrix} a_{11} & a_{12} \\ a_{21} & a_{22} \end{pmatrix} \text{ en base }\sigma _u, \sigma _v 
\]
Segunda forma fundamental
\[
  S = \begin{pmatrix} e & f \\ f & g \end{pmatrix} = \begin{pmatrix} -\langle \sigma _u , N_u \rangle & -\langle \sigma _u , N_v \rangle \\ -\langle \sigma _v , N_u \rangle & -\langle \sigma _v , N_v \rangle   \end{pmatrix}= \]
  \[=   \begin{pmatrix} \langle \sigma _{uu} , N \rangle & \langle \sigma _{uv} , N \rangle \\ \langle \sigma _{uv} , N \rangle & \langle \sigma _{vv} , N \rangle   \end{pmatrix} =  -PA = -A^TP 
\]
\[
II(w) = -\langle w, DN w \rangle  = w^TSw \text{ (base } \sigma_u, \sigma _v)
\] 
\[
k_1=-\lambda_1 \quad k_2 = -\lambda_2 \text{  (VAPs de A)}
\] 
\[
  k_i^2 -2Hk_i+K = 0
\] 
Curvatura de Gauss
\[
\det(DN) = \det(A) = \frac{\det(S)}{\det(P)} = k_1k_2
\]
$K>0$ elíptico,  $K<0$ hiperbólico \\
$K=0, k_1\neq 0$ parabólico,  $K=0, k_i=0$ plano

Curvatura Media
\[
H = \frac{k_1+k_2}{2} = \frac{-a_{11}-a_{22}}{2}= \frac{eG-2Ff+Eg}{2(EG-F^2)}
\] 
(fórmulas de aij)
\[
a_{11} = \frac{Ff-eG}{EG-F^2}, \quad a_{12} = \frac{gF-Fg}{EG-F^2}
\] 
\[
  a_{21}=\frac{eF-Ef}{EG-F^2}, \quad a_{22} = \frac{fF-Eg}{EG-F^2}
\] 

Curvatura normal ($T$ vector tangente)
\[
k_n = T^TST
\] 

Símbolos de Cristoffel
  \[
  \begin{cases}
    \sigma _{uu} = \Gamma_{11}^1\sigma _u + \Gamma_{11}^2 \sigma _v + eN \\
    \sigma _{uv} = \Gamma_{12}^1\sigma _u + \Gamma_{12}^2 \sigma _v + fN \\
    \sigma _{vv} = \Gamma_{22}^1\sigma _u + \Gamma_{22}^2 \sigma _v + gN
  \end{cases}
  \]
Los calculamos como
\[
  \begin{pmatrix} \Gamma_{11}^1 & \Gamma_{12}^1 & \Gamma_{22}^1 \\ \Gamma_{11}^2 & \Gamma_{12}^2 & \Gamma_{22}^2 &  \end{pmatrix}  = P^{-1} \begin{pmatrix} \frac{1}{2}E_u & \frac{1}{2}E_v & F_v - \frac{1}{2}G_u \\ F_u - \frac{1}{2}E_v & \frac{1}{2}G_u & \frac{1}{2}G_v \end{pmatrix} 
\]

Fórmulas de Gauss
\begin{enumerate}[topsep=-6pt, itemsep=0pt]
    \item $EK = (\Gamma_{11}^2)_v - (\Gamma_{12}^2)_u + \Gamma_{11}^1\Gamma_{12}^2 + \Gamma_{11}^2\Gamma_{22}^2 - \Gamma_{12}^1\Gamma_{11}^2 - \Gamma_{12}^2\Gamma_{12}^2$
    \item $FK = (\Gamma_{12}^1)_u - (\Gamma_{11}^1)_v + \Gamma_{12}^1\Gamma_{12}^2 - \Gamma_{11}^2\Gamma_{22}^1$
    \item $GK = (\Gamma_{22}^1)_u - (\Gamma_{12}^1)_v + \Gamma_{11}^1\Gamma_{22}^1 + \Gamma_{12}^1\Gamma_{22}^2 - \Gamma_{12}^1\Gamma_{12}^1 - \Gamma_{12}^2\Gamma_{22}^1$
  \end{enumerate}

  Codazzi-Mainardi
  \begin{enumerate}[topsep=-6pt, itemsep=0pt]
    \item $e_v-f_u = e\Gamma_{12}^1 + f(\Gamma_{12}^2 - \Gamma _{11}^1) - g\Gamma_{11}^2$
	\item $f_v-g_u = e\Gamma_{22}^1 + f(\Gamma_{22}^2-\Gamma_{12}^1)-g\Gamma_{12}^2$
  \end{enumerate}


\end{multicols}
\newpage

Campo tangente
\[
X(u(t), v(t)) = a(u(t), v(t))\sigma _u + b(u(t), v(t))\sigma _v
\]

Derivada Covariante
\[
  \nabla _wX = \left[ a' + \begin{pmatrix} a & b \end{pmatrix} \begin{pmatrix} \Gamma_{11}^1 & \Gamma _{12}^1 \\ \Gamma _{12}^1 & \Gamma_{22}^1 \end{pmatrix} \begin{pmatrix} u' \\ v' \end{pmatrix}   \right] \sigma _u + \left[ b' + \begin{pmatrix} a & b \end{pmatrix} \begin{pmatrix} \Gamma _{11}^2 & \Gamma _{12}^2 \\ \Gamma _{12}^2 & \Gamma _{22}^2 \end{pmatrix} \begin{pmatrix} u' \\ v' \end{pmatrix}   \right] \sigma _v
\]

Geodésicas
\[
  \begin{cases}
   u'' + \begin{pmatrix} u' & v' \end{pmatrix} \begin{pmatrix} \Gamma_{11}^1 & \Gamma _{12}^1 \\ \Gamma _{12}^1 & \Gamma_{22}^1 \end{pmatrix} \begin{pmatrix} u' \\ v' \end{pmatrix} =0  \\
  v'' + \begin{pmatrix} u' & v' \end{pmatrix} \begin{pmatrix} \Gamma _{11}^2 & \Gamma _{12}^2 \\ \Gamma _{12}^2 & \Gamma _{22}^2 \end{pmatrix} \begin{pmatrix} u' \\ v' \end{pmatrix} = 0
  \end{cases}
\]

Relación de Clairut. En una superficie de revolución
\[
\begin{cases}
  x = \varphi (u)\cos v\\
  y = \varphi (u)\sin v \\
  z = \psi (u)
\end{cases}
\text{ con } \varphi '^2 + \psi '^2 = 1
\] 
Entonces una geodésica $(u(t), v(t))$ satisface $\varphi (u)\cos \theta = \text{const}$































\end{document}
