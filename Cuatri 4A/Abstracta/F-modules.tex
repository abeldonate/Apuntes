\documentclass[leqno]{article}
\usepackage{verbatim}
\usepackage{array}
\usepackage{listings}
\usepackage{fancyvrb}
\usepackage{enumitem}

\usepackage[utf8]{inputenc}
\usepackage[T1]{fontenc}
\usepackage{textcomp}
\usepackage{multicol} \usepackage{mathtools}
\usepackage{amsmath}
\usepackage{wrapfig}
\usepackage{amssymb}
\usepackage{amsmath,amsfonts,amssymb,amsthm,epsfig,epstopdf,titling,url,array}
\usepackage{hyperref}
\usepackage{eso-pic}
\usepackage{pgf}
\usepackage{tikz}
\usepackage{tikz-cd}
\usepackage{graphicx}

% figure support
\usepackage{import}
\usepackage{xifthen}
\pdfminorversion=7
\usepackage{pdfpages}
\usepackage{transparent}
\usepackage{xcolor}

% geometry
\usepackage{geometry}
\geometry{a4paper, margin=1in}

% paragraph length
\setlength{\parindent}{0em}
\setlength{\parskip}{1em}


\newtheorem*{theorem}{Theorem}
\newtheorem*{lemma}{Lemma}
\newtheorem*{proposition}{Proposition}

\theoremstyle{definition}
\newtheorem*{definition}{Definition}
\newtheorem*{observation}{Observation}
\newtheorem*{note}{Note}


\newcommand{\com}[1]{\textcolor{red}{#1}}
\newcommand{\incfig}[1]{%
\center
\def\svgwidth{0.9\columnwidth}
\import{./figures/}{#1.pdf_tex}
}
\newcommand{\incimg}[1]{%
\center
\includegraphics[width=0.9\columnwidth]{images/#1}
}
\pdfsuppresswarningpagegroup=1

\title{F-Modules}
\author{Abel Doñate Muñoz}
\date{}

\begin{document}
\maketitle
\tableofcontents
\newpage

\section{Introduction}
We will work in positive characteristic. Let $R$ commutative unital ring of prime characteristic $p$.

\section{The Frobenius functor}

\begin{definition}[Frobenius endomorphism] The map
\[
  f: R \to R \quad \text{such that} \quad f(r) = r^p
\] 
defines a ring morphism in a ring of characteristic $p$ known as Frobenius endomorphism.
\end{definition}

Notice that the application is, in fact, a morphism. The behaviour for the product $f(ab)=(ab)^p=a^pb^p=f(a)f(b)$, and for the sum we have to make use of binomial expansion
\[
  f(a+b) = (a+b)^p = a^p + \binom{p}{1} a^{p-1}b^1 + \cdots + \binom{p}{p-1}a^{1}b^{p-1} + b^p = a^p + b^p = f(a) + f(b)
\] 
since $p|\binom{p}{k} \ \forall k=1, \ldots, p-1$.

\begin{observation} $f$ is not necessarily injective nor surjective. Some counterexamples are
   \begin{enumerate}[topsep=-6pt, itemsep=0pt]
	 \item 
  \end{enumerate}
\end{observation}






\end{document}
