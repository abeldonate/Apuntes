\documentclass{beamer}
\usepackage[utf8]{inputenc}
\usepackage{graphicx} % Required for inserting images
\usepackage{amsmath}
\usepackage{geometry}
\usepackage{svg}
\usepackage{float}
\usepackage{xcolor}
\usepackage{booktabs}
\usepackage{subcaption}
\usepackage{tikz}
\usepackage{tikz-cd}

\newcommand{\catname}[1]{{\mathbf{#1}}}
\newcommand{\Set}{\catname{Set}}
\newcommand{\Mod}{\catname{R-Mod}}
\DeclareMathOperator{\Hom}{Hom}
\DeclareMathOperator{\im}{Im}
\DeclareMathOperator{\coker}{Coker}
\DeclareMathOperator{\coim}{Coim}
\DeclareMathOperator{\heigth}{ht}
\DeclareMathOperator{\Spec}{Spec}
\DeclareMathOperator{\Ann}{Ann}

\usetheme{Madrid}
\usecolortheme{default}
%------------------------------------------------------------
%This block of code defines the information to appear in the
%Title page
\title[F-módulos] %optional
{$F-$módulos}

%\subtitle{}

\author[Abel Doñate] % (optional)
{Abel Doñate Muñoz}

\institute[UPC] % (optional)
{
  Universitat Politècica de Catalunya
}

\date[Enero 2024] % (optional)
{Presentación del trabajo final, Enero 2024}

%\logo{\includegraphics[height=1cm]{overleaf-logo}}

%End of title page configuration block
%------------------------------------------------------------



%------------------------------------------------------------
%The next block of commands puts the table of contents at the 
%beginning of each section and highlights the current section:

%------------------------------------------------------------


\begin{document}

%The next statement creates the title page.
\frame{\titlepage}


%---------------------------------------------------------
%This block of code is for the table of contents after
%the title page
\begin{frame}
\frametitle{Table of Contents}
\tableofcontents
\end{frame}
%---------------------------------------------------------


\section{Functor de Frobenius}

%---------------------------------------------------------
\begin{frame}{}
\begin{block}{Endomorfismo de Frobenius}
Sea $R$ un anillo con característica $p>0$. Definimos el endomorfismo de Frobenius como el mapa
\begin{align*}
f: R &\to R \\
r &\to r^p
\end{align*}
\end{block}    

\begin{block}{Observación}
    Este morfismo en general no es inyectivo ni exhaustivo.
\end{block}
\end{frame}

\begin{frame}{}
\begin{block}{Module with Frobenius action} Given $M$ an  $R-$Module, we define the module  $M^{(e)}$ induced by  $f^{(e)}$ as the abelian group $M$ endowed with the action
  \[
  r \cdot  m  = f ^{(e)}(r)m = r ^{p^e} m
  \] 
\end{block}
\end{frame}

\begin{frame}[fragile]{Functor de Frobenius}
\begin{block}{Functor de Frobenius}
  Definimos el functor de Frobenius como el el functor $F:\Mod \to \Mod$ que envía
  \[\begin{tikzcd}[column sep=small]
	M & {R'\otimes_RM,} & {(M} & {N)} & {R'\otimes_R M} & {R'\otimes_RN}
	\arrow["\phi", from=1-3, to=1-4]
	\arrow["{id\otimes_R\phi}", from=1-5, to=1-6]
	\arrow[maps to, from=1-4, to=1-5]
	\arrow[maps to, from=1-1, to=1-2]
\end{tikzcd}\]
\end{block}
\end{frame}

\begin{frame}[fragile]{}
\begin{block}{Frobenius of a complex}
  Given the complex $M^{\bullet}$, we define its induced complex $F(M^{\bullet})$ as the complex
\[\begin{tikzcd}
	\cdots & {M_{k-1}} & {M_{k}} & {M_{k+1}} & \cdots \\
	\cdots & {F(M_{k-1})} & {F(M_{k})} & {F(M_{k+1})} & \cdots
	\arrow["F", from=1-2, to=2-2]
	\arrow["F", from=1-3, to=2-3]
	\arrow["F", from=1-4, to=2-4]
	\arrow["{h_{k-1}}", from=1-2, to=1-3]
	\arrow["{h_k}", from=1-3, to=1-4]
	\arrow["{F(h_{k-1})}", from=2-2, to=2-3]
	\arrow["{F(h_k)}", from=2-3, to=2-4]
	\arrow[from=1-1, to=1-2]
	\arrow[from=1-4, to=1-5]
	\arrow[from=2-4, to=2-5]
	\arrow[from=2-1, to=2-2]
\end{tikzcd}\]
Exactly the same construction works for $F ^{(e)}$.
\end{block}
\end{frame}



\begin{frame}[fragile]{Properties}
\begin{block}{Properties of Frobenius functor}
\begin{enumerate}
  \item $F$ is right exact. Furthermore, if $R$ is regular, then $R'$ is flat and  $F$ is exact.
  \item $F$ commutes with direct sums.
  \item $F$ commutes with localization.
  \item $F$ commutes with direct limits. 
  \item $F$ preserves finitely generation of modules.
  \item If $R$ is regular, then $F$ commutes with cohomology of complexes.
\end{enumerate}
\end{block}
\end{frame}



\begin{frame}[fragile]{Properties}
\begin{block}{Frobenius power ideal}
 Given  $I = (x_1, \ldots, x_n)$ an ideal of $R$, we define its Frobenius $e-$power ideal as
   \[
	 I _{p^e} := (x_1^{p^e}, \ldots, x_n ^{p^e})R
  \] 
\end{block}
\begin{block}{Some examples of transformations}
  \begin{itemize}
	\item $F(I) \cong I_{p^e}$
	\item $F(R / I)\cong R / I_{p^e}$
  \end{itemize}
\end{block}
\end{frame}

%---------------------------------------------------
%---------------------------------------------------

\section{$F-$modules}

\begin{frame}[fragile]{$F-$module}
\begin{block}{Definition of $F-$module}
An $F-module$ is an $R-$module $M$ equipped with an $R-$isomorphism  $\theta:M \to F(M)$ called the structure morphism.
\end{block}
\begin{block}{Morphism of $F-$modules}
Given two  $F-$ modules  $(M, \theta _M)$ and $(N, \theta _N)$, we say $f:M\to N$ is a morphism of $F-$modules if the following diagram commutes
  % https://q.uiver.app/#q=WzAsNCxbMCwwLCJNIl0sWzEsMCwiTiJdLFswLDEsIkYoTSkiXSxbMSwxLCJGKE4pIl0sWzAsMiwiXFx0aGV0YV9NIl0sWzEsMywiXFx0aGV0YV9OIl0sWzAsMSwiZyJdLFsyLDMsIkYoZykiXV0=
\[\begin{tikzcd}
	M & N \\
	{F(M)} & {F(N)}
	\arrow["{\theta_M}", from=1-1, to=2-1]
	\arrow["{\theta_N}", from=1-2, to=2-2]
	\arrow["g", from=1-1, to=1-2]
	\arrow["{F(g)}", from=2-1, to=2-2]
\end{tikzcd}\]
\end{block}
\end{frame}



\end{document}


\begin{frame}[fragile]{Properties}
\begin{block}{Frobenius power ideal}
\end{block}
\end{frame}
