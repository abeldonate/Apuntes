\documentclass[leqno]{article}
\usepackage{verbatim}
\usepackage{array}
\usepackage{listings}
\usepackage{fancyvrb}
\usepackage{enumitem}

\usepackage[utf8]{inputenc}
\usepackage[T1]{fontenc}
\usepackage{textcomp}
\usepackage{multicol} \usepackage{mathtools}
\usepackage{amsmath}
\usepackage{wrapfig}
\usepackage{amssymb}
\usepackage{amsmath,amsfonts,amssymb,amsthm,epsfig,epstopdf,titling,url,array}
\usepackage{hyperref}
\usepackage{eso-pic}
\usepackage{pgf}
\usepackage{tikz}
\usepackage{tikz-cd}
\usepackage{graphicx}

% figure support
\usepackage{import}
\usepackage{xifthen}
\pdfminorversion=7
\usepackage{pdfpages}
\usepackage{transparent}
\usepackage{xcolor}

% geometry
\usepackage{geometry}
\geometry{a4paper, margin=1in}

% paragraph length
\setlength{\parindent}{0em}
\setlength{\parskip}{1em}

\newtheorem*{theorem}{Theorem}
\newtheorem*{lemma}{Lemma}
\newtheorem*{proposition}{Proposition}
\newtheorem*{definition}{Definition}
\newtheorem*{observation}{Observation}

\DeclareMathOperator{\Hom}{Hom}
\DeclareMathOperator{\im}{Im}
\DeclareMathOperator{\coker}{Coker}
\DeclareMathOperator{\coim}{Coim}
\newcommand{\prob}[2]{\fbox{\parbox{\textwidth}{\textbf{#1} #2}}}
\newcommand{\incfig}[1]{%
\center
\def\svgwidth{0.9\columnwidth}
\import{./figures/}{#1.pdf_tex}
}
\newcommand{\incimg}[1]{%
\center
\includegraphics[width=0.9\columnwidth]{images/#1}
}
\pdfsuppresswarningpagegroup=1

\title{Problems Abstract Algebra \\ \Large{First List}}
\author{Abel Doñate Muñoz}
\date{}

\begin{document}
\maketitle
\prob{1}{Let $f$ be a morphism in a category  $\mathcal{C}$. Prove the following:
  \begin{enumerate}[topsep=-6pt, itemsep=0pt]
	\item[(a)] If $f$ an isomorphism then $f$ is a monomorphism and an epimorphism.
	\item[(b)] The inclusion of $\mathbb{Z}$ in $\mathbb{Q}$ is a monomorphism and an epimorphism in the category of rings but not an isomorphism.
  \end{enumerate}\vspace{0.5em}
}

We begin with the proof of $(a)$. Since  $f:A\to B$ is an isomorphism, that means there exist  $g:B \to A$ such that both $g \circ f = Id_A $ and $f\circ g = Id_B$. 

Let $h, k$ morphisms of the category that fulfill $f\circ h=f\circ k$. Then by composing from the left with $g$ we have
\[
g\circ f\circ h = g\circ f \circ k \Rightarrow Id_A\circ h = Id_A\circ k \Rightarrow h=k
\]
so we conclude $f$ is a monomorphism.

Let $h, k$ morphisms of the category that fulfill $h \circ f=k\circ f$. Then by composing from the right with $g$ we have
\[
h\circ f\circ g=k\circ f\circ g \Rightarrow h\circ Id_B = k\circ Id_B \Rightarrow h=k
\]
so we conclude $f$ is an epimorphism.

We move to the proof of $(b)$. Let $i:\mathbb{Z} \to \mathbb{Q}$ the inclusion in $\mathbb{Q}$ ($i:n \mapsto n$). Now let $h, k \in \Hom_{rings}(A, \mathbb{Z})$ such that  $i\circ h = i\circ k$. It is clear that, since $i(n)=n\ \forall n\in \mathbb{Z}$, then $h(a)=k(a)\ \forall a\in A$, concluding $h=k$ and  $i$ monomorphism.

Now let $h, k \in \Hom_{rings}(\mathbb{Z}, A)$ such that  $h\circ i = k\circ i$. It is clear that, since $i(n)=n\ \forall n\in \mathbb{Z}$, then $h(i(a))=k(i(a)) \Rightarrow  h(a)=k(a)\ \forall a\in A$, concluding $h=k$ and  $i$ epimorphism.

Suppose $i$ is an isomorphism. Thus, it must exists  $g:\mathbb{Q}\to \mathbb{Z}$ such that $i\circ g = Id_\mathbb{Q}$ and $g\circ i = Id_\mathbb{Z}$. Let $a\in \mathbb{Z}$ such that $g(\frac{1}{2})=a$. Then $i\circ g (\frac{1}{2}) = i(a)=a \neq \frac{1}{2}$, so $i\circ g \neq Id_{\mathbb{Q}}$, concluding $f$ is not an isomorphism.


\prob{2}{Show that in the category of finite dimensional vector spaces over a field $\mathbb{K}$ we have a natural equivalente of functors between the identity $Id$ and the bidual  $(-)^{**}$}

We consider the map $\tau :X\to X^{* *}$ that sends $v \mapsto \text{eval}(v)$, where eval$(v)\in \Hom(\Hom(X, \mathbb{R}), \mathbb{R})$ is the evaluation in $v$ morphism.

% https://q.uiver.app/#q=WzAsNCxbMCwwLCJYIl0sWzEsMCwiWF57Kip9Il0sWzAsMSwiWSJdLFsxLDEsIlleeyp9Il0sWzAsMSwiXFx0YXUiXSxbMiwzLCJcXHRhdSJdLFswLDIsImYiLDJdLFsxLDMsIkcoZikiXV0=
\[\begin{tikzcd}
	X & {X^{**}} \\
	Y & {Y^{*}}
	\arrow["\tau", from=1-1, to=1-2]
	\arrow["\tau", from=2-1, to=2-2]
	\arrow["f"', from=1-1, to=2-1]
	\arrow["{G(f)}", from=1-2, to=2-2]
\end{tikzcd}\]

Now we have to see that the diagram commutes. We will see that the construction $G(f):\text{eval}(v)\mapsto \text{eval}(f(v))$ works. If we pick $v\in X$, then
\[
G(f)(\tau (v))= \text{eval}(f(v)) = \text{eval}(w) = \tau (f(v)) \quad \Rightarrow \tau \circ f = G(f)\circ \tau 
\] 
Proving the commutativity of the diagram.

(terminar)

\prob{3}{Show that two categories $\mathcal{B}$ and $\mathcal{C}$ are naturally equivalent if and only if there exists a fully faithful and essentially surjective covariant functor $F:\mathcal{B}\to \mathcal{C}$.
}


\prob{4}{
\textit{Pullbacks in the category of abelian groups:} Let $A$ and  $B$ be abelian groups together with homomorphisms $f:A\to S$ and $g:B\to S$. Prove that 
\[
  A \times_SB = \{(a,b)\in A\times B | f(a)=g(b)\} 
\] 
}

Let $U=\{(a,b)\in A\times B|f(a)=g(b)\}$. We will show that the pullback $A\times _SB$ is, in fact, $U$. We construct the following diagram:

% https://q.uiver.app/#q=WzAsNSxbMSwxLCJVIl0sWzEsMiwiQiJdLFsyLDEsIkEiXSxbMiwyLCJTIl0sWzAsMCwiQyJdLFswLDIsIlxccGlfQSJdLFswLDEsIlxccGlfQiIsMl0sWzEsMywiZyIsMl0sWzIsMywiZiJdLFs0LDAsImgiXSxbNCwyLCJoX0EiLDAseyJjdXJ2ZSI6LTJ9XSxbNCwxLCJoX0IiLDAseyJjdXJ2ZSI6Mn1dXQ==
\[\begin{tikzcd}
	C \\
	& U & A \\
	& B & S
	\arrow["{\pi_A}", from=2-2, to=2-3]
	\arrow["{\pi_B}"', from=2-2, to=3-2]
	\arrow["g"', from=3-2, to=3-3]
	\arrow["f", from=2-3, to=3-3]
	\arrow["h", from=1-1, to=2-2]
	%\arrow["{h_A}", curve={height=-12pt}, from=1-1, to=2-3]
	\arrow["{h_A}", bend left,from=1-1, to=2-3]
	\arrow["{h_B}", bend right,from=1-1, to=3-2]
\end{tikzcd}\]

We first construct the morphisms $\pi_A$ and $\pi_B$ that make the square commute. Those are
\[
  \begin{cases}
    \pi_A((a,b))=a\\
    \pi_B((a,b))=b
  \end{cases} \Rightarrow f\circ \pi_A ((a,b)) = f(a) = g(b) = g\circ \pi_B((a,b)) \ \forall (a,b)\in U
\]
thus, the square commutes.

Now we construct $h$ from  $h_A$ and  $h_B$. Note that, for the two triangular diagrams to commute, the $h$ must fulfill:
\[
\begin{cases}
  \pi_A(h(c))=h_A(c)\\
  \pi_B(h(c))=h_B(c)
\end{cases} \ \forall c\in C \quad \Rightarrow \quad h = (h_A, h_B)
\]
and the $h$ is unique, concluding the proof.


\prob{5}{
\textit{Pushouts in the category of abelian groups:} Let $A$ and  $B$ be abelian groups together with homomorphisms $f:S\to A$ and $g:S\to B$. Prove that 
\[
A \sqcup_SB = \frac{A \oplus B}{W}
\] 
where $W$ is the subgroup generated by $(f(s), -g(s))$ with  $s\in S$.
}

Let $U=\frac{A\oplus B}{W}$. We will show that the pushout $A\sqcup_S B $ is, in fact,  $U$. We construct the following diagram:

% https://q.uiver.app/#q=WzAsNSxbMSwxLCJVIl0sWzEsMiwiQiJdLFsyLDEsIkEiXSxbMiwyLCJTIl0sWzAsMCwiQyJdLFsyLDAsImkiLDJdLFsxLDAsImoiXSxbMywxLCJnIl0sWzMsMiwiZiIsMl0sWzAsNCwiaCIsMl0sWzIsNCwiaF9BIiwyLHsiY3VydmUiOjJ9XSxbMSw0LCJoX0IiLDIseyJjdXJ2ZSI6LTJ9XV0=
\[\begin{tikzcd}
	C \\
	& U & A \\
	& B & S
	\arrow["i"', from=2-3, to=2-2]
	\arrow["j", from=3-2, to=2-2]
	\arrow["g", from=3-3, to=3-2]
	\arrow["f"', from=3-3, to=2-3]
	\arrow["h"', from=2-2, to=1-1]
	\arrow["{h_A}"', bend right, from=2-3, to=1-1]
	\arrow["{h_B}"', bend left, from=3-2, to=1-1]
\end{tikzcd}\]

We first construct the morphisms $i, j$ such that the  square diagram commutes. We propose 
 \[
  i(a)=[(a,0)] \qquad  j(b)=[(0,b)]
\] 
and we check for commutativity for all $s\in S$
\[
  \begin{cases}
  i\circ f (s) = [(f(s),0)]\\
  j\circ g (s) = [(0,g(s))]
  \end{cases} \text{ but } 
  [(0,g(s))]=[(0,g(s))+(f(s),-g(s))]=[(f(s),0)] \Rightarrow i\circ f = j\circ g \ \forall s\in S
\] 
so we have proved the square commutes.

Now we construct the morphism $h$ through  $h_A$ and  $h_B$. We construct it in the following way:
\[
  \begin{cases}
	h([(a,0)])=h_A(a)\\
	h([(0,b)])=h_B(b)
  \end{cases} \Rightarrow h([(a,b)])=h([(a,0)]+[(0,b)])=h([(a,0)])+h([(0,b)]) = h_A(a)+h_B(b)
\] 
and clearly this is well defined and unique as morphism in the category of abelian groups, so the other triangular diagrams commute as well and we conclude with the proof.

\prob{6}{\textit{Inverse limits in the category of sets / groups / abelian groups / modules}: Let $(\{A_i\}, \{f_{ji}\})$ be an inverse system over a preordered set $I$. Prove that
  \[
	\varprojlim A_i = \{(a_i)\in \prod A_i | f_{ji}(a_j)=a_i \ i\le j\}
  \] 
}

Let $U=\{(a_i)\in \prod A_i | f_{ji}(a_j)=a_i \ i\le j\}$. We will prove that this is, in fact, the inverse limit we are looking for through the following diagram:

% https://q.uiver.app/#q=WzAsNCxbMCwxLCJCIl0sWzEsMSwiVSJdLFsyLDAsIkFfaSJdLFsyLDIsIkFfaiJdLFswLDEsImgiXSxbMSwyLCJcXGxhbWJkYV9pIiwyXSxbMSwzLCJcXGxhbWJkYV9qIl0sWzMsMiwiZl97aml9IiwyXSxbMCwyLCJmX2kiXSxbMCwzLCJmX2oiLDJdXQ==
\[\begin{tikzcd}
	&& {A_i} \\
	B & U \\
	&& {A_j}
	\arrow["h", from=2-1, to=2-2]
	\arrow["{\lambda_i}"', from=2-2, to=1-3]
	\arrow["{\lambda_j}", from=2-2, to=3-3]
	\arrow["{f_{ji}}"', from=3-3, to=1-3]
	\arrow["{f_i}", bend left, from=2-1, to=1-3]
	\arrow["{f_j}"', bend right, from=2-1, to=3-3]
\end{tikzcd}\]

We can easily check the commutativity of the right part of the diagram because $f_{ji}\circ \lambda_j((a_k))=f_{ji}(a_j)=a_i = \lambda_i((a_k))$.

We now prove that the morphism $h$ is unique.  $h$ must be of the form  $h(c)=(h_k(c))$, but, since the upper and lower part of the diagram must commute, we have $\lambda_i\circ h(c)=f_i(c)\Rightarrow h_i=f_i$, so the morphism we are looking form is $h=(h_k)$, and is unique.


\prob{7}{ \textit{Direct limits in the category of sets / groups / abelian groups / modules / rings with unit}: Let $(\{A_i\}, \{f_ij\})$ be a direct system over a directed set  $I$. Prove that
   \[
   \varinjlim A_i = \bigsqcup A_i / \sim    
  \] 
  where $a_i\sim a_j \iff f_{il}(a_i)=f_{jl}(a_j)$ for $i,j\le l$
}

Let $U = \bigsqcup A_i / \sim $. We will prove that this is, in fact, the direct limit we are looking for through the following diagram:

% https://q.uiver.app/#q=WzAsNCxbMCwxLCJCIl0sWzEsMSwiVSJdLFsyLDAsIkFfaSJdLFsyLDIsIkFfaiJdLFsxLDAsImgiLDJdLFsyLDEsIlxcbGFtYmRhX2kiXSxbMywxLCJcXGxhbWJkYV9qIiwyXSxbMywyLCJmX3tqaX0iLDJdLFsyLDAsImZfaSIsMl0sWzMsMCwiZl9qIl1d
\[\begin{tikzcd}
	&& {A_i} \\
	B & U \\
	&& {A_j}
	\arrow["h"', from=2-2, to=2-1]
	\arrow["{\lambda_i}", from=1-3, to=2-2]
	\arrow["{\lambda_j}"', from=3-3, to=2-2]
	\arrow["{f_{ji}}"', from=3-3, to=1-3]
	\arrow["{f_i}"', bend right, from=1-3, to=2-1]
	\arrow["{f_j}", bend left, from=3-3, to=2-1]
\end{tikzcd}\]

We have to check first that the right triangle of the diagram commutes. Since $\lambda_i:a_i \mapsto  [a_i]$, we have that
\[
  \lambda_j\circ  f_{ij}(a_i) = \lambda_j(a_j)=[a_j] = [a_i] = \lambda_i(a_i) \Rightarrow \lambda_j\circ f_{ij}=\lambda_i
\] 
where we have set $l=j$, so $f_{ij}(a_i)=a_j = f_{jj}(a_j) \Rightarrow a_i\sim a_j \Rightarrow [a_j]=[a_i]$.

Now we check the commutativity of the $h: [a_i] \mapsto f_i(a_i)$. First we have to ensure that it is well defined. That is, if $a_i\sim  a_j \Rightarrow f_i(a_i)=f_j(a_j)$. We suppose by symmetry that  $j>i$, then  by definition $a_i\sim a_j \iff f_{il}(a_i)=f_{jl}(a_j)$ for $i, j\le l$. Setting $l = j$ we get  $f_{ij}(a_i)=f_{jj}(a_j)=a_j$. Now, for the big diagram $f_j(f_{ij}(a_i))=f_i(a_i) \Rightarrow f_j(a_j)= f_i(a_i)$ as desired. 

Note that $h([a_i])=f_i(a_i)$ is the only possible choice we could have done in order to assure the commutativity of the upper diagram.




\prob{8}{Show that in an abelian category we have:
  \begin{enumerate}[topsep=-6pt, itemsep=0pt]
	\item[(a)] $f$ is a monomorphism  $\iff \ker(f)=0$ 
	\item[(b)] $f$ is an epimorphism $\iff \coker(f)=0$ 
	\item[(c)] A monomorphism is the kernel of its cokernel
	\item[(d)] An epimorphism is the cokernel of its kernel
	\item[(e)] Every morphism can be expressed as the composition of an epimorphism and a monomorphism
	\item[(f)] $f$ is an isomorphism  $\iff f$ is an epimorphism and a monomorphism
  \end{enumerate}
  \vspace{1em}
}

We draw our commutative diagram. For being abelian the morphism $\overline{g}$ must be unique for all $g$ and the morphism $\overline{h}$ must be unique for all $h$. Furthermore $\overline{f}$ must be an isomorphism for all  $f$.

% https://q.uiver.app/#q=WzAsOCxbMSwwLCJBIl0sWzIsMCwiQiJdLFszLDAsIlxcdGV4dHtDb2tlcn0iXSxbMiwxLCJcXHRleHR7SW19Il0sWzEsMSwiXFx0ZXh0e0NvaW19Il0sWzAsMCwiXFx0ZXh0e2tlcn0iXSxbMCwxLCJDIl0sWzMsMSwiRCJdLFs1LDAsImkiXSxbMCwxLCJmIl0sWzEsMiwiXFxwaSJdLFszLDEsImoiXSxbMCw0LCJcXHRhdSJdLFs0LDMsIlxcYmFye2Z9Il0sWzYsMCwiZyJdLFs2LDUsIlxcYmFye2d9Il0sWzEsNywiaCJdLFsyLDcsIlxcYmFye2h9Il1d
\[\begin{tikzcd}
	{\text{ker}} & A & B & {\text{Coker}} \\
	C & {\text{Coim}} & {\text{Im}} & D
	\arrow["i", from=1-1, to=1-2]
	\arrow["f", from=1-2, to=1-3]
	\arrow["\pi", from=1-3, to=1-4]
	\arrow["j", from=2-3, to=1-3]
	\arrow["\tau", from=1-2, to=2-2]
	\arrow["{\bar{f}}", from=2-2, to=2-3]
	\arrow["g", from=2-1, to=1-2]
	\arrow["{\bar{g}}", from=2-1, to=1-1]
	\arrow["h", from=1-3, to=2-4]
	\arrow["{\bar{h}}", from=1-4, to=2-4]
\end{tikzcd}\]

\textbf{(a)}

\fbox{$\Rightarrow$} If $f$ is mono, that means  $f\circ k =f\circ l \Rightarrow k=l$. Let $k = i, l = 0$, then $0=f\circ i=f\circ 0\Rightarrow i=0$. Since there exists a unique morphism $\overline{g}:C\to \ker f$, then $\ker f$ is the unique terminal element of the abelian category $0$.

\textbf{(b)}

\fbox{$\Rightarrow$} If $f$ is mono, that means  $k\circ f =l\circ f \Rightarrow k=l$. Let $k = \pi, l = 0$, then $0=\pi\circ f=0\circ f\Rightarrow \pi=0$. Since there exists a unique morphism $\overline{h}:\coker f\to D$, then $\coker f$ is the unique initial element of the abelian category $0$.

\end{document}

