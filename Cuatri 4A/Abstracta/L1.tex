\documentclass[leqno]{article}
\usepackage{verbatim}
\usepackage{array}
\usepackage{listings}
\usepackage{fancyvrb}
\usepackage{enumitem}

\usepackage[utf8]{inputenc}
\usepackage[T1]{fontenc}
\usepackage{textcomp}
\usepackage{multicol} \usepackage{mathtools}
\usepackage{amsmath}
\usepackage{wrapfig}
\usepackage{amssymb}
\usepackage{amsmath,amsfonts,amssymb,amsthm,epsfig,epstopdf,titling,url,array}
\usepackage{hyperref}
\usepackage{eso-pic}
\usepackage{pgf}
\usepackage{tikz}
\usepackage{tikz-cd}
\usepackage{graphicx}

% figure support
\usepackage{import}
\usepackage{xifthen}
\pdfminorversion=7
\usepackage{pdfpages}
\usepackage{transparent}
\usepackage{xcolor}

% geometry
\usepackage{geometry}
\geometry{a4paper, margin=1in}

% paragraph length
\setlength{\parindent}{0em}
\setlength{\parskip}{1em}

\newtheorem*{theorem}{Theorem}
\newtheorem*{lemma}{Lemma}
\newtheorem*{proposition}{Proposition}
\newtheorem*{definition}{Definition}
\newtheorem*{observation}{Observation}

\DeclareMathOperator{\Hom}{Hom}
\newcommand{\prob}[2]{\fbox{\parbox{\textwidth}{\textbf{#1} #2}}}
\newcommand{\incfig}[1]{%
\center
\def\svgwidth{0.9\columnwidth}
\import{./figures/}{#1.pdf_tex}
}
\newcommand{\incimg}[1]{%
\center
\includegraphics[width=0.9\columnwidth]{images/#1}
}
\pdfsuppresswarningpagegroup=1

\title{Problems Abstract Algebra \\ \Large{First List}}
\author{Abel Doñate Muñoz}
\date{}

\begin{document}
\maketitle
\prob{1}{Let $f$ be a morphism in a category  $\mathcal{C}$. Prove the following:
  \begin{enumerate}[topsep=-6pt, itemsep=0pt]
	\item[(a)] If $f$ an isomorphism then $f$ is a monomorphism and an epimorphism.
	\item[(b)] The inclusion of $\mathbb{Z}$ in $\mathbb{Q}$ is a monomorphism and an epimorphism in the category of rings but not an isomorphism.
  \end{enumerate}\vspace{0.5em}
}

We begin with the proof of $(a)$. Since  $f:A\to B$ is an isomorphism, that means there exist  $g:B \to A$ such that both $g \circ f = Id_A $ and $f\circ g = Id_B$. 

Let $h, k$ morphisms of the category that fulfill $f\circ h=f\circ k$. Then by composing from the left with $g$ we have
\[
g\circ f\circ h = g\circ f \circ k \Rightarrow Id_A\circ h = Id_A\circ k \Rightarrow h=k
\]
so we conclude $f$ is a monomorphism.

Let $h, k$ morphisms of the category that fulfill $h \circ f=k\circ f$. Then by composing from the right with $g$ we have
\[
h\circ f\circ g=k\circ f\circ g \Rightarrow h\circ Id_B = k\circ Id_B \Rightarrow h=k
\]
so we conclude $f$ is an epimorphism.

We move to the proof of $(b)$. Let $i:\mathbb{Z} \to \mathbb{Q}$ the inclusion in $\mathbb{Q}$ ($i:n \mapsto n$). Now let $h, k \in \Hom_{rings}(A, \mathbb{Z})$ such that  $i\circ h = i\circ k$. It is clear that, since $i(n)=n\ \forall n\in \mathbb{Z}$, then $h(a)=k(a)\ \forall a\in A$, concluding $h=k$ and  $i$ monomorphism.

Now let $h, k \in \Hom_{rings}(\mathbb{Z}, A)$ such that  $h\circ i = k\circ i$. It is clear that, since $i(n)=n\ \forall n\in \mathbb{Z}$, then $h(i(a))=k(i(a)) \Rightarrow  h(a)=k(a)\ \forall a\in A$, concluding $h=k$ and  $i$ epimorphism.

Suppose $i$ is an isomorphism. Thus, it must exists  $g:\mathbb{Q}\to \mathbb{Z}$ such that $i\circ g = Id_\mathbb{Q}$ and $g\circ i = Id_\mathbb{Z}$. Let $a\in \mathbb{Z}$ such that $g(\frac{1}{2})=a$. Then $i\circ g (\frac{1}{2}) = i(a)=a \neq \frac{1}{2}$, so $i\circ g \neq Id_{\mathbb{Q}}$, concluding $f$ is not an isomorphism.



\prob{4}{
\textit{Pullbacks in the category of abelian groups:} Let $A$ and  $B$ be abelian groups together with homomorphisms $f:A\to S$ and $g:B\to S$. Prove that 
\[
  A \times_SB = \{(a,b)\in A\times B | f(a)=g(b)\} 
\] 
}

Let $U=\{(a,b)\in A\times B|f(a)=g(b)\}$. We will show that the pullback $A\times _SB$ is, in fact, $U$. We construct the following diagram:

(insert diagram)

We first construct the morphisms $\pi_A$ and $\pi_B$ that make the square commute. Those are
\[
  \begin{cases}
    \pi_A((a,b))=a\\
    \pi_B((a,b))=b
  \end{cases} \Rightarrow f\circ \pi_A ((a,b)) = f(a) = g(b) = g\circ \pi_B((a,b)) \ \forall (a,b)\in U
\]
thus, the square commutes.

Now we construct $h$ from  $h_A$ and  $h_B$. Note that, for the two triangular diagrams to commute, the $h$ must fulfill:
\[
\begin{cases}
  \pi_A(h(c))=h_A(c)\\
  \pi_B(h(c))=h_B(c)
\end{cases} \ \forall c\in C \quad \Rightarrow \quad h = (h_A, h_B)
\]
and the $h$ is unique, concluding the proof.


\prob{5}{
\textit{Pushouts in the category of abelian groups:} Let $A$ and  $B$ be abelian groups together with homomorphisms $f:S\to A$ and $g:S\to B$. Prove that 
\[
A \sqcup_SB = \frac{A \oplus B}{W}
\] 
where $W$ is the subgroup generated by $(f(s), -g(s))$ with  $s\in S$.
}

Let $U=\frac{A\oplus B}{W}$. We will show that the pushout $A\sqcup_S B $ is, in fact,  $U$. We construct the following diagram:

(insert diagram)

We first construct the morphisms $i, j$ such that the  square diagram commutes. We propose 
 \[
  i(a)=[(a,0)] \qquad  j(b)=[(0,b)]
\] 
and we check for commutativity for all $s\in S$
\[
  \begin{cases}
  i\circ f (s) = [(f(s),0)]\\
  j\circ g (s) = [(0,g(s))]
  \end{cases} \text{ but } 
  [(0,g(s))]=[(0,g(s))+(f(s),-g(s))]=[(f(s),0)] \Rightarrow i\circ f = j\circ g \ \forall s\in S
\] 
so we have proved the square commutes.

Now we construct the morphism $h$ through  $h_A$ and  $h_B$. We construct it in the following way:
\[
  \begin{cases}
	h([(a,0)])=h_A(a)\\
	h([(0,b)])=h_B(b)
  \end{cases} \Rightarrow h([(a,b)])=h([(a,0)]+[(0,b)])=h([(a,0)])+h([(0,b)]) = h_A(a)+h_B(b)
\] 
and clearly this is well defined and unique as morphism in the category of abelian groups, so the other triangular diagrams commute as well and we conclude with the proof.

\end{document}

