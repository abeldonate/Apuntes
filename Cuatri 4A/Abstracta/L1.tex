\documentclass[leqno]{article}
\usepackage{verbatim}
\usepackage{array}
\usepackage{listings}
\usepackage{fancyvrb}
\usepackage{enumitem}

\usepackage[utf8]{inputenc}
\usepackage[T1]{fontenc}
\usepackage{textcomp}
\usepackage{multicol} \usepackage{mathtools}
\usepackage{amsmath}
\usepackage{wrapfig}
\usepackage{amssymb}
\usepackage{amsmath,amsfonts,amssymb,amsthm,epsfig,epstopdf,titling,url,array}
\usepackage{hyperref}
\usepackage{eso-pic}
\usepackage{pgf}
\usepackage{tikz}
\usepackage{tikz-cd}
\usepackage{graphicx}

% figure support
\usepackage{import}
\usepackage{xifthen}
\pdfminorversion=7
\usepackage{pdfpages}
\usepackage{transparent}
\usepackage{xcolor}

% geometry
\usepackage{geometry}
\geometry{a4paper, margin=1in}

% paragraph length
\setlength{\parindent}{0em}
\setlength{\parskip}{1em}

\newtheorem*{theorem}{Theorem}
\newtheorem*{lemma}{Lemma}
\newtheorem*{proposition}{Proposition}
\newtheorem*{definition}{Definition}
\newtheorem*{observation}{Observation}

\DeclareMathOperator{\Hom}{Hom}
\newcommand{\prob}[2]{\fbox{\parbox{\textwidth}{\textbf{#1} #2}}}
\newcommand{\incfig}[1]{%
\center
\def\svgwidth{0.9\columnwidth}
\import{./figures/}{#1.pdf_tex}
}
\newcommand{\incimg}[1]{%
\center
\includegraphics[width=0.9\columnwidth]{images/#1}
}
\pdfsuppresswarningpagegroup=1

\title{Problems Abstract Algebra \\ \Large{First List}}
\author{Abel Doñate Muñoz}
\date{}

\begin{document}
\maketitle
\prob{1}{Let $f$ be a morphism in a category  $\mathcal{C}$. Prove the following:
  \begin{enumerate}[topsep=-6pt, itemsep=0pt]
	\item[(a)] If $f$ an isomorphism then $f$ is a monomorphism and an epimorphism.
	\item[(b)] The inclusion of $\mathbb{Z}$ in $\mathbb{Q}$ is a monomorphism and an epimorphism in the category of rings but not an isomorphism.
  \end{enumerate}\vspace{0.5em}
}

We begin with the proof of $(a)$. Since  $f:A\to B$ is an isomorphism, that means there exist  $g:B \to A$ such that both $g \circ f = Id_A $ and $f\circ g = Id_B$. 

Let $h, k$ morphisms of the category that fulfill $f\circ h=f\circ k$. Then by composing from the left with $g$ we have
\[
g\circ f\circ h = g\circ f \circ k \Rightarrow Id_A\circ h = Id_A\circ k \Rightarrow h=k
\]
so we conclude $f$ is a monomorphism.

Let $h, k$ morphisms of the category that fulfill $h \circ f=k\circ f$. Then by composing from the right with $g$ we have
\[
h\circ f\circ g=k\circ f\circ g \Rightarrow h\circ Id_B = k\circ Id_B \Rightarrow h=k
\]
so we conclude $f$ is an epimorphism.

We move to the proof of $(b)$. Let $i:\mathbb{Z} \to \mathbb{Q}$ the inclusion in $\mathbb{Q}$ ($i:n \mapsto n$). Now let $h, k \in \Hom_{rings}(A, \mathbb{Z})$ such that  $i\circ h = i\circ k$. It is clear that, since $i(n)=n\ \forall n\in \mathbb{Z}$, then $h(a)=k(a)\ \forall a\in A$, concluding $h=k$ and  $i$ monomorphism.

Now let $h, k \in \Hom_{rings}(\mathbb{Z}, A)$ such that  $h\circ i = k\circ i$. It is clear that, since $i(n)=n\ \forall n\in \mathbb{Z}$, then $h(i(a))=k(i(a)) \Rightarrow  h(a)=k(a)\ \forall a\in A$, concluding $h=k$ and  $i$ epimorphism.

Suppose $i$ is an isomorphism. Thus, it must exists  $g:\mathbb{Q}\to \mathbb{Z}$ such that $i\circ g = Id_\mathbb{Q}$ and $g\circ i = Id_\mathbb{Z}$. Let $a\in \mathbb{Z}$ such that $g(\frac{1}{2})=a$. Then $i\circ g (\frac{1}{2}) = i(a)=a \neq \frac{1}{2}$, so $i\circ g \neq Id_{\mathbb{Q}}$, concluding $f$ is not an isomorphism.


\prob{5}{
\textit{Pushouts in the category of abelian groups:} Let $A$ and  $B$ be abelian groups together with homomorphisms $f:S\to A$ and $g:S\to B$. Prove that 
\[
A \sqcup_SB = \frac{A \oplus B}{W}
\] 
where $W$ is the subgroup generated by $(f(s), -g(s))$ with  $s\in S$.
}

Let $U=\frac{A\oplus B}{W}$. We will show that the pushout $A\sqcup_S B $ is, in fact,  $U$. We construct the following diagram:

(insert diagram)

in 


\end{document}
