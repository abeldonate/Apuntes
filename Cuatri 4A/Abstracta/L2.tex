\documentclass[leqno]{article}
\usepackage{verbatim}
\usepackage{array}
\usepackage{listings}
\usepackage{fancyvrb}
\usepackage{enumitem}

\usepackage[utf8]{inputenc}
\usepackage[T1]{fontenc}
\usepackage{textcomp}
\usepackage{multicol} \usepackage{mathtools}
\usepackage{amsmath}
\usepackage{wrapfig}
\usepackage{amssymb}
\usepackage{amsmath,amsfonts,amssymb,amsthm,epsfig,epstopdf,titling,url,array}
\usepackage{hyperref}
\usepackage{eso-pic}
\usepackage{pgf}
\usepackage{tikz}
\usepackage{tikz-cd}
\usepackage{graphicx}

% figure support
\usepackage{import}
\usepackage{xifthen}
\pdfminorversion=7
\usepackage{pdfpages}
\usepackage{transparent}
\usepackage{xcolor}

% geometry
\usepackage{geometry}
\geometry{a4paper, margin=1in}

% paragraph length
\setlength{\parindent}{0em}
\setlength{\parskip}{1em}

\newtheorem*{theorem}{Theorem}
\newtheorem*{lemma}{Lemma}
\newtheorem*{proposition}{Proposition}
\newtheorem*{definition}{Definition}
\newtheorem*{observation}{Observation}
\DeclareMathOperator{\im}{Im}

\DeclareMathOperator{\Hom}{Hom}
\newcommand{\prob}[2]{\fbox{\parbox{\textwidth}{\textbf{#1} #2}}}
\newcommand{\incfig}[1]{%
\center
\def\svgwidth{0.9\columnwidth}
\import{./figures/}{#1.pdf_tex}
}
\newcommand{\incimg}[1]{%
\center
\includegraphics[width=0.9\columnwidth]{images/#1}
}
\pdfsuppresswarningpagegroup=1

\title{Problems Abstract Algebra \\ \Large{Second List}}
\author{Abel Doñate Muñoz}
\date{}

\begin{document}
\maketitle
\prob{1}{\textit{Nakayama's lemma}. Let $M$ be a finitely generated  $A-$module and  $I$ an ideal of $A$ contained in the  Jacobson  radical. Prove:
\[
IM=M \Rightarrow M=0
\] 
}

First we prove a characterization of the elements of $J$, the Jacobson radical:  $x\in J \iff 1-xy$ is a unity for all  $y\in A$. 

(prove it)

We suppose $M\neq 0$. Let  $x_1, x_2,\ldots,x_n$ be a minimal set of generators of the module  $M$. Because  $M=IM$ we can express the element  $x_1= a_1x_1 + a_2x_2 + \cdots + a_nx_n$, where $a_i \in I$. Then let $b$ the inverse of $1-a_1$ (that we have previously seen that exists).
\[
  (1-a_1)x_1 = a_2x_2+\cdots + a_nx_n = 0 \Rightarrow b(a_1-1)x_1= x_1 = ba_2x_2 + \cdots ba_nx_n 
\]
entering in contradiction with $\{x_i\}$ being a minimal set unless  $x_i=0$, thus  $M=0$


(rehacer)


\prob{2}{Under the previous hypothesis, prove:
  \begin{enumerate}[topsep=-6pt, itemsep=0pt]
    \item $A / I \otimes_A M =0 \Rightarrow M=0$
	\item If  $N\subseteq M$ is a submodule, $M=IM+N \Rightarrow M=N$
	\item If $f:N\to M$ is a homomorphism, $\overline{f}:N / IN \to M / IM$ surjective $\Rightarrow f $ surjective
  \end{enumerate}
  \vspace{1em}
}

\prob{3}{Let $A$ be a non-local ring. Prove that the  $A-$module  $A$ has two minimal system of generators with a different number of generators.
}

\prob{4}{Let (diagram) be a short exact sequence of $A-modules$. Prove that if  $M'$ and  $M''$ are finitely generated, then  $M$ is finitely generated.}

We start by fixing the set of generators of $M'$ as  $x_1, \ldots, x_n$ and of $M''$ as  $z_1, \ldots, z_m$.

Since $g$ is surjective, we can find elements  $y_1, \ldots, y_m$ such that $g(y_i)=z_i$. Now we select an arbitrary element $y\in M$. Then we have
\[
g(y)=b_1z_1+\cdots +b_mz_m = g(b_1y_1)+ \cdots + g(b_my_m) \Rightarrow g(y-\sum b_iy_i) =0 \Rightarrow y-\sum  b_iy_i \in \ker(g)
\]
for some $b_i\in A$. By exactness of the sequence we have $y-\sum b_iy_i\in \im (f)$, so
 \[
y-\sum b_iy_i = f(\sum a_ix_i) = \sum a_if(x_i) \Rightarrow y = \sum a_if(x_i) + \sum b_iy_i
\] 
for some $a_i\in A$. Thus, a set of generators of $M$ is  $f(x_1), \ldots, f(x_n), y_1, \ldots, y_m$

\prob{5}{Prove that $\mathbb{Z}[\sqrt{d}]$ is a Noetherian ring
}

This is equivalent to prove that $M=\mathbb{Z}[\sqrt{d} ]$ is a Noetherian module. Since every submodule of  $M$ is finitely generated (by $1$ and $\sqrt{d} $), then the module is Noetherian.


\prob{6}{Prove that the ring $\mathbb{Z}[2T, 2T^2, 2T^3, \ldots] \subseteq \mathbb{Z}[T]$ is not Noetherian}

We search for an ascending chain of ideals $I_1\subseteq I_2\subseteq \ldots$ in which for every $I_i$ we have  $x_i\in I_i$ but $x_i \notin I_{i-1}$. This chain can be $I_i = (2T, 2T^2, \ldots, 2T^{i-1}, 2T^i+2T^{i+1}+\ldots)$. Notice that the containments are obvious and $x_i = 2T^{i-1}\in I_i$, but not in $I_{i-1}$.

\prob{7}{ Let $M$ be an  $A-$module and let  $N_1, N_2$ be submodules of $M$. Prove that if  $M / N_1$ and  $M/ N_2$ are Noetherian (Artinian), then $M / (N_1 \cap N_2)$ is Noetherian (Artinian) as well.
}

\prob{8}{Let $M$ be an  $A-$module, $f:M\to N$ an $A-$endomorphism. Prove:
  \begin{enumerate}[topsep=-6pt, itemsep=0pt]
    \item If $M$ is Noetherian and  $f$ surjective  $\Rightarrow$ $f$ isomorphism
    \item If $M$ is Artinian and  $f$ injective $\Rightarrow$ $f$ isomorphism
  \end{enumerate}
\vspace{1em}
}

\prob{9}{Compute:
\begin{enumerate}[topsep=-6pt, itemsep=0pt]
  \item $\Hom_{\mathbb{Z}}(\mathbb{Q}, \mathbb{Z})$ 
\item $\Hom_{\mathbb{Z}}(\mathbb{Q}, \mathbb{Q})$
\item $\Hom_{\mathbb{Z}}(\mathbb{Z} / (m), \mathbb{Q})$
\end{enumerate}
\vspace{1em}
}
\textbf{(1)} We look for an element in $\Hom_{\mathbb{Z}}(\mathbb{Q}, \mathbb{Z})$. Let $f(\frac{1}{n})=x_n$ for $n$ a nonzero integer and $f(1)=C \in \mathbb{Z}$. Then we have
\[
C=f(1) = f\left(\frac{n}{n}\right) = nf\left(\frac{1}{n}\right) = nx_n.  \quad \Rightarrow \quad
x_n = 0\ \forall |n|>C
\]
But if we take into account $C = nx_n$ holds for all nonzero $n$, then $C = 0$, meaning all the  $x_n$ are zero. We end up with $f\left(\frac{a}{b}\right)=af\left( \frac{1}{b} \right)=a\times 0 = 0$. So $\Hom_{\mathbb{Z}}(\mathbb{Q}, \mathbb{Z})=0$

\textbf{(2)} We look for an element in $\Hom_{\mathbb{Z}}(\mathbb{Q}, \mathbb{Q})$. Let $f(\frac{1}{n})=\frac{x_n}{y_n}$ for $n$ a nonzero integer and $f(1)=C\in \mathbb{Q}$. Then we have 
\[
C=f(1) = f\left(\frac{n}{n}\right) = nf\left(\frac{1}{n}\right) = n \frac{x_n}{y_n}.  \quad \Rightarrow \quad
\frac{x_n}{y_n} = \frac{C}{n}
\]
That means our morphism $f_C$ is uniquely determined by the choice of  $C\in \mathbb{Q}$, and is the morphism that sends $1\to C$ and $\frac{1}{n}\to \frac{c}{n}$ and extends linearly $\frac{a}{b}\to \frac{a}{b}C$. So $\Hom_{\mathbb{Z}}(\mathbb{Q}, \mathbb{Q})\simeq \mathbb{Q}$

\textbf{(3)} We look for an element in $\Hom_{\mathbb{Z}}(\mathbb{Z} / (m), \mathbb{Q})$. Let $f(\overline{1})=r\in \mathbb{Q}$. Then 
\[0=f(\overline{0})=f(\overline{m})=mf(\overline{1})=mr \Rightarrow r=0\]
So the only possibility is the morphism $0$ and  $\Hom_{\mathbb{Z}}(\mathbb{Z} / (m), \mathbb{Q})=0$


\end{document}
