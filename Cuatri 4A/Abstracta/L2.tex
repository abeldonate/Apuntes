\documentclass[leqno]{article}
\usepackage{verbatim}
\usepackage{array}
\usepackage{listings}
\usepackage{fancyvrb}
\usepackage{enumitem}

\usepackage[utf8]{inputenc}
\usepackage[T1]{fontenc}
\usepackage{textcomp}
\usepackage{multicol} \usepackage{mathtools}
\usepackage{amsmath}
\usepackage{wrapfig}
\usepackage{amssymb}
\usepackage{amsmath,amsfonts,amssymb,amsthm,epsfig,epstopdf,titling,url,array}
\usepackage{hyperref}
\usepackage{eso-pic}
\usepackage{pgf}
\usepackage{tikz}
\usepackage{tikz-cd}
\usepackage{graphicx}

% figure support
\usepackage{import}
\usepackage{xifthen}
\pdfminorversion=7
\usepackage{pdfpages}
\usepackage{transparent}
\usepackage{xcolor}

% geometry
\usepackage{geometry}
\geometry{a4paper, margin=1in}

% paragraph length
\setlength{\parindent}{0em}
\setlength{\parskip}{1em}

\newtheorem*{theorem}{Theorem}
\newtheorem*{lemma}{Lemma}
\newtheorem*{proposition}{Proposition}
\newtheorem*{definition}{Definition}
\newtheorem*{observation}{Observation}
\DeclareMathOperator{\im}{Im}

\DeclareMathOperator{\Hom}{Hom}
\newcommand{\prob}[2]{\fbox{\parbox{\textwidth}{\textbf{#1} #2}}}
\newcommand{\incfig}[1]{%
\center
\def\svgwidth{0.9\columnwidth}
\import{./figures/}{#1.pdf_tex}
}
\newcommand{\incimg}[1]{%
\center
\includegraphics[width=0.9\columnwidth]{images/#1}
}
\pdfsuppresswarningpagegroup=1

\title{Problems Abstract Algebra \\ \Large{Second List}}
\author{Abel Doñate Muñoz}
\date{}

\begin{document}
\maketitle
\prob{1}{\textit{Nakayama's lemma}. Let $M$ be a finitely generated  $A-$module and  $I$ an ideal of $A$ contained in the  Jacobson  radical. Prove:
\[
IM=M \Rightarrow M=0
\] 
}

First we prove a characterization of the elements of $J$, the Jacobson radical:  $x\in J \iff 1-xy$ is a unity for all  $y\in A$. 

(prove it)

We suppose $M\neq 0$. Let  $x_1, x_2,\ldots,x_n$ be a minimal set of generators of the module  $M$. Because  $M=IM$ we can express the element  $x_1= a_1x_1 + a_2x_2 + \cdots + a_nx_n$, where $a_i \in I$. Then let $b$ the inverse of $1-a_1$ (that we have previously seen that exists).
\[
  (1-a_1)x_1 = a_2x_2+\cdots + a_nx_n = 0 \Rightarrow b(a_1-1)x_1= x_1 = ba_2x_2 + \cdots ba_nx_n 
\]
entering in contradiction with $\{x_i\}$ being a minimal set unless  $x_i=0$, thus  $M=0$


(rehacer)


\prob{2}{Under the previous hypothesis, prove:
  \begin{enumerate}[topsep=-6pt, itemsep=0pt]
    \item $A / I \otimes_A M =0 \Rightarrow M=0$
	\item If  $N\subseteq M$ is a submodule, $M=IM+N \Rightarrow M=N$
	\item If $f:N\to M$ is a homomorphism, $\overline{f}:N / IN \to M / IM$ surjective $\Rightarrow f $ surjective
  \end{enumerate}
  \vspace{1em}
}

\prob{3}{Let $A$ be a non-local ring. Prove that the  $A-$module  $A$ has two minimal system of generators with a different number of generators.
}

\prob{4}{Let (diagram) be a short exact sequence of $A-modules$. Prove that if  $M'$ and  $M''$ are finitely generated, then  $M$ is finitely generated.}

We start by fixing the set of generators of $M'$ as  $x_1, \ldots, x_n$ and of $M''$ as  $z_1, \ldots, z_m$.

Since $g$ is surjective, we can find elements  $y_1, \ldots, y_m$ such that $g(y_i)=z_i$. Now we select an arbitrary element $y\in M$. Then we have
\[
g(y)=b_1z_1+\cdots +b_mz_m = g(b_1y_1)+ \cdots + g(b_my_m) \Rightarrow g(y-\sum b_iy_i) =0 \Rightarrow y-\sum  b_iy_i \in \ker(g)
\]
for some $b_i\in A$. By exactness of the sequence we have $y-\sum b_iy_i\in \im (f)$, so
 \[
y-\sum b_iy_i = f(\sum a_ix_i) = \sum a_if(x_i) \Rightarrow y = \sum a_if(x_i) + \sum b_iy_i
\] 
for some $a_i\in A$. Thus, a set of generators of $M$ is  $f(x_1), \ldots, f(x_n), y_1, \ldots, y_m$

\prob{5}{Prove that $\mathbb{Z}[\sqrt{d}]$ is a Noetherian ring
}

This is equivalent to prove that $M=\mathbb{Z}[\sqrt{d} ]$ is a Noetherian module. Since every submodule of  $M$ is finitely generated (by $1$ and $\sqrt{d} $), then the module is Noetherian.


\prob{6}{Prove that the ring $\mathbb{Z}[2T, 2T^2, 2T^3, \ldots] \subseteq \mathbb{Z}[T]$ is not Noetherian}

We search for an ascending chain of ideals $I_1\subseteq I_2\subseteq \ldots$ in which for every $I_i$ we have  $x_i\in I_i$ but $x_i \notin I_{i-1}$. This chain can be $I_i = (2T, 2T^2, \ldots, 2T^{i-1}, 2T^i+2T^{i+1}+\ldots)$. Notice that the containments are obvious and $x_i = 2T^{i-1}\in I_i$, but not in $I_{i-1}$.

\prob{7}{ Let $M$ be an  $A-$module and let  $N_1, N_2$ be submodules of $M$. Prove that if  $M / N_1$ and  $M/ N_2$ are Noetherian (Artinian), then $M / (N_1 \cap N_2)$ is Noetherian (Artinian) as well.
}

\prob{8}{Let $M$ be an  $A-$module, $f:M\to N$ an $A-$endomorphism. Prove:
  \begin{enumerate}[topsep=-6pt, itemsep=0pt]
    \item If $M$ is Noetherian and  $f$ surjective  $\Rightarrow$ $f$ isomorphism
    \item If $M$ is Artinian and  $f$ injective $\Rightarrow$ $f$ isomorphism
  \end{enumerate}
\vspace{1em}
}

\prob{9}{Compute:
\begin{enumerate}[topsep=-6pt, itemsep=0pt]
  \item $\Hom_{\mathbb{Z}}(\mathbb{Q}, \mathbb{Z})$ 
\item $\Hom_{\mathbb{Z}}(\mathbb{Q}, \mathbb{Q})$
\item $\Hom_{\mathbb{Z}}(\mathbb{Z} / (m), \mathbb{Q})$
\end{enumerate}
\vspace{1em}
}
\textbf{(1)} We look for an element in $\Hom_{\mathbb{Z}}(\mathbb{Q}, \mathbb{Z})$. Let $f(\frac{1}{n})=x_n$ for $n$ a nonzero integer and $f(1)=C \in \mathbb{Z}$. Then we have
\[
C=f(1) = f\left(\frac{n}{n}\right) = nf\left(\frac{1}{n}\right) = nx_n.  \quad \Rightarrow \quad
x_n = 0\ \forall |n|>C
\]
But if we take into account $C = nx_n$ holds for all nonzero $n$, then $C = 0$, meaning all the  $x_n$ are zero. We end up with $f\left(\frac{a}{b}\right)=af\left( \frac{1}{b} \right)=a\times 0 = 0$. So $\Hom_{\mathbb{Z}}(\mathbb{Q}, \mathbb{Z})=0$

\textbf{(2)} We look for an element in $\Hom_{\mathbb{Z}}(\mathbb{Q}, \mathbb{Q})$. Let $f(\frac{1}{n})=\frac{x_n}{y_n}$ for $n$ a nonzero integer and $f(1)=C\in \mathbb{Q}$. Then we have 
\[
C=f(1) = f\left(\frac{n}{n}\right) = nf\left(\frac{1}{n}\right) = n \frac{x_n}{y_n}.  \quad \Rightarrow \quad
\frac{x_n}{y_n} = \frac{C}{n}
\]
That means our morphism $f_C$ is uniquely determined by the choice of  $C\in \mathbb{Q}$, and is the morphism that sends $1\to C$ and $\frac{1}{n}\to \frac{c}{n}$ and extends linearly $\frac{a}{b}\to \frac{a}{b}C$. So $\Hom_{\mathbb{Z}}(\mathbb{Q}, \mathbb{Q})\simeq \mathbb{Q}$

\textbf{(3)} We look for an element in $\Hom_{\mathbb{Z}}(\mathbb{Z} / (m), \mathbb{Q})$. Let $f(\overline{1})=r\in \mathbb{Q}$. Then 
\[0=f(\overline{0})=f(\overline{m})=mf(\overline{1})=mr \Rightarrow r=0\]
So the only possibility is the morphism $0$ and  $\Hom_{\mathbb{Z}}(\mathbb{Z} / (m), \mathbb{Q})=0$

\prob{10}{
Let $A$ be a ring,  $M$ an  $A-$module and  $I\subseteq A$ an ideal. Prove 
\[
  M / IM \cong A / I \otimes _A M
\] 
}


We construct the following maps and see that they are well-defined

\begin{minipage}{0.5\textwidth}
\begin{align*}
  f: M / IM &\to  A / I \otimes _AM \\
  x+IM &\mapsto  (1+I)\otimes_A x
\end{align*}
\end{minipage}
\begin{minipage}{0.5\textwidth}
\begin{align*}
  g: A / I\otimes _A M &\to M / IM\\
  (a+I) \otimes _A y &\mapsto ay+IM
\end{align*}
\end{minipage}

If we pick $x'\sim_{IM} x \Rightarrow x'= x+n$ for $n\in IM$ and we have
\[
f(x'+IM) = (1+I)\otimes_A (x+n)) = (1+I)\otimes _A x + (1+I) \otimes _A n = (1+I) \otimes _A x = f(x+I)
\] 
Since the second term of the sum vanishes as $n= \sum i_km_k$ for $i_k\in I, m_k\in M$, so
\[
  (1+I)\otimes _A \sum i_km_k = \sum (i_k+I) \otimes _A m_k = \sum (0+I) \otimes _A m_k = 0
\] 
Therefore the application $f$ is well-defined.

If we pick $a'\sim_A a \Rightarrow a'= a+i$ for $i\in I$ and we have
\[
g((a'+I)\otimes _A y) = (a+i)y +IM = ay + iy + IM = ay + IM = g((a+I)\otimes _A y)
\] 
proving that the application is well defined.

It is trivial to check that $f$ and  $g$ satisfy all the necessary conditions in order to be morphisms of modules, since only quotients, multiplications and tensor products are involved.

Finally we see $f\circ g = Id$ and $g\circ f = Id$:
\begin{align*}
& f(g((a+I)\otimes _Ay)) = f(ay+IM) = (1+I)\otimes _A ay = (a+I)\otimes _A y &\Rightarrow \quad f\circ g = Id\\
& g(f(x+IM)) = g((1+I)\otimes _A x) = x+IM &\Rightarrow \quad g\circ f = Id
\end{align*}
finishing the proof.


\prob{11}{Let $A$ be a ring and $I, J\subseteq A$ ideals. Prove
  \[
  A /I \otimes _A A / J \cong A / (I+J)
  \] 
}

We construct the following maps and see that they are well-defined

\begin{minipage}{0.5\textwidth}
\begin{align*}
  f: A / I \otimes _A A / J &\to A / (I+J)\\
  (x+I)\otimes _A (y+J) &\mapsto xy + (I+J)
\end{align*}
\end{minipage}
\begin{minipage}{0.5\textwidth}
\begin{align*}
  g: A / (I+J) &\to  A / I \otimes _A A / J\\ 
  z+(I+J) &\mapsto (z+I)\otimes _A (1+J) 
\end{align*}
\end{minipage}

If we pick $x'\sim x$ and $y'\sim y $ that means $x' = x+i, y'=y+j$ with  $i\in I, j\in J$ and we have 
\[
f((x'+I)\otimes (y'+J)) = x'y' + (I+J) = xy + iy'+x'j+ij +I+J = xy + (I+J) = f((x+I)\otimes (y+J))
\]
so $f$ is well-defined.

If we pick $z'\sim z$  that means $z'=z+i+j$ with  $i\in I, j\in J$ and we have
\begin{align*}
&g(z'+(I+J))= (z'+I)\otimes (1+J) = (z+i+j +I)\otimes (1+J) = z\otimes_A (1+J) + (i+I) \otimes (1+J) +\\
&+ (j+I) \otimes_A (1+J) = (z+I)\otimes _A (1+J) + (0+I)\otimes _A (1+J) + (1+I)\otimes _A (0+J)  = g(z+(I+J))
\end{align*}

It is trivial to check that $f$ and  $g$ satisfy all the necessary conditions in order to be morphisms of modules, since only quotients, multiplications and tensor products are involved.

Finally we see $f\circ g = Id$ and $g\circ f = Id$:
\begin{align*}
  &f(g(z+(I+J))) = f((z+I)\otimes _A (1+J)) = z+(I+J) & \Rightarrow \quad f\circ g = Id\\
  &g(f((x+I)\otimes _A (y+J))) = g(xy + (I+J)) = (xy+I) \otimes (1+J) = (x+I) \otimes _A (y+J)& \Rightarrow \quad g\circ f = Id
\end{align*}
finishing the proof.



\prob{12}{
Let $A$ be a ring, $M, N$ finitely generated $A-$ modules. Prove:
\begin{enumerate}[topsep=-6pt, itemsep=0pt]
  \item $M\otimes _A N$ is a finitely generated $A-$module 
  \item If $A$ is Noetherian, then $\Hom _A(M, N)$ is a finitely generated $A-$module
\end{enumerate}
\vspace{1em}
}

\prob{13}{
Let $A$ be a local ring,  $M, N$ finitely generated  $A-$modules. Prove that
\[
  M\otimes _A N = 0 \iff (M=0 \quad \text{ó} \quad N=0)
\] 
}

\prob{14}{
Let $M$ be a finitely generated $A-$module and let  $S\subseteq A$ be a multiplicatively closed set. Prove that 
\[
S^{-1}M=0 \iff \ \exists s\in S : sM=0
\] 
}

\prob{15}{
Let $S\subseteq  A$ be a multiplicatively closed set. Prove that the localization functor $S^{-1}-$ is exact.
}


\end{document}
