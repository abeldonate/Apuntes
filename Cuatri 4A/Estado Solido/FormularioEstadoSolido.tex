\documentclass[leqno]{article}
\usepackage{verbatim}
\usepackage{array} \usepackage{listings}
\usepackage{fancyvrb}
\usepackage{enumitem}
\usepackage{multicol}
\usepackage[utf8]{inputenc}
\usepackage[T1]{fontenc}
\usepackage{textcomp}
\usepackage{multicol}
\usepackage{mathtools}
\usepackage{amsmath}
\usepackage{wrapfig}
\usepackage{amssymb}
\usepackage{amsmath,amsfonts,amssymb,amsthm,epsfig,epstopdf,titling,url,array}
\usepackage{makecell}
\usepackage{hyperref}
\usepackage{eso-pic}
\usepackage{pgf}
\usepackage{tikz}
\usepackage{graphicx}
\usepackage{cancel}
\usepackage{svg}

% figure support
\usepackage{import}
\usepackage{xifthen}
\pdfminorversion=7
\usepackage{pdfpages}
\usepackage{transparent}
\usepackage{xcolor}

% geometry
\usepackage{geometry}
\geometry{a4paper, margin=0.5in}

% paragraph length
\setlength{\parindent}{0em}
\setlength{\parskip}{1em}

\newtheorem*{theorem}{Theorem}
\newtheorem*{lemma}{Lemma}
\newtheorem*{proposition}{Proposition}
\newtheorem*{definition}{Definition}
\newtheorem*{observation}{Observation}

\newcommand{\incfig}[1]{%
\begin{center}
\def\svgwidth{0.9\columnwidth}
\import{./figures/}{#1.pdf_tex}
\end{center}
}
\newcommand{\incimg}[1]{%
\center
\includegraphics[width=0.9\columnwidth]{images/#1}
}
\pdfsuppresswarningpagegroup=1

\title{Formulario Estadística}
\author{Abel Doñate Muñoz}
\date{}

\begin{document}

\begin{multicols}{2}[\columnsep2em]

\textbf{Constantes}
\begin{align*}
&k_B = 1.381\times 10^{-23}JK^{-1} = 8.26\times 10^{-5}eVK^{-1}\\
&m_e = 9.11\times 10^{-31}kg = 0.511MeVc^{-2}\\
&\varepsilon _0 = \frac{1}{4\pi K} = 8.85\times 10^{-12} Fm^{-1}\\
&\hbar = 1.055\times 10^{-34} Js = 6.58\times 10^{-16}eVs \\
&e = 1.602 \times 10^{-19} C
\end{align*}



\section{Estructura cristalina}
\subsection{Redes de Bravais}
\begin{minipage}{0.45\columnwidth}
\begin{center}
\begin{tabular}{|l|l|}
\hline
$a$ & triclínica  \\ \hline
$m$ & monoclínica\\ \hline
$o$ & ortorómbica\\ \hline
$t$ & tetragonal \\ \hline
$h$ & hexagonal\\ \hline
$c$ & cúbica \\ \hline
\end{tabular}
\end{center}
\end{minipage}
\begin{minipage}{0.45\columnwidth}
\begin{center}
\begin{tabular}{|l|l|}
\hline
$P$ & Primitiva \\\hline
$S$ & Centrada en una cara\\\hline
$I$ & Centrada en el cuerpo\\\hline
$R$ & Centrada romboidal\\\hline
$F$ & Centrada en las caras \\\hline
\end{tabular}
\end{center}
\end{minipage}

14 posibles redes de Bravais
\begin{center}
\begin{tabular}{|c|c|c|c|c|c|}
\hline
Tric. & Monoc. & Ortor. & Tetra. & Hex. & Cúbico \\\hline
$aP$ &  $mP, mS$ &  $oP, oS, oF, oI$ &  $tP, tI$ &  $hP, hR$ &  $cP, cF, cI$ \\\hline
\end{tabular}
\end{center}

\subsection{Cosas}
Base dual y matriz métrica
\begin{align*}
  & a^* = \frac{b\times c}{V}, \quad b^* = \frac{c\times a}{V}  , \quad c^* = \frac{a\times b}{V}, \quad V = \det(\overline{a}, \overline{b}, \overline{c})  \\
  & (\overline{a}^*, \overline{b}^*, \overline{c}^*)=\begin{pmatrix} \overline{a}^T\\\overline{b}^T\\\overline{c}^T \end{pmatrix} ^{-1}, G = \begin{pmatrix} a\cdot a & a\cdot b & a\cdot c \\ b\cdot a & b\cdot b & b\cdot c \\ c\cdot a & c\cdot b & c\cdot c \end{pmatrix}, G^* = G^{-1}
\end{align*}

Cambio de base
\begin{align*}
  (\overline{a}', \overline{b}', \overline{c}') = (\overline{a}, \overline{b}, \overline{c})P, \quad \begin{pmatrix} x' \\ y' \\z' \end{pmatrix} =P^{-1} \begin{pmatrix} x\\y\\z \end{pmatrix} \\
  (x, y, z) = (x^*, y^*, z^*)P, \quad \begin{pmatrix} a'^* \\ b'^* \\z'^* \end{pmatrix} =P^{-1} \begin{pmatrix} a^*\\b^*\\c^* \end{pmatrix} 
\end{align*}

Red recíproca y distancia interplanar $g_{hkl}= \frac{1}{d_{hkl}}$

Transferencia de momento $Q = \frac{4\pi \sin\theta }{\lambda}$

Condiciones de Laue $\overline{Q}=2\pi \overline{g}_{hkl}$

Ley de Bragg $g_{hkl} = \frac{2\sin \theta_{hkl}}{\lambda}$ 

Módulo de Young $\nu_s = \sqrt{\frac{\gamma}{\rho }} $

Factor de estructura
\begin{align*}
  F_{hkl} = \sum_p f_p e^{-i2\pi \overline{g}_{hkl}\cdot \overline{r}_p}, \quad I \propto |F_{hkl}|^2
\end{align*}

\subsection{Estructuras comunes}

FCC
\[
\begin{cases}
  \overline{a} = \frac{1}{2} (1\ 1\ 0) \\
  \overline{b} = \frac{1}{2} (0\ 1\ 1) \\
  \overline{c} = \frac{1}{2} (1\ 0\ 1)
\end{cases} \quad 
\begin{cases}
  \overline{a}^* = (1\ 1\ -1)\\
  \overline{b}^* = (-1\ 1\ 1 )\\
  \overline{c}^* = (1\ -1\ 1)
\end{cases} 
\]
BCC
\[
\begin{cases}
  \overline{a} = \frac{1}{2} (1\ 1\ -1) \\
  \overline{b} = \frac{1}{2} (-1\ 1\ 1) \\
  \overline{c} = \frac{1}{2} (1\ -1\ 1)
\end{cases} \quad 
\begin{cases}
  \overline{a}^* = (1\ 1\ 0)\\
  \overline{b}^* = (0\ 1\ 1 )\\
  \overline{c}^* = (1\ 0\ 1)
\end{cases} 
\]
\begin{minipage}{\columnwidth}
\incfig{fcc}
\end{minipage}

Hexagonal
\[
\begin{cases}
  \overline{a} = (1, 0)\\
  \overline{b} = (-\frac{1}{2}, \frac{\sqrt{3} }{2})
\end{cases} \quad
\begin{cases}
  \overline{a}^* = \frac{2 \sqrt{3} }{3}(\frac{\sqrt{3} }{2}, \frac{1}{2}) \\
  \overline{b}^* = \frac{2 \sqrt{3} }{3}( 0, 1)
\end{cases}
\]
\[
  G = \begin{pmatrix} a^2 & -\frac{a^2}{2} & 0 \\ -\frac{a^2}{2} & a^2 & 0 \\ 0 & 0 & c^2\end{pmatrix} , \quad
  G^* = \begin{pmatrix} \frac{4}{3a^2} & \frac{2}{3a^2} & 0 \\ \frac{2}{3a^2} & \frac{4}{3a^2} & 0 \\ 0 & 0 & \frac{1}{c^2}\end{pmatrix} 
\] 

\begin{minipage}{\columnwidth}
  \incfig{hexagonal}
\end{minipage}

En una hcp $c = 1.633a$

\subsection{Grupos}
\[
  m_{100} = \begin{pmatrix} -1&0&0\\0&1&0\\0&0&1 \end{pmatrix} 
  n_{001} = \begin{pmatrix} \cos\left( \frac{360}{n} \right) &-\sin\left( \frac{360}{n} \right) &0\\ \sin\left( \frac{360}{n} \right) &\cos\left( \frac{360}{n} \right) &0\\0&0&1 \end{pmatrix}
\] 
Cambio de base a $\mathcal{B}=\{\overline{u}, \overline{v}, \overline{w}\}$
\[
  M_{\mathcal{C}} = M_{\mathcal{B}\to \mathcal{C}} M_{\mathcal{B}} M_{\mathcal{B}\to \mathcal{C}}^{-1}, \quad M_{\mathcal{B}\to \mathcal{C}} = (\overline{u}, \overline{v}, \overline{w})
\] 

Reflexión vector director $(a, b, c)$
 \[
   M = \frac{1}{a^2+b^2+c^2} \begin{pmatrix} -a^2 + b^2+c^2 & -2ab & -2ac \\ -2ab & a^2-b^2+c^2 & -2bc \\ -2ac & -2bc & a^2+b^2-c^2 \end{pmatrix} 
\] 
Rotación respecto $\hat{u}=(u_x, u_y, u_z)$ ($c=\cos\theta , \quad s=\sin \theta $). $R = $
\[
\begin{pmatrix} c+u_x^2(1-c) & u_xu_y(1-c)-u_zs & u_xu_z(1-c)+u_ys \\
	u_yu_x(1-c)+u_zs & c+u_y^2(1-c) & u_yu_z(1-c)-u_xs\\
	u_zu_x(1-c)-u_ys & u_zu_y(1-c) + u_xs & c + u_z^2(1-c)
  \end{pmatrix} 
\] 
Centrosimétricos $(x, y, z)\to (-x, -y, -z)$ no tienen polarización espontánea

\section{Dinámica de cristales}
\subsection{Densidad de estados}
\[
  \overline{k} = \begin{pmatrix} \frac{2\pi}{L}n & \frac{2\pi}{L}m & \frac{2\pi}{L}l \end{pmatrix} \ \forall n, m, l \in \mathbb{Z}
\] 
Número de estados hasta $k$
 \[
N(k) = \int_{(\frac{2\pi}{L})^2(n^2 + m^2+l^2)\le k^2} dV =\frac{L^3}{6\pi^2}k^3 = \frac{V}{6\pi^2}k^3
\] 
1, 2 y 3 dimensiones respectivamente (y se cumple $\omega = \nu_s k$)
\begin{align*}
  \begin{cases}
    g(k) = \frac{L}{\pi} \\  g(\omega ) = \frac{L}{\pi \nu} 
  \end{cases}
  \begin{cases}
    g(k) = \frac{L^2}{2\pi}k \\  g(\omega ) = \frac{L^2}{2\pi \nu^2}\omega 
  \end{cases}
  \begin{cases}
    g(k) = \frac{V}{2\pi^2}k^2 \\  g(\omega ) = \frac{V}{2\pi^2 \nu_s^3}\omega ^2
  \end{cases}
\end{align*}

\subsection{Dispersión}
Oscilador con masa $m$ y constante $k_s$
\[
F_n =m\ddot{x}_n = k_s(x_{n+1}+x_{n-1}-2x_n)
\]
\begin{align*}
   &-m\omega ^2 Ae^{i(kna-\omega t)} = k_sAe^{i(kna-\omega t)}(e^{ika}+e^{-ika}-2) =\\
   &=-4k_s \sin^2\left( \frac{ka}{2} \right)  \Rightarrow \boxed{\omega = 2 \sqrt{\frac{k_s}{m}}\left|\sin \left( \frac{ka}{2} \right) \right| }
\end{align*}

Oscilador con masa $m$ y constantes alternadas  $k_1, k_2$
\[
  \begin{cases}
m\ddot{x}_n = k_1(y_{n-1}-x_n)+k_2(y_n-x_n)\\
m\ddot{y}_n = k_1(x_{n+1}-y_n)+k_2(x_n-y_n)
  \end{cases}
\] 
Ansatz
\[
x_n = A e^{i(kna-\omega t)} \qquad
y_n = B e^{i(kna-\omega t)}
\] 
Ecuaciones
\[
  \begin{cases}
-m\omega ^2 A = -A(k_1+k_2)+ B(k_1e^{ika}+k_2)\\
-m\omega ^2 B = -A(k_1e^{ika}+k_2)+ B(-k_1-k_2)
  \end{cases}
\] 
Forma matricial
\[
  m\omega ^2 \begin{pmatrix} A\\B \end{pmatrix} = \begin{pmatrix} (k_1+k_2) & -k_2 -k_1e^{ika} \\ -k_2-k_1e^{ika} & (k_1+k_2) \end{pmatrix}  \begin{pmatrix} A\\B \end{pmatrix} = K\begin{pmatrix} A\\B \end{pmatrix} 
\] 
 \[
0 = \det(K-m\omega ^2I) = |(k_1+k_2)-m\omega ^2|^2 - |k_2 +k_1e^{ika}|^2
\] 
\[
  \boxed{\omega _{\pm}(k) = \sqrt{ \frac{k_1+k_2}{m}\pm \frac{1}{m} \sqrt{(k_1+k_2)^2-4k_1k_2\sin^2(ka / 2)} } }
\] 
Si $m_1\neq m_2$ y $k_s$ es la misma, sea  $K_i = \frac{k}{m_i}$, entonces
\[
  \boxed{\omega _{\pm}(k) = \sqrt{ (K_1+K_2)\pm \sqrt{(K_1+K_2)^2-4K_1K_2\sin^2(ka / 2)} } }
\] 
\begin{minipage}{0.5\columnwidth}
\incfig{dispersion}
\end{minipage}
\begin{minipage}{0.5\columnwidth}
\incfig{dispersion2}
\end{minipage}

\subsection{Modelo de Einstein}
\begin{align*}
  &E_n = \hbar \omega (n+\frac{1}{2}) \quad \Rightarrow \quad Z_1 = \frac{1}{2 \sinh(\frac{\beta \hbar \omega }{2})} \\
  &\langle E_1 \rangle = - \frac{\partial }{\partial \beta } \ln Z_1 = \frac{\hbar\omega }{2} \coth \left( \frac{\beta \hbar \omega }{2} \right) 
\end{align*}
  Energía y capacidad calorífica
\begin{align*}
&\langle E\rangle = \frac{3}{2} N \hbar \omega \coth \left( \frac{\beta \hbar \omega  }{2} \right) \\
&C_v = \frac{\partial \langle E\rangle}{\partial T}  = 3Nk_B (\beta \hbar \omega )^2 \frac{e^{\beta \hbar \omega }}{(e^{\beta \hbar \omega }-1)^2}
\end{align*}

Definimos ahora $T_E = \frac{\hbar \omega_E}{k_B}$. En los límites

\begin{itemize}[topsep=-6pt, itemsep=0pt]
  \item Si  $T\gg T_E \quad \Rightarrow \quad C_v = 3Nk_b$
  \item Si  $T\ll T_E \quad \Rightarrow \quad C_v = 3Nk_b (\frac{T_E}{T})^2 \frac{1}{\sinh ^2(\frac{T_E}{2T})}$
\end{itemize}

\subsection{Modelo de Debye}
Aproximamos la ecuación de dispersión para $k$ baja como  $\omega = \nu k$
\[
  3N = \int_0^{\omega _D} 3g(\omega )d\omega = \frac{V}{2\pi^2 \nu^3}\omega _D^3 \Rightarrow \boxed{\omega _D = \sqrt[3]{\frac{6\pi^2\nu^3N}{V}} }
\] 
donde hemos contado cada partícula y cada estado 3 veces y hemos usado
\[
\omega = \nu k, \qquad g(k)=\frac{V}{2\pi^2}k^2, \qquad g(\omega )= \frac{V}{2\pi^2 \nu^3}\omega ^2
\]
La energía y la capacidad calorífica
\begin{align*}
  \langle E \rangle &= \int_0^{\omega _D}\hbar \omega 3g(\omega )\left( \frac{1}{e^{\beta \hbar \omega }-1} +\frac{1}{2} \right) d\omega  =\\
					&=E_0 + \frac{3V\hbar }{2\pi^2\nu^3} \int_0^{\omega _D}\frac{\hbar \omega ^3}{e^{\beta \hbar \omega }-1} d\omega \qquad (x=\frac{\hbar \omega }{k_B T})\\
  T_D :&= \frac{\hbar \omega }{k_B} \quad   \Rightarrow  \boxed{\langle E \rangle = \frac{3Vk_B^4T^4}{2\pi^2\nu^3\hbar ^3}\int_0^{\frac{T_D}{T}}\frac{x^3}{e^x-1}dx }
\end{align*}
La capacidad calorífica $C_v = \frac{\partial \langle E \rangle }{\partial T}$ en los extremos:
\begin{itemize}[topsep=-6pt, itemsep=0pt]
  \item Si $T\gg T_D \quad \Rightarrow \quad \langle E \rangle \sim 3Nk_BT \quad \Rightarrow \quad C_v \sim  3Nk_B $
  \item Si $T\ll T_D \quad \Rightarrow \quad \langle E \rangle \sim  \frac{3\pi^4Nk_BT^4}{5T_D^3}\quad\Rightarrow \quad C_v \sim  \frac{12\pi^4}{5}Nk_B \left( \frac{T}{T_D} \right)^3 $
\end{itemize}

\section{No se, cuanticocosas}
\subsection{Drude model}
\begin{align*}
  &n=\frac{N}{V}; \quad\frac{dp}{dt} = F-\frac{p}{\tau}, \overline{j} = -ne \overline{v} = \sigma \overline{E}\\
  & mv=p=-e\tau E; \quad R_H = \frac{-1}{ne} = \frac{\rho_{yx}}{|B|}\\
  &\overline{E} = \tilde{\rho }\overline{j}; \quad \rho _{x x} = \rho _{yy}=\rho _{zz} \frac{m}{ne^2\tau }
\end{align*}

Hall resistivity $\rho _{xy}=-\rho_{yx} = \frac{B}{ne}$ $(\overline{B}\propto \hat{z} )$

Peltier coefficient $\Pi = -\frac{k_BT}{2e}= \frac{-c_vT}{3e}$

Seebeck coefficient $S = \frac{\Pi}{T}$

\begin{align*}
  <v>_{gas id.} = \sqrt{\frac{8k_BT}{\pi m}} ; \quad \kappa = \frac{1}{3} nc<v>^2\tau = \frac{4}{\pi }\frac{n\tau k_B^2T}{m}
\end{align*}

\subsection{Capacidad calorífica}
$g$ densidad de estados / $V$
\begin{align*}
  &g(\varepsilon ) = \frac{3n}{2(E_F)^{\frac{3}{2}}}\varepsilon ^{\frac{1}{2}} = \frac{(2m)^{\frac{3}{2}}}{2\pi ^2\hbar ^3}\varepsilon ^{\frac{1}{2}} , \quad k = \sqrt{\frac{2\varepsilon m}{\hbar ^2}}  \\
  & N = \int_0^\infty \hspace{-0.5em} d\varepsilon g(\varepsilon )n_F(\beta (\varepsilon -\mu )), \quad
  E_T = \int_0^\infty \hspace{-0.5em} d\varepsilon \varepsilon g(\varepsilon )n_F(\beta (\varepsilon -\mu ))\\
  & C = \frac{\pi ^2}{3}\left( \frac{3Nk_B}{2} \right) \left( \frac{T}{T_F} \right) 
\end{align*}
\[
\overline{M} = g(E_F)\mu _B^2 \overline{B}; \quad \mu _B = 0.67 \left( \frac{K}{Tesla} \right) k_B
\] 
\incfig{freemodel}
\incfig{quasifreemodel}
\incfig{brillouinejemplo}

\textbf{Teorema de Bloch} ($V(\overline{r})$ periódico)
\[
  \psi _{\overline{k}} (\overline{r}) = u_{\overline{k}}(\overline{r})e^{i \overline{k}\cdot \overline{r}}, \quad E(\overline{k}) = E(\overline{k}+\overline{G})
\] 
(1D) Fourier del potencial de dos formas
\[
V(x)=V_0 + \sum _{j=1}^\infty V_j\cos\left( \frac{2\pi j}{a}x \right)  \quad \text{ ó } \quad 
V(x)=\sum_{j=-\infty}^\infty V_{\frac{2\pi j}{a}} e ^{\frac{2\pi j}{a}ix}
\]
Con las relaciones $V_j = 2 V_{\frac{2\pi j}{a}}$, y donde los coeficientes son
\[
  V_j = \frac{2}{a} \int _{-\frac{a}{2}} ^{\frac{a}{2}} dx V(x)\cos\left( \frac{2\pi }{a}jx \right); \quad
  V_{\frac{2\pi j}{a}} = \frac{1}{a} \int _{-\frac{a}{2}} ^{\frac{a}{2}} dx V(x)e ^{\frac{-2\pi }{a}ijx}
\]

\underline{Gas de electrones libres}
\begin{align*}
\overline{k}=\frac{2\pi}{L}(n_x, n_y, n_z), \quad E(\overline{k})=\frac{\hbar ^2}{2m}|\overline{k}|^2, \quad n_F(x)=\frac{1}{e^x+1}\\
N = 2\sum_{\overline{k}} n_F(\beta(E(\overline{k})- \mu)) = 2 \frac{V}{(2\pi)^3}\int d \overline{k}n_F(\beta (E( \overline{k})-\mu))
\end{align*}

Fermi energy $(E_F = \mu (T\to 0))$
\begin{align*}
&E_F = \frac{\hbar ^2k_F^2}{2m}=k_BT_F, \quad p_F = \hbar k_F \\
&N = 2 \frac{V}{(2\pi)^3}\int _{|k|<k_F} dk \Rightarrow k_F = (3\pi ^2n)^{\frac{1}{3}}, \quad E_F = \frac{\hbar ^2(3\pi ^2n)^{\frac{2}{3}}}{2m}
\end{align*}

\[
  \overline{k} = \frac{2\pi }{L} \begin{pmatrix} n_x\\n_y\\n_z \end{pmatrix}, \quad n_x, n_y, n_z\in \mathbb{Z} \Rightarrow k^2 = \left( \frac{2\pi }{L} \right)^2 (n_x^2+n_y^2+n_z^2)  
\] 
Considerando los dos espines (multiplicamos por 2)
\[
N_T = 2 \cdot \left( \frac{4}{3}\pi (n_x^2+n_y^2+n_z^2)^{3 / 2} \right) \quad \Rightarrow \quad k_{max}^2 = k_F^2 = (3n\pi ^2) ^{ \frac{2}{3}}
\]


\underline{Electrones casi-libres}
\begin{align*}
  &\psi _+ \sim \cos (\pi \frac{x}{a}), \quad \psi _- \sim  \sin(\pi \frac{x}{a})\\
  &E^{\pm} = \frac{1}{2}(E^0_{\overline{k}-\overline{G}} + E^0_{\overline{k}})\pm \sqrt{\frac{1}{4}(E^0_{{\overline{k}}-\overline{G}}-E^0_{\overline{k}})^2 + |V_{\overline{G}}|^2} 
\end{align*}

\underline{Enlace fuerte, celda primitiva cúbica}
\begin{align*}
  &E(\overline{k}) \approx E_i-A-2B (\cos k_xa + \cos k_ya + \cos k_za)\\
  &A = -\langle \varphi _{i, n} | v | \varphi _{i,n} \rangle , \quad 
  B = -\langle \varphi _{i, m} | v | \varphi _{i,n} \rangle\\
  &\overline{v} = \nabla _{\overline{k}} \omega (\overline{k}) = \frac{1}{\hbar}\nabla _{\overline{k}}E(\overline{k})
\end{align*}

Carga de un campo $\overline{\mathcal{E}}$
\begin{align*}
  \dot{v}_i = \frac{1}{\hbar ^2}\sum_{j} \frac{\partial^2E}{\partial k_i\partial k_j} (-e\mathcal{E}_j), \quad \left( \frac{1}{m^*} \right)_{ij} = \frac{1}{\hbar ^2} \frac{\partial ^2E(\overline{k})}{\partial k_i\partial k_j}
\end{align*}

Caso totalmente degenerado
\begin{align*}
  m^* = \frac{\hbar ^2}{\left( \frac{d^2E}{dk^2} \right) }, \quad E(\overline{k}) = E_0 + \frac{\hbar^2}{2m^*}|k|^2, \quad \sigma \simeq \frac{e^2\tau (E_F)n}{m^*}
\end{align*}

\incfig{fermisufraces}

\underline{Tipos de materiales}

\textbf{Aislante:} Banda llena (2$e^-$). $V_g>4eV$\\
\textbf{Semiconductor} Banda llena ($2e^-$).  $V_g<4eV$.\\
\textbf{Metal} Banda semillena ($1e^2$ ó $2e^-$ con bandas solapantes).

\incfig{tiposmateriales}

\section{Semiconductores}
Densidad de estados
\begin{align*}
&D_C = \frac{(2m^*_n) ^{2 /3}}{2\pi ^2\hbar ^3}\sqrt{E-E_C} , \quad 
D_V = \frac{(2m^*_p) ^{2 /3}}{2\pi ^2\hbar ^3}\sqrt{E_V-E} \\
&n = 2\left(\frac{2\pi m^*_nk_BT}{h^2}\right)^{3 /2} e ^{\beta(E_F-E_C)} = N_{eff}^C e ^{\beta (E_F-E_C)}\\
&p = 2\left(\frac{2\pi m^*_pk_BT}{h^2}\right)^{3 /2} e ^{\beta(E_V-E_F)} = N_{eff}^V e ^{\beta (E_V-E_F)}\\
& np = N_{eff}^CN_{eff}^V e ^{-\beta E_g} = 4 \left(\frac{k_BT}{2\pi \hbar ^2}\right)^3(m_n^*m_p^*)^{3 / 2}e ^{-\beta E_g}\\
& e ^{2\beta E_F} = \frac{N_{eff}^V}{N_{eff}^C} e ^{\beta (E_V + E_C)}, \quad
E_F = \frac{E_C+E_V}{2} + \frac{3}{4}k_BT\ln\left( \frac{m_p^*}{m_n^*} \right) \\
& \mu = \frac{e\tau }{m^*}, \quad \sigma  = e(n\mu _n+ p\mu _p), \quad E_g = E_C-E_V
\end{align*}
Semiconductores dopados
\begin{align*}
  &E_n = \frac{m^*e^4}{2(4\pi \varepsilon \hbar )^2} \frac{1}{n^2}, \quad r = \varepsilon \frac{h^2}{\pi m^*e^2}\\
  &n \approx \frac{2N_D}{1+\sqrt{1+4 \frac{N_D}{N_{eff}^C} e ^{\beta E_d} }} 
\end{align*}
Unión p-n
\begin{align*}
  &n_n = N^C_{eff} e ^{\beta (E_F-E_C^n)}; \quad p_p = N_{eff}^Ve ^{\beta (E_V-E_F)}\\
  &d^0_n = \sqrt{ \frac{2\varepsilon V_D}{e} \frac{N_A / N_D}{N_A+N_D}}; \quad d_p^0 = \sqrt{\frac{2\varepsilon V_D}{e}\frac{N_D / N_A}{N_A+N_D}} \\
  & d_n(U) = d_n^0 \sqrt{1- \frac{U}{V_D}}; \quad d_p(U) = d_p^0 \sqrt{1- \frac{U}{V_D}}\\
  &eV_D=k_BT\ln\left( \frac{n_np_p}{n^2_i}\right); \quad I(U) = (I_n ^{gen} +I_p ^{gen})(e ^{\beta eU}-1)
\end{align*}




\section{Mates}
\begin{align*}
&\sin^2 \left(\frac{x}{2}\right) = \frac{1-\cos a}{2}\\
&\int_0^\infty \frac{1}{e^x-1}dx = +\infty, \quad \int_0^\infty \frac{1}{e^x+1}dx = \ln(2)\\
&\int_0^\infty \frac{x}{e^x-1}dx = \frac{\pi^2}{6}, \quad \int_0^\infty \frac{x}{e^x+1}dx = \frac{\pi^2}{12} \\
&\int_0^\infty \frac{x^2}{e^x-1}dx = 2\zeta(3), \quad \int_0^\infty \frac{x^2}{e^x+1}dx = \frac{3}{2}\zeta(3)\\
&\int_0^\infty \frac{x^3}{e^x-1}dx = \frac{\pi^4}{15}, \quad \int_0^\infty \frac{x^3}{e^x+1}dx = \frac{7\pi^4}{120} 
\end{align*}

\end{multicols}
\end{document}
