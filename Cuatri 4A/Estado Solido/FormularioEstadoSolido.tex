\documentclass[leqno]{article}
\usepackage{verbatim}
\usepackage{array} \usepackage{listings}
\usepackage{fancyvrb}
\usepackage{enumitem}
\usepackage{multicol}
\usepackage[utf8]{inputenc}
\usepackage[T1]{fontenc}
\usepackage{textcomp}
\usepackage{multicol}
\usepackage{mathtools}
\usepackage{amsmath}
\usepackage{wrapfig}
\usepackage{amssymb}
\usepackage{amsmath,amsfonts,amssymb,amsthm,epsfig,epstopdf,titling,url,array}
\usepackage{makecell}
\usepackage{hyperref}
\usepackage{eso-pic}
\usepackage{pgf}
\usepackage{tikz}
\usepackage{graphicx}
\usepackage{cancel}
\usepackage{svg}

% figure support
\usepackage{import}
\usepackage{xifthen}
\pdfminorversion=7
\usepackage{pdfpages}
\usepackage{transparent}
\usepackage{xcolor}

% geometry
\usepackage{geometry}
\geometry{a4paper, margin=0.5in}

% paragraph length
\setlength{\parindent}{0em}
\setlength{\parskip}{1em}

\newtheorem*{theorem}{Theorem}
\newtheorem*{lemma}{Lemma}
\newtheorem*{proposition}{Proposition}
\newtheorem*{definition}{Definition}
\newtheorem*{observation}{Observation}

\newcommand{\incfig}[1]{%
\center
\def\svgwidth{0.9\columnwidth}
\import{./figures/}{#1.pdf_tex}
}
\newcommand{\incimg}[1]{%
\center
\includegraphics[width=0.9\columnwidth]{images/#1}
}
\pdfsuppresswarningpagegroup=1

\title{Formulario Estadística}
\author{Abel Doñate Muñoz}
\date{}

\begin{document}

\begin{multicols}{2}[\columnsep2em]

\textbf{Constantes}
\begin{align*}
&k_B = 1.381\times 10^{-23}JK^{-1} = 8.26\times 10^{-5}eVK^{-1}\\
&m_e = 9.11\times 10^{-31}kg = 0.511MeVc^{-2}\\
&\varepsilon _0 = \frac{1}{4\pi K} = 8.85\times 10^{-12} Fm^{-1}\\
&\hbar = 1.055\times 10^{-34} Js = 6.58\times 10^{-16}eVs \\
&e = 1.602 \times 10^{-19} C
\end{align*}



\section{Estructura cristalina}
\subsection{Cosas}
Base dual y matriz métrica
\begin{align*}
  & a^* = \frac{b\times c}{V}, \quad b^* = \frac{c\times a}{V}  , \quad c^* = \frac{a\times b}{V}, \quad V = \det(\overline{a}, \overline{b}, \overline{c})  \\
  & (\overline{a}^*, \overline{b}^*, \overline{c}^*)=\begin{pmatrix} \overline{a}^T\\\overline{b}^T\\\overline{c}^T \end{pmatrix} ^{-1}, G = \begin{pmatrix} a\cdot a & a\cdot b & a\cdot c \\ b\cdot a & b\cdot b & b\cdot c \\ c\cdot a & c\cdot b & c\cdot c \end{pmatrix}, G^* = G^{-1}
\end{align*}

Cambio de base
\begin{align*}
  (\overline{a}', \overline{b}', \overline{c}') = (\overline{a}, \overline{b}, \overline{c})P, \quad \begin{pmatrix} x' \\ y' \\z' \end{pmatrix} =P^{-1} \begin{pmatrix} x\\y\\z \end{pmatrix} \\
  (x, y, z) = (x^*, y^*, z^*)P, \quad \begin{pmatrix} a'^* \\ b'^* \\z'^* \end{pmatrix} =P^{-1} \begin{pmatrix} a^*\\b^*\\c^* \end{pmatrix} 
\end{align*}

Red recíproca y distancia interplanar $g_{hkl}= \frac{1}{d_{hkl}}$

Transferencia de momento $Q = \frac{4\pi \sin\theta }{\lambda}$

Condiciones de Laue $\overline{Q}=2\pi \overline{g}_{hkl}$

Ley de Bragg $g_{hkl} = \frac{2\sin \theta_{hkl}}{\lambda}$ 

Módulo de Young $\nu_s = \sqrt{\frac{\gamma}{\rho }} $

Factor de estructura
\begin{align*}
  F_{hkl} = \sum_p f_p e^{-i2\pi \overline{g}_{hkl}\cdot \overline{r}_p}, \quad I \propto |F_{hkl}|^2
\end{align*}

\subsection{Estructuras comunes}

FCC
\[
\begin{cases}
  \overline{a} = \frac{1}{2} (1\ 1\ 0) \\
  \overline{b} = \frac{1}{2} (0\ 1\ 1) \\
  \overline{c} = \frac{1}{2} (1\ 0\ 1)
\end{cases} \quad 
\begin{cases}
  \overline{a}^* = (1\ 1\ -1)\\
  \overline{b}^* = (-1\ 1\ 1 )\\
  \overline{c}^* = (1\ -1\ 1)
\end{cases} 
\]
BCC
\[
\begin{cases}
  \overline{a} = \frac{1}{2} (1\ 1\ -1) \\
  \overline{b} = \frac{1}{2} (-1\ 1\ 1) \\
  \overline{c} = \frac{1}{2} (1\ -1\ 1)
\end{cases} \quad 
\begin{cases}
  \overline{a}^* = (1\ 1\ 0)\\
  \overline{b}^* = (0\ 1\ 1 )\\
  \overline{c}^* = (1\ 0\ 1)
\end{cases} 
\]
\begin{minipage}{\columnwidth}
\incfig{fcc}
\end{minipage}

Hexagonal
\[
\begin{cases}
  \overline{a} = (1, 0)\\
  \overline{b} = (-\frac{1}{2}, \frac{\sqrt{3} }{2})
\end{cases} \quad
\begin{cases}
  \overline{a}^* = \frac{2 \sqrt{3} }{3}(\frac{\sqrt{3} }{2}, \frac{1}{2}) \\
  \overline{b}^* = \frac{2 \sqrt{3} }{3}( 0, 1)
\end{cases}
\]
\begin{minipage}{\columnwidth}
  \incfig{hexagonal}
\end{minipage}



\section{Dinámica de cristales}
\subsection{Densidad de estados}
\[
  \overline{k} = \begin{pmatrix} \frac{2\pi}{L}n & \frac{2\pi}{L}m & \frac{2\pi}{L}l \end{pmatrix} \ \forall n, m, l \in \mathbb{Z}
\] 
Número de estados hasta $k$
 \[
N(k) = \int_{(\frac{2\pi}{L})^2(n^2 + m^2+l^2)\le k^2} dV \frac{L^3}{6\pi^2}k^3 = \frac{V}{6\pi^2}k^3
\] 
1, 2 y 3 dimensiones respectivamente (y se cumple $\omega = \nu_s k$)
\begin{align*}
  \begin{cases}
    g(k) = \frac{L}{\pi} \\  g(\omega ) = \frac{L}{\pi \nu} 
  \end{cases}
  \begin{cases}
    g(k) = \frac{L^2}{2\pi}k \\  g(\omega ) = \frac{L^2}{2\pi \nu^2}\omega 
  \end{cases}
  \begin{cases}
    g(k) = \frac{V}{2\pi^2}k^2 \\  g(\omega ) = \frac{V}{2\pi^2 \nu_s^3}\omega ^2
  \end{cases}
\end{align*}

\subsection{Dispersión}
Oscilador con masa $m$ y constante $k_s$
\[
F_n =m\ddot{x}_n = k_s(x_{n+1}+x_{n-1}-2x_n)
\]
\begin{align*}
   &-m\omega ^2 Ae^{i(kna-\omega t)} = k_sAe^{i(kna-\omega t)}(e^{ika}+e^{-ika}-2) =\\
   &=-4k_s \sin^2\left( \frac{ka}{2} \right)  \Rightarrow \boxed{\omega = 2 \sqrt{\frac{k_s}{m}}\left|\sin \left( \frac{ka}{2} \right) \right| }
\end{align*}

Oscilador con masa $m$ y constantes alternadas  $k_1, k_2$
\[
  \begin{cases}
m\ddot{x}_n = k_1(y_{n-1}-x_n)+k_2(y_n-x_n)\\
m\ddot{y}_n = k_1(x_{n+1}-y_n)+k_2(x_n-y_n)
  \end{cases}
\] 
Ansatz
\[
x_n = A e^{i(kna-\omega t)} \qquad
y_n = B e^{i(kna-\omega t)}
\] 
Ecuaciones
\[
  \begin{cases}
-m\omega ^2 A = -A(k_1+k_2)+ B(k_1e^{ika}+k_2)\\
-m\omega ^2 B = -A(k_1e^{ika}+k_2)+ B(-k_1-k_2)
  \end{cases}
\] 
Forma matricial
\[
  m\omega ^2 \begin{pmatrix} A\\B \end{pmatrix} = \begin{pmatrix} (k_1+k_2) & -k_2 -k_1e^{ika} \\ -k_2-k_1e^{ika} & (k_1+k_2) \end{pmatrix}  \begin{pmatrix} A\\B \end{pmatrix} = K\begin{pmatrix} A\\B \end{pmatrix} 
\] 
 \[
0 = \det(K-m\omega ^2I) = |(k_1+k_2)-m\omega ^2|^2 - |k_2 +k_1e^{ika}|^2
\] 
\[
  \boxed{\omega _{\pm}(k) = \sqrt{ \frac{k_1+k_2}{m}\pm \frac{1}{m} \sqrt{(k_1+k_2)^2-4k_1k_2\sin^2(ka / 2)} } }
\] 
Si $m_1\neq m_2$ y $k_s$ es la misma, sea  $K_i = \frac{k}{m_i}$, entonces
\[
  \boxed{\omega _{\pm}(k) = \sqrt{ (K_1+K_2)\pm \sqrt{(K_1+K_2)^2-4K_1K_2\sin^2(ka / 2)} } }
\] 
\begin{minipage}{0.5\columnwidth}
\incfig{dispersion}
\end{minipage}
\begin{minipage}{0.5\columnwidth}
\incfig{dispersion2}
\end{minipage}

\subsection{Modelo de Einstein}
\begin{align*}
  &E_n = \hbar \omega (n+\frac{1}{2}) \quad \Rightarrow \quad Z_1 = \frac{1}{2 \sinh(\frac{\beta \hbar \omega }{2})} \\
  &\langle E_1 \rangle = - \frac{\partial }{\partial \beta } \ln Z_1 = \frac{\hbar\omega }{2} \coth \left( \frac{\beta \hbar \omega }{2} \right) 
\end{align*}
  Energía y capacidad calorífica
\begin{align*}
&\langle E\rangle = \frac{3}{2} N \hbar \omega \coth \left( \frac{\beta \hbar \omega  }{2} \right) \\
&C_v = \frac{\partial \langle E\rangle}{\partial T}  = 3Nk_B (\beta \hbar \omega )^2 \frac{e^{\beta \hbar \omega }}{(e^{\beta \hbar \omega }-1)^2}
\end{align*}

Definimos ahora $T_E = \frac{\hbar \omega_E}{k_B}$. En los límites

\begin{itemize}[topsep=-6pt, itemsep=0pt]
  \item Si  $T\gg T_E \quad \Rightarrow \quad C_v = 3Nk_b$
  \item Si  $T\ll T_E \quad \Rightarrow \quad C_v = 3Nk_b (\frac{T_E}{T})^2 \frac{1}{\sinh ^2(\frac{T_E}{2T})}$
\end{itemize}

\subsection{Modelo de Debye}
Aproximamos la ecuación de dispersión para $k$ baja como  $\omega = \nu k$
\[
  3N = \int_0^{\omega _D} 3g(\omega )d\omega = \frac{V}{2\pi^2 \nu^3}\omega _D^3 \Rightarrow \boxed{\omega _D = \sqrt[3]{\frac{6\pi^2\nu^3N}{V}} }
\] 
donde hemos contado cada partícula y cada estado 3 veces y hemos usado
\[
\omega = \nu k, \qquad g(k)=\frac{V}{2\pi^2}k^2, \qquad g(\omega )= \frac{V}{2\pi^2 \nu^3}\omega ^2
\]
La energía y la capacidad calorífica
\begin{align*}
  \langle E \rangle &= \int_0^{\omega _D}\hbar \omega 3g(\omega )\left( \frac{1}{e^{\beta \hbar \omega }-1} +\frac{1}{2} \right) d\omega  =\\
					&=E_0 + \frac{3V\hbar }{2\pi^2\nu^3} \int_0^{\omega _D}\frac{\hbar \omega ^3}{e^{\beta \hbar \omega }-1} d\omega \qquad (x=\frac{\hbar \omega }{k_B T})\\
  T_D :&= \frac{\hbar \omega }{k_B} \quad   \Rightarrow  \boxed{\langle E \rangle = \frac{3Vk_B^4T^4}{2\pi^2\nu^3\hbar ^3}\int_0^{\frac{T_D}{T}}\frac{x^3}{e^x-1}dx }
\end{align*}
La capacidad calorífica $C_v = \frac{\partial \langle E \rangle }{\partial T}$ en los extremos:
\begin{itemize}[topsep=-6pt, itemsep=0pt]
  \item Si $T\gg T_D \quad \Rightarrow \quad \langle E \rangle \sim 3Nk_BT \quad \Rightarrow \quad C_v \sim  3Nk_B $
  \item Si $T\ll T_D \quad \Rightarrow \quad \langle E \rangle \sim  \frac{3\pi^4Nk_BT^4}{5T_D^3}\quad\Rightarrow \quad C_v \sim  \frac{12\pi^4}{5}Nk_B \left( \frac{T}{T_D} \right)^3 $
\end{itemize}

\section{Mates}
\begin{align*}
&\sin^2 \left(\frac{x}{2}\right) = \frac{1-\cos a}{2}\\
&\int_0^\infty \frac{x}{e^x-1}dx = \frac{\pi^2}{6} \\
&\int_0^\infty \frac{x^2}{e^x-1}dx = 2\zeta(3)\approx 2.40411 \\
&\int_0^\infty \frac{x^3}{e^x-1}dx = \frac{\pi^4}{15} 
\end{align*}

\end{multicols}
\end{document}
