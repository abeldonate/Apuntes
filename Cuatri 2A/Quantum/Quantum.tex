\documentclass{myclass}

\usepackage{verbatim}
\usepackage{array}

\usepackage[utf8]{inputenc}
\usepackage{mathtools}
\usepackage{amsmath}
\usepackage{amssymb}
\usepackage{amsmath,amsfonts,amssymb,amsthm,epsfig,epstopdf,titling,url,array}
\usepackage{hyperref}
\usepackage{tikz}
\usepackage{graphicx}
\usepackage{multicol}

\usepackage{import}
\usepackage{xifthen}
\usepackage{pdfpages}
\usepackage{transparent}


\newcommand{\norm}[1]{\lvert \lvert #1 \rvert \rvert }

\newcommand{\h}{\hbar}

\renewcommand{\th}{\textbf{Theorem} }

\newcommand{\defi}[1]{\textbf{Definition:} \textit{#1}}

\newcommand{\N}{\nabla}

\title{Quantum Physics}

\begin{document}
\maketitle
\tableofcontents
\newpage

\section{Thermal radiation}
\subsection{Black Bodies}
\defi{Thermal Radiation} Emission of radiation of a body due to its temperature. \\
\\
\defi{Blackbody} A hypothetical perfect absorber and radiator of energy, with no reflecting power.\\
\\
\defi{Spectral Radiance $R_\nu(T)$} Distribution of thermal radiation from blackbody. Power emitted per unit surface from an object at temperature $T$ int the frequency interval $[v, \nu + d\nu]$. Integrating for all $\nu$ gives the \textit{radiance}. \\
The \textit{Spectral Energy Density} is the energy per volume at $T$ in the interval $[\nu, \nu + d\nu]$. It is proportional to radiance $R(T)=\frac{c}{4}\rho(T)$. \\
\\
Experimental results came with the \textit{Stefan's Law}, that establish $R(T)=\sigma T^4$, where $\sigma$ is the \textit{Stefan-Boltzmann constant}. \\ As $T$ increases, the frequency $\nu_{max}$ at which $R(T)$ is maximum also increases due to \textit{Wien's Law} $\lambda_{max}T=0.2898 cmK$.

\subsection{Planck's approach to radiation}
Experimentally, Rayleigh have observed that for low frequencies $<E>\to kT$, but for high $<E>\to 0$. He proposed given the Boltzmann distribution $P_n=\dfrac{e^{-E_n/kT}}{\sum e^{-E_n/kT}}$:
$$
E_n=nh\nu \implies <E>=\frac{\sum E_nP_n}{\sum P_n}=h\nu\sum nP_n = \frac{h\nu}{e^{h\nu/kT}-1}
$$
Integrating for each value we have:
$$\rho(T)=\frac{8\pi^5k^4}{15c^3h^3}T^4 \implies R(T)=\sigma T^4 =\rho \frac{c}{4} $$

\subsection{Photoelectric Effect}
Einstein showed that given the electrons in a metal follow the following equation
$$
eV_0=K_{max} = h\nu -W_0
$$
Where $W_0$ is the work function.

\subsection{Relativity}
We have to take it into account when velocities are close to c:
\begin{itemize}
    \item $m=\gamma m_0 $
    \item $E^2=(pc)^2 + (m_0c^2)^2$
    \item $p = \frac{h}{\lambda}$ (\textit{De Broigle relation})
\end{itemize}

\subsection{Compton Effect}

\subsection{X-Rays}
\defi{X-Rays} Waves with $\lambda = [0.01, 10] nm$ \\
The energy of an X-ray is:
$$
E = h\nu = K_0-K_f \implies \lambda = \frac{hc}{K_0-K_f} 
$$

\section{Quantization and Atomic Models}
\subsection{Wave-Particle Duality}

\subsection{Uncertainty Principle}

\subsection{Atomic Models}
\subsubsection{Thomson}
Model with negative electrons distributed in a continuous positive charge. \\
Atoms could be in two states:
\begin{itemize}
    \item \textit{Ground State} Electrons remains at equilibrium
    \item \textit{Exited State} Electrons vibrate around equilibrium emitting electromagnetic radiation
\end{itemize}
\subsubsection{Rutherford}
Positive charge is concentrated in the nucleus, but some problems like fall of electrons arise with the model.

\subsubsection{Bohr}
Bohr's posutlates:
\begin{itemize}
    \item The electron moves in a circular orbit around the nucleus due to Coulomb's Law
    \item The angular momentum of the orbit of an electron only can be $L=n\hbar$
    \item Electron does not emmit radiation
    \item Electron emmit radiation when hopping from orbit with frequency $\nu = \frac{E_i-E_f}{h}$
\end{itemize}
$$
mvr = pr = n\hbar, \qquad p = \frac{h}{\lambda} \implies \boxed{2\pi r=n\lambda}
$$
Making equal the forces in the orbit
$$
\frac{1}{4\pi\varepsilon_0}\frac{Ze^2}{r^2}=m\frac{v^2}{r} \implies \boxed{r_n = \frac{4\pi \varepsilon_0h^2}{mZe^2} \qquad n=1, 2, 3, \ldots}
$$
Velocity and Energy of orbits also quantized
$$
E_n = -\frac{1}{2} \frac{Ze^2}{4\pi \varepsilon_0r_n}
\frac{m(Ze^2)^2}{2(4\pi \varepsilon_0\hbar)^2}\frac{1}{n^2} 
\implies \boxed{\nu_{i\to f} = \frac{m(Ze^2)^2}{\varepsilon_0^2(4\pi \hbar)^3}\left(\frac{1}{n_f^2} - \frac{1}{n_i^2}\right)}
$$


\section{Schrödinger's Theory}

\section{One-electron Atoms}
\subsection{Previous considerations}
\textbf{Reduced mass}\\
In order to take into account both forces ($F_{a\to b}$ and $F_{b\to a}$) we can work with:
$$
\mu = \frac{Mm}{M+m} \implies F_e = \mu a_{rel}
$$
\textbf{Rydberg's constant}
$$
R_\infty = \left( \frac{1}{4\pi \epsilon_0}\right)^2\frac{me^4}{4\pi \hbar^3 c}, \qquad R_M = \frac{\mu}{m}R_\infty 
$$

\subsection{TDSE in 3D}
Developing the Schrodinger's equation with Coulomb potential $V = - \frac{Ze^2}{4\pi \epsilon_0r}$:
$$
-\frac{\hbar^2}{2m}\N^2\Psi - \frac{Ze^2}{4\pi \epsilon_0r}\Psi = E\Psi
$$
Suppose that $\Psi = R(r)\Theta(\theta)\Phi(\phi)$. Then we have the following ODEs:\\
\\
\textbf{Azimuthal}\\
$$
\frac{d^2\Phi}{d\phi^2} = -m_l^2\Phi \implies \boxed{\Phi= e^{im_l\phi}}, \qquad m_l = \ldots, -2, -1, 0, 1, 2, \ldots
$$
\textbf{Polar}\\
$$
-\frac{1}{\sin\theta} \frac{d}{d\theta}\left(\sin\theta\frac{d\Theta}{d\theta}\right) + \frac{m_l^2\Theta}{\sin^2\theta} = l(l+1)\Theta
$$
Using the change of variable $z = cos\theta$:
$$
\frac{d}{dz}\left((1-z^2)\frac{d\theta}{dz}\right) + \left( l(l+1) - \frac{m_l^2}{1-z^2} \right)\Theta = 0
$$
The solution is given in terms of Legendre Polynomials $P_l$:
$$
\boxed{\Theta(z) = (1-z^2)^{|m_l|/2}\frac{d^{|m_l|}P_l(z)}{dz^{|m_l|}}}
$$
\textbf{Radial}\\
$$
\frac{1}{r^2} \frac{d}{dr}\left(r^2\frac{dR}{dr}\right) + \frac{2\mu}{\hbar^2}(E-V(r))R = l(l+1)\frac{R}{r^2}
$$
Using the parameters $ \rho = 2\beta r, \quad \beta^2 = -\frac{2\mu E}{\hbar^2}, \quad
\gamma = \frac{\mu Z e^2}{4\pi \epsilon_0 \hbar \beta}$:
$$
\frac{1}{\rho^2}\frac{d}{d\rho}\left( \rho^2\frac{dR}{d\rho} \right) + \left( -\frac{1}{4} - \frac{l(l+1)}{\rho^2} + \frac{\gamma}{\rho} \right)R = 0
$$
If we consider $\rho$ large we can express the solution in terms of Laguerre Polynomials $F$:
$$
\boxed{R_{nl}(\rho) = F_{nl}(\rho)e^{-\rho/2}}
$$


\subsection{Quantum numbers}
Note that $\Psi$ depends of three parameters 
\begin{itemize}
    \item Principal $n = 1, 2, 3, \ldots$ (Total Energy)
    \item Azimuthal $l= 0, 1, 2, \ldots, n-1$
    \item Magnetic $m_l = 0, \pm 1,\pm 2 \ldots, \pm l$
\end{itemize}
Other quantities appear such as:

\begin{center}
\begin{tabular}{ccccc}
  Orbital angular momentum & $L = \sqrt{l(l+1)}\h$ & $\displaystyle \overline{u}_l = -\frac{g_I \mu_b}{\h}\overline{L}$ & $L_z = m_l\h$ & $u_{l_z} = -g_I\mu_b m_l$ \\
  Spin angular momentum & $S = \sqrt{s(s+1)}\h$ & $ \displaystyle\overline{u}_s = -\frac{g_s \mu_b}{\h}\overline{S}$ & $S_z = m_s\h$ & $u_{s_z} = -g_s\mu_bm_s$ \\
  Total angular momentum & $J=\sqrt{j(j+1)}\h $ & $\displaystyle\overline{J} = \overline{L} + \overline{S}$ & $J_z = m_j\h$ &  $m_j = m_l + m_s$ 
\end{tabular}
\end{center}

Con $g_I = 1, \quad g_s = 2$

\subsection{Probability densities}
\textbf{Radial Probability density} \\
$$
\boxed{P_{nl}(r)dr = R_{nl}^*(r)R_{nl}(r)4\pi r^2dr} 
$$
\textbf{Angular Probability Density} \\
$$
\boxed{A_{nlm_l}(r)dr \propto \Theta^*(\theta)\Theta(\theta)d\theta} 
$$
\textbf{Total Probability Density} \\
$$
\boxed{T = P(r)A(\theta)} 
$$
\subsection{Angular Momentum and Spin}
From the definition $\overline{L} = \overline{r}\times \overline{p}$ we get, as $\overline{p} = -i\h \N  $:
 \[
L_x=-i\hbar\left(y\frac{\partial}{\partial z} - z\frac{\partial}{\partial y} \right), \qquad
L_y=-i\hbar\left(z\frac{\partial}{\partial x} - x\frac{\partial}{\partial z} \right), \qquad
L_z=-i\hbar\left(x\frac{\partial}{\partial y} - y\frac{\partial}{\partial x} \right), \qquad
\] 
We have also that $\psi$ is eigenfunction of $L_z\psi = \h m_l \psi, \quad L^2 \psi = \h^2l\left( l+1 \right)\psi $. \\
\\
We write the magnetic dipole moment as:
\[
\overline{\mu}_l = -\frac{eL}{2m} =-\frac{g_I\mu_b}{\h}\overline{L} \qquad g_I = 1
\] 
\subsection{Magnetic Fields}
The experiment of Stern-Gerlach is as follows: (brief explanation)\\
\\
Knowing the magnetic field gradient $\displaystyle F_z = \pm \frac{\partial B_z}{\partial z}\mu_b$, the temperature of the oven $T$, and the length of the detector $L$, we can compute the $Z$ deviation by:
\[
\frac{1}{2}v_x^2 = 2k_bT \implies v_x = \sqrt{\frac{4k_bT}{M}} \implies \tau = \frac{L}{v_x} = L\sqrt{\frac{M}{4k_bT}} \implies Z = \frac{1}{2} a_z \tau^2 = \frac{\frac{\partial B_z}{\partial z}\mu_bL^2}{8k_bT}  
\] 
The spin is a quantum number that does not come from the resolution of Schrödinger equation, but is added to match the theoretical development with the experiments. \\
The value of the spin can only be $m_s = \pm \frac{1}{2}$

\subsection{Transition and selection rules}
A transition must obey $\Delta l = \pm 1, \quad \Delta j = 0, \pm 1$ \\
\\
We define the electric dipole as $\overline{p} = -e\overline{r}$. If we compute the expected value of $p_{ij} = \int \Psi_i^* er\Psi_jdV$. \\
Knowing $\Psi(r, \pi-\theta, \varphi  + \pi) = (-1)^l\Psi(r, \theta, \varphi )$ \\
\\
The transmission rate is $\displaystyle R = \frac{16\pi^3\nu^3p_{ij}^2}{3 \epsilon_0 h c^3}$

\section{Elementary Particles}
\subsection{Two particles system}
We will study a system with two particles. Then we will have a wave function $\Psi \left( \overline{r_1}, \overline{r_2}, t \right) $. This function must follow the Schrödinger's Equation. \\
\\
Now we consider the potential $V(\overline{r_1}, \overline{r_2})$ time-independent. As we have shown in chapter 3:
\[
\Psi\left( \overline{r_1}, \overline{r_2} \right)\phi(t) = \Psi(\overline{r_1}, \overline{r_2})e^{-\frac{iEt}{\h}}
\] 
But we cannot distinguish both particles if they are equal, but we can suppose $\Psi\left( \overline{r_1}, \overline{r_2} \right) = \Psi(\overline{r}_1)\Psi(\overline{r}_2)$ \\
\\
We define the symmetric $S$ and antisymmetric $A$ eigenfunctions as:
\[
\begin{cases}
\Psi_S = \frac{1}{\sqrt{2} }\left( \Psi_\alpha(\overline{r}_1)\Psi_\beta(\overline{r}_2) + \Psi_\alpha(\overline{r}_2)\Psi_\beta(\overline{r}_1) \right) \\
\Psi_S = \frac{1}{\sqrt{2} }\left( \Psi_\alpha(\overline{r}_1)\Psi_\beta(\overline{r}_2) - \Psi_\alpha(\overline{r}_2)\Psi_\beta(\overline{r}_1) \right) 
\end{cases}   
\] 
In the case that they are confined in an infinite potential well of length $a$ and $n_\alpha = 1, n_\beta = 2$ we have:
\[
\begin{cases}
  \Psi_\alpha = B\cos\left( \frac{\pi x}{a} \right) 
  \Psi_\beta = \sin\left( \frac{2\pi x}{a} \right) 
\end{cases}
\] 

\subsection{Exclusion Principle}
\textbf{Weak Exclusion Principle:} In a multi-electron atom, there can never be more than 1 electron in the same quantum state\\
\\
\textbf{Strong Exclusion Principle:} A system containing several electrons (Fermions) must be described by an antisymmetric wave function \\
\\
Thus, a multi-electron atom can be described antisymmetric.





\section{Multielectron Atoms}
\subsection{Combined Spin states}
If we have a system of 2 particles whose state of spin are $\alpha(), \beta()$ :
\begin{itemize}
  \item Symmetric: (Triplet) $\alpha(1)\alpha(2), \quad \beta(1)\beta(2), \quad \frac{1}{\sqrt{2} }\left( \alpha(1)\beta(2) + \alpha(2)\beta(1) \right)  $ \\
	$s=1, m_s = 1, 0, -1$
  \item Antisymmetric: (Singlet) $\frac{1}{\sqrt{2} }\left( \alpha(1)\beta(2)-\alpha(2)\beta(1) \right) $ \\
	$s=0, m_s = 0$
\end{itemize}



















\end{document}
