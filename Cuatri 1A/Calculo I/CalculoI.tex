\documentclass[12pt]{article}
\title{Cálculo}
\author{Abel Doñate Muñoz \\
\small{\href{mailto:abel.donate@estudiantat.upc.edu}{abel.donate@estudiantat.upc.edu}}}
\date{}

\usepackage{amsmath}
\usepackage{amssymb}
\usepackage{hyperref}
\usepackage{amsmath,amsfonts,amssymb,amsthm,epsfig,epstopdf,titling,url,array}

%Geometry
\usepackage{geometry}
\geometry{a4paper, margin=1in}

\newtheorem{theorem}{Theorem}[section]

\theoremstyle{plain}
\newtheorem{thm}{Theorem}[section]
\newtheorem{lem}[thm]{Lemma}
\newtheorem{prop}[thm]{Proposition}
\newtheorem*{cor}{Corollary}

\theoremstyle{definition}
\newtheorem{defn}{Definition}[section]
\newtheorem{conj}{Conjecture}[section]
\newtheorem{exmp}{Example}[section]

\theoremstyle{remark}
\newtheorem*{rem}{Remark}
\newtheorem*{note}{Note}

\begin{document}

\maketitle
\tableofcontents
\newpage

\section{Los axiomas de R}
\begin{itemize}
  \item 9 axiomas algebraicos $(\mathbb{R},+,\cdot)$ anillo conmutativo
  \item 4 axiomas de orden $(\mathbb{R},+,\cdot, \leq)$ + 2 más de consistencia
  \item R16 Axioma arquimediano: $\exists n \in \mathbb{N}\ |\ nx>y$
  \item R17 $\mathbb{Q} \subset \mathbb{R}$
  \item R18 Axioma de completitud (o del Supremo): \\
  \[A \subset \mathbb{R}, \ A \ \text{acotado superiormente} \ \implies \ \exists \alpha \in 			  \mathbb{R} \ | \ \alpha = sup(A)\]
\end{itemize}

\section{Sucesiones}
\begin{defn}
	$(a_n)$ está acotada $\implies \exists \ M \in \mathbb{R}\ \mid \  a_n \leq M,\  \forall n \in \mathbb{N}$
\end{defn}
\begin{defn}
	$(a_n)$ es monótona $\implies a_n\leq a_{n+1}\ \forall n\in \mathbb{N}$
\end{defn}
\begin{defn}
	$(a_n)$ tiene límite $L$ $\implies \forall \varepsilon >0 \ \exists \ n_0\in \mathbb{N} \ | \ n\geq n_0 \ \implies |a_n-L|<\varepsilon$
\end{defn}
\begin{defn}
	Definimos el límite superior de $(a_n)$ como $\lim\ sup\ a_n$, el máximo de los límites de sucesiones parciales de $(a_n)$. Se tiene que lim sup $a_n=$ inf \{$sup_{m\geq n}{\{a_n\}}$\}
\end{defn}


\subsection{Criterios de convergencia}
\begin{itemize}
	\item \textbf{Teorema de la convergencia monótona}\\
	\[a_n \text{ está acotada y es monótona } \implies \ a_n \text{ tiene límite.}\]
	\item \textbf{Teorema de Bolzano-Weierstrass}\\
	\[a_n \text{ está acotada } \implies \ a_n \text{ tiene una sucesión parcial convergente.}\]
	\item \textbf{Criterio de Stolz}\\
	\[y_n \ \text{divergente, creciente y positiva\\
	   } \implies \ \ \ \lim \frac{x_n}{y_n}=\lim \frac{x_{n+1}-x_n}{y_{n+1}-y_n} \]   
	\item \textbf{Teorema del Sandwich}\\ Buscar una cota superior e inferior con el mismo límite			\item $L=a_n^{b_n}=1^\infty \ \ \implies \ \ L=e^{(b_n-1)a_n}$
	\item $a_n\to L \ \ \implies \ \ \sqrt[n]{a_1\cdots a_n}\to L$
	\item $\frac{a_{n+1}}{a_n}\to L \ \ \implies \ \ \sqrt[n]{a_n}\to L$
	\item $si \ \lvert x_{n+1}-x_n \rvert \leq \rho \lvert x_{n}-x_{n-1} \rvert \ \ (\rho<1) \ \ \forall n>n_0\ \ \implies x \ \text{es Cauchy y tiene límite}$
\end{itemize}

\subsection{Sucesiones de Cauchy}
\begin{defn}
	.\\
	$(a_n)$ es una sucesión de Cauchy $\implies \forall \varepsilon \geq 0 \ \ \exists n_0 \ \ | \ \ |a_n-a_m|\leq \varepsilon \ \ \forall n,m\geq n_0$\\
	Si $a_n$ es Cauchy, entonces tiene límite.
\end{defn}


\section{Funciones}
	\subsection{Límite de una función en un punto}
	\[\boxed{
	\lim_{x\to a}f(x)=L \ \ \iff \ \ \forall \varepsilon >0 \ \exists \ \delta \ | \ 0<\lvert x-a \rvert <\delta \ \implies \lvert f(x)-L\rvert <\varepsilon
	}\]
	Observa que $L$ no tiene porque ser igual que $f(a)$ necesariamente
	\subsection{Inyectividad, sobreyectividad y biyectividad}
	\begin{itemize}
		\item $f \text{ inyectiva} \ \ \implies \ \ \forall x,y\in A, \ x\neq y, \ f(x)\neq f(y)$
		\item $f \text{ sobreyectiva} \ \ \implies \ \ f(A)=B$
		\item $f \text{ biyectiva} \ \ \implies \ \ f$ es inyectiva y sobreyectiva
	\end{itemize}
	\subsection{Tipos de discontinuidades}
	\begin{itemize}
		\item[1)] \textbf{Evitable} $\lim_{x\to a} f(x) =b, f(a)\neq b$
		\item[2)] \textbf{Salto} $\lim_{x\to a}{f(x)} \nexists \ \ \implies \ \ \lim_{x \to a^-}{f(x)}=b \neq \lim_{x\to a^+}{f(x)}=c$
		\item[3)] \textbf{Esencial} un limite lateral se va a infinito
	\end{itemize}
	\subsection{Teoremas sobre continuidad}
	
	La función $f$ es \textbf{Uniformemente continua}:
	\[\boxed{\forall \varepsilon >0 \ \exists \ \delta >0 \ | \ si \ |x-y|<\delta \implies \ |f(x)-f(y)|<\varepsilon}\] \\
	\textbf{Teorema de Bolzano.} \\
	Sea $f:I\to \mathbb{R}$ continua, $f(a)f(b)<0$:
	\[\exists c\in (a,b) \ | \ f(c) =0\]
	Similar a \textbf{Teorema del valor intermedio}. \\ \\
	\textbf{Teorema de Weirestrass.} \\
	Sea $f$ continua en $[a,b]$ tiene máximo y mínimo. \\ \\
	\textbf{Teorema de Heine-Cantor.} \\
	Sea $f$ continua y compacta (fitada y cerrada) $\implies$ uniformemente continua.
\section{Derivadas}
La derivada existe en el punto $a$ si existe el límite
\[\boxed{f'(a)=\lim_{h \to =0}{\frac{f(a+h)-f(a)}{h}}}\]
	\subsection{Teoremas sobre derivación}
	\textbf{Teorema de Rolle}\\
	Sea $f$ continua y derivable en $[a,b]$ y $f(a)=f(b)$
	\[\implies \ \ \ \exists c \in (a,b) \ | \ f'(c)=0\] \\
	\textbf{Teorema de Cauchy} \\
	Sean $f, g: [a,b] \to \mathbb{R}$ cumpliendo
	\begin{itemize}
		\item[1)] $f,g$ continuas
		\item[2)] $f,g$ derivables en el interior
		\item[3)] $g(a)\neq g(b)$
		\item[4)] Si $g'(x)=0\ \implies \ f'(x)\neq 0$
	\end{itemize}
	\[\implies \ \ \ \exists c \in (a,b) \ | \ \frac{f(b)-f(a)}{g(b)-g(a)}=\frac{f'(c)}{g'(c)}\] \\	
	\textbf{Teorema del valor medio}\\
	Sea $f$ continua en $[a,b]$, derivable en $(a,b)$
	\[\implies \ \ \ \exists c \in (a,b) \ | \ \frac{f(b)-f(a)}{b-a}=f'(c)\]\\
	\\
	\textbf{Regla de L'Hopital}\\
	Sean $f, g$ derivables en $[a-r,a]$ tal que:
	\begin{itemize}
		\item[1)] $g(x)\neq 0, g'(x)\neq 0$ en todo $I$
		\item[2)] $\lim_{x\to a}{f(x)}=\lim_{x\to a}{g(x)}=0$
		\item[3)] $\lim_{x\to a}{\frac{f'(x)}{g'(x)}}=L\in \mathbb{R}\cup \infty$		
	\end{itemize}
	\[\implies \ \ \ \lim_{x\to a}{\frac{f(x)}{g(x)}}=L\in \mathbb{R}\cup \infty\] \\
	\subsection{Convexidad}
	$f$ es convexa $\iff \ f((1-t)a+tb)\leq (1-t)f(a)+tf(b)$\\
	\\
	\[a<x<y<b \ \ \ \implies \frac{f(x)-f(a)}{x-a}\leq \frac{f(y)-f(x)}{y-x}\leq \frac{f(b)-f(y)}{b-y}\]
	Haciendo los límites $x\to a, y\to b$ y sandwitch tenemos $f'$ creciente $\implies f''\geq 0$
	
	\subsection{Serie de Taylor}
	\textbf{Teorema de Taylor}\\
	\[\lim_{x\to a}\frac{f(x)-P_{n, a, f}(x)}{(x-a)^n} = \frac{R_{n, a, f}(x)}{(x-a)^n} = 0\]
	\\
	\textbf{Residuo de Lagrange}\\
	\[R_{n,a,f}(x) = f(x) - P_{n,a,f}(x)\]
	\\
	\textbf{Teorema de Lagrange}\\
	\[\exists c \in (a,x) \ \ \ \ R_{n,a,f}(x) = \frac{f^{(n+1)}(c)}{(n+1)!}(x-a)^{(n+1)}\]
	\\
	
	
	
\section{Integrales}
Una función es \textbf{Riemman integrable} si:
\[\int_a^b{f}= \underline{\int_a^b}f=\overline{\int_a^b}f\]
	
	\subsection{Teoremas sobre integrabilidad}
	\textbf{Th:}
	$f$ es integrable  $\iff \ \ \ \forall \varepsilon > 0 \ \exists P \ | \ U(P,f)-L(P,f)<\varepsilon$ \\
	\\
	\textbf{Th:}
	$f$ es integrable  $\iff \ \ \ \exists (P_n)_n \ | \ \lim_{n\to \infty} L(P,f)=\lim_{n\to \infty} L(P,f)=\int_a^bf$ \\
	\\
	\textbf{Th:}
	$f:[a,b]\to \mathbb{R}$ monotona ($\implies$ fitada) $\implies$ integrable\\
	Hint: partición $x_i=a+\frac{b-a}{n}i$ y usar monotonicidad.\\
	\\
	\textbf{Th:}
	$f:[a,b]\to \mathbb{R}$ continua ($\implies$ fitada por Weierstraß) $\implies$ integrable\\
	Hint: Uniformemente continua y aplicar el $\varepsilon$.\\
	\\
	\textbf{Th:}
	$f$ integrable, $g$ continua $\Rightarrow \ g\circ f$ integrable\\
	\\
	\subsection{Teorema fundamental del cálculo}
	Sea $f:[a,b]\to \mathbb{R}$ integrable. $F:[a,b]\to \mathbb{R}$ es primitiva de $f$ si
	\begin{itemize}
		\item $F$ es continua
		\item $F$ es derivable en $(a, b)$
		\item $F'=f(x) \ \forall x \in (a,b)$
	\end{itemize}
	Si $f$ tiene primitiva $\Rightarrow$ $f$ es integrable. (No es doble implicación, e.g. step function).\\
	\\
	\textbf{Teorema fundamental del cálculo}
	Sea $f:[a,b]\to \mathbb{R}$ integrable. Definimos $F(x)=\int_a^xf$:
	\begin{itemize}
		\item F es contínua
		\item Si $f$ es continua en $c\in (a,b) \ \Rightarrow \ F$ es derivable en $c$ y $F'(c)=f(c)$
	\end{itemize}
	\textbf{Regla de Barrow}
	Sea $f:[a,b]\to \mathbb{R}$ integrable:
	\[\int_a^bf=F(b)-F(a)\]
	\textbf{Teorema del valor medio para integrales}
	Sea $f:[a,b]\to \mathbb{R}$ continua.
	\[\exists c \in (a,b) : \int_a^bf=f(c)(b-a) \]
	




\end{document}


























