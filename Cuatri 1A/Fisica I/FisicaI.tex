\documentclass[12pt]{article}

\title{Física}
\author{Abel Doñate}
\date{}

\usepackage{amsmath}
\usepackage{pgfplots}

%Geometry
\usepackage{geometry}
\geometry{a4paper, margin=1in}

\begin{document}

\maketitle
\tableofcontents
\newpage

\section{Cinemática}
\subsection{MRU}
\[\bar{a}=\bar{0},  \ \ \ \bar{x}=\bar{x_0}+\bar{v}t\]

\subsection{MRUA}
\[\bar{a} \ constante,  \ \ \ \bar{x}=\bar{x_0}+\bar{v_0}t+\frac{1}{2}\bar{a}t^2\]

\subsection{a no constante, modulo de a constante}
\[\bar{a}=\frac{d}{dt}(\frac{ds}{dt}\hat{\tau})=\frac{d^2s}{dt^2}\hat{\tau}+\frac{ds}{dt}\frac{d\hat{\tau}}{dt}=\frac{d^2s}{dt^2}\hat{\tau}+\frac{v^2}{\rho}\hat{n}=a_t\hat{\tau}+a_n\hat{n}\]



\begin{align*}
\hat{\tau} &=  \frac{\bar{v}}{v} \\
\bar{a_t} &=  (\bar{a}\cdot \hat{\tau})\hat{\tau} \\
\bar{a_n} &= \bar{a} - \bar{a_t} \\
\hat{n}  &=\frac{\bar{a_n}}{a_n}
\end{align*}

\subsection{MCUA}
\[ \text{Tenemos variables como } \phi=\frac{x}{R}, \ \ \omega=\frac{v}{R}, \  \alpha=\frac{a}{R} \text{ que cumplen las mismas relaciones que } x,v,a.\]
\\
\\
En un MCUA $a_t = \alpha$ y $a_n=\frac{v^2}{R}=\omega^2R$

\section{Fuerzas y ecuaciones del movimiento de una partícula}
	\subsection{Leyes de Newton}
\begin{itemize}
	\item Toda partícula aislada tiene $v$ constante.
	\item La $a$ de un curpo es proporcional a la $F$ que actúa sobre él.
	\item Toda $F$ de acción conlleva una de reacción con el mismo módulo y sentido contrario.
\end{itemize}

\subsection{Fuerzas de contacto}
\subsubsection{Reacción Normal}
Causada por la interacción molecular. Siempre perpendicular a la superficie.
\subsubsection{Tensiones en cuerdas}
Suponiendo una curda sin masa, entonces la tensión a ambos lados de la cuerda es la misma.
\subsubsection{Fricción en sólidos}



\begin{tikzpicture}
\begin{axis}[
    axis lines = left,
    xlabel = $F$,
    ylabel = {$F_r$},
    legend style={at={(0.2,0.2)},anchor=west}
]
%Below the red parabola is defined
\addplot [
    domain=0:4, 
    samples=100, 
    color=red,
]
{x};
\addlegendentry{$F_r=F$ if $F<F_{femax}$}
%Here the blue parabloa is defined
\addplot [
    domain=4:8,
    samples=100,
    color=blue,
    ]
    {3};
\addlegendentry{$F_{r}=F_{fd}$ if $F>F_{femax}$}

\end{axis}
\end{tikzpicture}

\begin{align*}
	F_{femax} &= \mu_eN \\
	F_{fd} &= \mu_dN \\
	\mu_d&<\mu_e
\end{align*}

\subsubsection{Fricción en fluidos}
\[\bar{F_D}(\bar{v})=-(k_1v+k_2v^2)\hat{v} \ \ \ \left\{
\begin{array}{c}
	k_1 \mbox{ término viscoso} \\
	k_2 \mbox{término de presión}
\end{array}\right. \]

\section{Teoremas de conservación y movimientos angulares}
	\subsection{Comparación movimientos lineales y angulares}
	\begin{table}[h!]
\begin{tabular}{ll}
\underline{Lineales} & \underline{Angulares} \\
$x$ & $\theta$  \\
$v$ & $\omega$ \\
$a$ & $\alpha$ \\
$m$ & $I=mr^2$ \\
$F$ & $\tau=r \times F$\\
$p$ & $L=r \times p$
\end{tabular}
\end{table}
	\subsection{Trabajo y Energía}
	Definimos el trabajo como 
	\[W=\int_{r_a}^{r_b}{\bar{F}\cdot d\bar{r}}\]
	De aquí podemos deducir las Energías cinética y potencial gravitatoria y elástica
	\[E_c=\frac{1}{2}mv^2, \ \ \ \ U=mgh, \ \ \ \ U=\frac{1}{2}kx^2\]
	Definiremos el impulso $I$ como:
	\[I=\int_{t_1}^{t_2}{Fdt}=\Delta p\]
	Tenemos también que en un campo conservativo:
	\[\bar{F}=-\nabla U\]
	Introducimos también la Potencia:
	\[P=\frac{dW}{dt}=Fv\]
	\subsection{Conservación de variables}	
	En un sistema cerrado se conservan:
	\begin{itemize}
		\item $E_m=E_c+U$ (Energía mecánica)
		\item $p$ (Momento lineal)
		\item $L$ (Momento angular)
	\end{itemize}
	Si el sistema no es conservativo podemos separar su energía mecánica:
	\[E_{m_f}=E_{m_0}+W_{nc} \ \ \implies \ \ W_{nc}=\Delta E_c + \Delta U\]
	
\section{Oscilaciones}
	\[m\ddot{x} +b\dot{x}+kx=F_0cos(\omega t)\]
	\[\omega_0=\sqrt{\frac{k}{m}}, \ \ \ \gamma=\frac{b}{2m},\ \ \ \omega_1 = \sqrt{\omega_0^2-\gamma_0^2}, \ \ \ D=m\sqrt{(\omega_0^2-\omega^2)^2+(2\gamma \omega)^2}\] 
	\subsection{Tipos de osciladores}
	\textbf{1) Oscilador débilmente amortiguado ($\omega_0>\gamma$)}
	\[x=Ae^{-\gamma t}cos(\omega_1t+\phi)\]\\
	\textbf{2) Oscilador críticamente amortiguado ($\omega_0=\gamma$)}
	\[x=(A+Bt)e^{-\gamma t}\] \\
	\textbf{3) Oscilador fuertemente amortiguado ($\omega_0<\gamma$)}
	\[x=Ae^{(-\gamma+\sqrt{\gamma^2-\omega_0^2})t} + Be^{(-\gamma-\sqrt{\gamma^2-\omega_0^2})t}\] \\	
	\textbf{4) Oscilador forzado ($F_{ext}=F_0cos(\omega t)$)}\\
	$x(t)$ se puede expresar como la suma de la homogénea mas la particular $x=x_h+x_p$
	\[x_p=A_pcos(\omega t +\theta-\beta) \ \ \ \text{donde} \ \ \ tan\beta=\frac{2 \gamma \omega}{\omega_0^2-\omega^2}, \ \ \ A=\frac{F_0}{D}, \ \ \ Z=\frac{D}{\omega}\]
	
	\subsection{Variables de calidad}
	Las valiables mas comunes para ver la calidad de un oscilador débilmente amortiguado oscilador son:
	\begin{itemize}
		\item $\tau_E=\frac{1}{2\gamma}$ el tiempo que pasa de tener $E$ a $E/e$
		\item $n=\frac{\tau_E}{T}$ las oscilaciones antes de $\tau_E$
		\item $Q=\frac{2\pi}{|\frac{\Delta E}{E}|_T}=\frac{\omega_1}{2\gamma}=2\pi n$
	\end{itemize}
	
\section{Gravitación}
	Con un planeta en órbita tenemos:
	\[F(\bar{r})=-\frac{GMm}{r^2}\hat{r}, \ \ \ \ U=-\frac{GMm}{r} \ \text{es un campo conservativo} \]
	En cuanto a las variables angulares:
	\[\bar{\tau}=r\times F=0 \ \ \ \ \implies \ \ \ \ \bar{L}=const.\]
	
	\subsection{Órbitas elípticas}
	\[L_0=mr^2\dot{\theta} \ \ \ \ \alpha=\frac{L_0^2}{GMm^2}\ \ \ \ \varepsilon = \sqrt{1+\frac{2\alpha E}{GMm}}\]
	Entonces la función de la órbita es:
	\[r(\theta) = \frac{\alpha}{1 + \varepsilon cos(\theta)}\]
	
	
	

\end{document}

