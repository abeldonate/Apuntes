\documentclass[12pt,landscape]{article}
\usepackage{multicol}
\usepackage{calc}
\usepackage{ifthen}
\usepackage[landscape]{geometry}

% To make this come out properly in landscape mode, do one of the following
% 1.
%  pdflatex latexsheet.tex
%
% 2.
%  latex latexsheet.tex
%  dvips -P pdf  -t landscape latexsheet.dvi
%  ps2pdf latexsheet.ps


% If you're reading this, be prepared for confusion.  Making this was
% a learning experience for me, and it shows.  Much of the placement
% was hacked in; if you make it better, let me know...


% 2008-04
% Changed page margin code to use the geometry package. Also added code for
% conditional page margins, depending on paper size. Thanks to Uwe Ziegenhagen
% for the suggestions.

% 2006-08
% Made changes based on suggestions from Gene Cooperman. <gene at ccs.neu.edu>


% To Do:
% \listoffigures \listoftables
% \setcounter{secnumdepth}{0}


% This sets page margins to .5 inch if using letter paper, and to 1cm
% if using A4 paper. (This probably isn't strictly necessary.)
% If using another size paper, use default 1cm margins.
\ifthenelse{\lengthtest { \paperwidth = 11in}}
	{ \geometry{top=.5in,left=.5in,right=.5in,bottom=.5in} }
	{\ifthenelse{ \lengthtest{ \paperwidth = 297mm}}
		{\geometry{top=1cm,left=1cm,right=1cm,bottom=1cm} }
		{\geometry{top=1cm,left=1cm,right=1cm,bottom=1cm} }
	}

% Turn off header and footer
\pagestyle{empty}
 

% Redefine section commands to use less space
\makeatletter
\renewcommand{\section}{\@startsection{section}{1}{0mm}%
                                {2ex plus -.5ex minus -.2ex}%
                                {0.5ex plus .2ex}%x
                                {\normalfont\small\bfseries}}
\renewcommand{\subsection}{\@startsection{subsection}{2}{0mm}%
                                {2ex plus -.5ex minus -.2ex}%
                                {0.5ex plus .2ex}%
                                {\normalfont\footnotesize\bfseries}}
\renewcommand{\subsubsection}{\@startsection{subsubsection}{3}{0mm}%
                                {-1ex plus -.5ex minus -.2ex}%
                                {1ex plus .2ex}%
                                {\normalfont\tiny\bfseries}}
\makeatother

% Define BibTeX command
\def\BibTeX{{\rm B\kern-.05em{\sc i\kern-.025em b}\kern-.08em
    T\kern-.1667em\lower.7ex\hbox{E}\kern-.125emX}}

% Don't print section numbers
\setcounter{secnumdepth}{0}


\setlength{\parindent}{0pt}
\setlength{\parskip}{0pt plus 0.5ex}

% -----------------------------------------------------------------------

\begin{document}

\raggedright
\footnotesize
\begin{tabular}{@{}p{\linewidth / 4}
                @{}p{\linewidth / 4}
                @{}p{\linewidth / 4}
                @{}p{\linewidth / 4}@{}}

\begin{center}
     \large{\textbf{Formulario de F\'{i}sica}}\\
     \tiny{Abel Doñate Muñoz}
\end{center}


\section{Cinemática}

\begin{tabular}{@{}p{\linewidth}@{}}

\textbullet $\Bar{a}$ constante \\
$\bar{x}=\bar{x_0}+\bar{v_0}t+\frac{1}{2}\bar{a}t^2$ 
\\
\\

\textbullet MCUA \\
$a_t=\frac{v^2}{R}=\omega^2R$ 
\\
\\

\textbullet $|a|$ constante \\
$\bar{a}= a_t\hat{\tau}+a_n\hat{n}$ 
\\
\\


\textbullet Comparación de variables \\
$x  = r\phi$ \\
$v  = r\omega$ \\
$a  = r\alpha$ \\
$m \ \Rightarrow I=mr^2$ \\
$p=mv \Rightarrow L=I\omega=r\times p$ \\
$F=ma \Rightarrow \tau=I\alpha=r\times F$ 
\\
\\

\end{tabular}
\vspace{-1.5em}

\subsection{Fricción}
\begin{tabular}{@{}p{\linewidth}@{}}

\textbullet Estática y dinámica \\
$F_{fmax}=\mu_eN$\\
$F_{fd}=\mu_dN$\\
$\mu_d<\mu_e$\\
\\

\textbullet En fluídos $v$ \\
$F_D=-bv \Rightarrow ma=mg-bv$ \\
$\gamma = \frac{b}{m}$ \\
$v_z=\frac{g}{\gamma} - \left( \frac{g}{\gamma} - v_{z0} \right)e^{-\gamma t}$ \\
$z = z_0+ \frac{g}{\gamma}t - \frac{1}{\gamma}\left( \frac{g}{\gamma} - v_0 \right)\left( 1- e^{-\gamma t} \right)$
\\
\\

\textbullet En fluídos $v^2$ \\
$F_D=-bv^2 \Rightarrow ma=-bv^2$ \\
$v = \frac{v_0}{1+v_0\gamma t}$ \\
$x=x_0 + \frac{ln(1+v_0\gamma t)}{\gamma}$
\\
\\

\end{tabular}
\vspace{-1.5em}

% SECOND COLUMN PAGE TABLE
&

\section{Energía}

\begin{tabular}{@{}p{\linwidth}@{}}

\textbullet Trabajo \\
$W=\int_{r_a}^{r_b}{\bar{F}\cdot d\bar{r}}$
\\
\\

\textbullet Energías \\
Cinética $K= \frac{1}{2}mv^2$ \\
Pot. gravitatoria $U=mgh$ \\
Pot. elástica $U = \frac{1}{2}kx^2$ 
\\
\\

\textbullet Impulso \\
$I=\int_{t_1}^{t_2}{Fdt}=\Delta p$
\\
\\

\textbullet Potencia\\
$P=\frac{dW}{dt}=Fv$
\\
\\

\textbullet En un campo conservativo \\
$\bar{F}=-\nabla U$
\\
\\

\textbullet Sistema no conservativo \\
$E_{m_f}=E_{m_0}+W_{nc}$ \\
$W_{nc}=\Delta E_c + \Delta U$ 
\\
\\



\end{tabular}
\vspace{-1.5em}

% THIRD COLUMN PAGE TABLE
&
\section{Oscilaciones}

\begin{tabular}{@{}p{\linewidth}@{}}

\textbullet Variables
\begin{itemize}
    \item $m\ddot{x} +b\dot{x}+kx=F_0cos(\omega t)$
    \item $\omega_0=\sqrt{\frac{k}{m}}$
    \item $\gamma=\frac{b}{2m}$
    \item $\omega_1 = \sqrt{\omega_0^2-\gamma^2}$
    \item $D=m\sqrt{(\omega_0^2-\omega^2)^2+(2\gamma \omega)^2}$
    \item $A^2 = x_0^2+ \frac{v_0^2}{\omega_0^2}$
    \item $tan(\phi)= -\frac{v_0^2}{x_0 \omega_0}$
\end{itemize}

\end{tabular}
\vspace{-1.5em}


\section{Tipos de Osciladores}

\begin{tabular}{@{}p{\linewidth}@{}}

\textbullet Débilmente amortiguado $(\omega_0 > \gamma)$ \\
$x=Ae^{-\gamma t}sin(\omega_1t+\phi)$\\
$A_0^2=x_0^2+(\frac{\gamma x_0+v_0}{\omega_1})^2$\\
$tan(\phi) = \frac{\omega_1 x_0}{\gamma x_0 + v_0}$ \\
Muy débil $|\Delta E/E|_T \approx T/\tau_E$
\\
\\

\textbullet Críticamente amortiguado $(\omega_0 = \gamma)$\\
$x=(A+Bt)e^{-\gamma t}$
\\
\\

\textbullet Fuertemente amortiguado $(\omega_0 < \gamma)$ \\
$x=A_1e^{-\gamma_1 t} + A_2e^{-\gamma_2 t}$\\
$\gamma_1 = \gamma - \sqrt{\gamma^2-\omega_0^2}$ \\
$\gamma_2 = \gamma + \sqrt{\gamma^2-\omega_0^2}$ \\
$A_1=\frac{v_0+\gamma_2 x_0}{\gamma_2 - \gamma_1}$ \\
$A_2=\frac{v_0+\gamma_1 x_0}{\gamma_1 - \gamma_2}$
\\
\\

\textbullet Forzado \\
$x=x_h+x_p$\\
$x_p=A_pcos(\omega t - \beta)$\\
$tan\beta=\frac{2 \gamma \omega}{\omega_0^2-\omega^2} = \frac{b\omega}{k-m\omega^2}$\\    
$A_p=\frac{F_0}{D} = \frac{F_0}{m\sqrt{(\omega_0^2-\omega^2)^2+(2\gamma \omega)^2}}$\\
(Impedancia mecánica) \\
$Z=\frac{D}{\omega} = \sqrt{b^2 + (mk-k/\omega)^2}$\\
$cos(\beta - \pi)= \frac{b}{Z}, \ sin(\beta - \pi) = \frac{m\omega - k/\omega}{Z}$

\end{tabular}
\vspace{-1.5em}






% FOURTH COLUMN PAGE TABLE
&



\section{Energía y potencia}

\begin{tabular}{@{}p{\linewidth}@{}}

\textbullet  Energía\\
Si $\gamma << \omega_0$:\\
$E =\frac{1}{2}kA(t)^2= \frac{1}{2}kA_0^2e^{-2\gamma t}$ \\
\\

\textbullet  Potencia\\
$P=-F\dot v=F_0 \omega A_p cos(\omega t)sin(\omega t -\beta)$ \\
$<P>=\frac{1}{T}\int F_{ext}vdt=\frac{bF_0^2}{2Z^2}$ \\
\\

\section{Resonancia}

\begin{tabular}{@{}p{\linewidth}@{}}

\textbullet  Frecuencia\\
$\omega_A=\omega_0 \to D_{min}=b\omega \to A_{max}$ \\ 
Velocidad máxima \\
$V_0=A\omega=\frac{F_0}{Z} = \frac{F_0}{b} \Rightarrow \omega = \omega_0$ \\
$<P> = \frac{F_0^2}{4m\gamma}$ \\
\\

\textbullet  Banda\\
$\omega=\omega_0 \pm \gamma$ \\
$<P>=\frac{F_0^2}{8m\gamma}$ \\

\end{tabular}


\section{Variables de calidad}

\begin{tabular}{@{}p{\linewidth}@{}}

\textbullet  Constante de tiempo (Energía)\\
$\tau_E=\frac{1}{2\gamma}$ \\
el tiempo que pasa de tener $E$ a $E/e$
\\
\\

\textbullet Teorema del trabajo-energ\'{i}a  \\
$n=\frac{\tau_E}{T}$ las oscilaciones antes de $\tau_E$
\\
\\

\textbullet Factor de calidad \\
$Q=\frac{2\pi}{|\frac{\Delta E}{E}|_T}=\frac{\omega_1}{2\gamma}=2\pi n$
\\
\\

\end{tabular}

\section{Péndulo}

\begin{tabular}{@{}p{\linewidth}@{}}

\textbullet  Periodo del péndulo\\
$T=2\pi\sqrt{\frac{l}{g}}$\\
$\omega=\sqrt{\frac{g}{l}}$\\
\\


\end{tabular}
\vspace{-1.5em}


\end{tabular}
\vspace{-1.5em}


\end{tabular}
\vspace{-1.5em}

\newpage

\begin{tabular}{@{}p{\linewidth / 4}
                @{}p{\linewidth / 4}
                @{}p{\linewidth / 4}
                @{}p{\linewidth / 4}@{}}
                

\section{Campo gravitatorio}

\begin{tabular}{@{}p{\linewidth}@{}}

\textbullet Ley de grav. universal \\
$F(\bar{r})= -\frac{GMm}{r^2}\hat{r}$ 
\\
\\

\textbullet Energía potencial grav. \\
$U(r)= -\frac{GMm}{r}$ 
\\
\\

\textbullet Momento angular y fuerza \\
$\bar{L}=\bar{r}\times \bar{p}$ constante \\
$\bar{\tau}=\bar{r}\times \bar{F}=\frac{d\bar{L}}{dt}=0$
\\
\\


\textbullet Leyes de Kepler 
\begin{itemize}
    \item[1)] Órbitas elípticas casi circulares
    \item[2)] Áreas = en tiempos =
    \item[3)] $T^2 \propto R^3$
\end{itemize}
\\
\\

\end{tabular}
\vspace{-1.5em}

\subsection{Coordenadas polares}
\begin{tabular}{@{}p{\linewidth}@{}}

\textbullet Vectores unitarios \\
$\bar{r}=x\hat{i}+ y\hat{j} = r\hat{r}$ \\
$\hat{r}=cos(\theta)\hat{i} + sin(\theta)\hat{j}$\\
$\hat{\theta}=-sin(\theta)\hat{i} + cos(\theta)\hat{j}$
\\
\\

\textbullet  Velocidad y $L$\\
$\bar{v}= \dot{r}\hat{r} + r\dot{\theta}\hat{\theta}=$ radial + angular \\
$L_0=mr^2\dot{\theta}\hat{k} \Rightarrow r^2\dot{\theta}$ const 
\\
\\

\textbullet Energía \\
$E=K+U=\frac{1}{2}m\dot{r}^2 + V_{eff}$ \\
$v_{eff} = \frac{L_0^2}{2mr^2} - \frac{GMm}{r}$\\
Dependiendo de $V_{eff}$ la órbita es:
\begin{itemize}
    \item $E>0 \Rightarrow $ Hiperbólica
    \item $E=0 \Rightarrow$ Parabólica
    \item $V_{eff min}<E<0 \Rightarrow$ Elíptica
    \item $E=V_{eff min} \Rightarrow$ Circular
\end{itemize}
\\
\\

\end{tabular}
\vspace{-1.5em}

% SECOND COLUMN PAGE TABLE
&

\section{Órbitas}

\begin{tabular}{@{}p{\linewidth}@{}}

\textbullet Ecuación diferencial \\
$\dot{\theta}= \frac{L_0}{mr^2}$\\
$\dot{r} = \sqrt{\frac{2}{m}(E-V_{eff}(r))}$ \\
$r(\theta) = \frac{\alpha}{1+ \varepsilon cos(\theta)}$
\\
\\

\textbullet Variables \\
$\alpha = \frac{L_0^2}{GMm^2} = \frac{b^2}{a} = a(1-\varepsilon^2)$ \\
$\varepsilon = \sqrt{1+ \frac{2\alpha E}{GMm}} = \frac{c}{a}$ \\
$T=2\pi \sqrt{\frac{a^3}{GM}}$
Perihelio $r_p = \frac{\alpha}{1 + \varepsilon}$ \\
Afelio $r_a = \frac{\alpha}{1 - \varepsilon}$
\\
\\

\textbullet Órbitas circulares \\
$V_{orb}= \sqrt{\frac{GM}{R_{orb}}}$ \\
$V_{esc}= \sqrt{\frac{2GM}{R_{T}}}$
\\
\\

\textbullet Órbitas circulares \\
$2a=b+c=r_p+r_a$ \\
$E=-\frac{GMm}{2a}$
$V_{esc}= \sqrt{\frac{2GM}{R_{T}}}$
\\
\\



\end{tabular}
\vspace{-1.5em}

% THIRD COLUMN PAGE TABLE
&
\section{Sistemas de partículas}

\begin{tabular}{@{}p{\linewidth}@{}}

\textbullet Ecuaciones centro masa\\
$\bar{r}_{CM}=\frac{\sum m_i\bar{r}_i}{M}$ \\
\\

\textbullet Ecuaciones \\
$\bar{p}=M\bar{v}_{CM}$ \\
$\bar{F}=M\bar{a}_{CM}$ \\
$F^{ext}=\frac{dp}{dt}$ \\
\\

\textbullet Conservación de variables \\
$\bar{L}= m_i\bar{r}_i\times \bar{v}_i$ \\
$\tau_i=\frac{dL}{dt}$\\
$\tau^{ext}=\frac{dL}{dt}$ \\
$E=K+U$ const. $\iff$ $F_i$ conserv. \\

\subsection{Choques}

\textbullet Choques elásticos \\
$p, K$ se conservan \\
$m_1(v_1-u_1)=m_2(u_2-v_2)$\\
$v_2-v_1=-(u_2-u_1)$ \\
$u_1=\frac{m_1-m_2}{m_1+m_2}v_1+\frac{2m_2}{m_1+m_2}v_2$ \\
$u_2=\frac{2m_1}{m_1+m_2}v_1+\frac{m_2-m_1}{m_1+m_2}v_2$ \\
\\

\textbullet Choques inelásticos \\
$p$ se conserva, $K$ no \\
$m_1(v_1-u_1)=m_2(u_2-v_2)$\\
$v_2-v_1=-e(u_2-u_1); \ 0\leq e<1$ \\
\\


\end{tabular}
\vspace{-1.5em}


\section{Masa variable}

\begin{tabular}{@{}p{\linewidth}@{}}

\textbullet  Conservación de $p$)\\
$dp=vdm + mdv -udm = 0$ \\
$ma=-v_{rel}\frac{dm}{dt}$ \\
$v= v_0+ v_{rel}ln(\frac{m(t)}{m_0})$
\\
\\

\end{tabular}
\vspace{-1.5em}


% FOURTH COLUMN PAGE TABLE
&

\section{Vectores}
\begin{tabular}{@{}p{\linewidth}@{}}

\hspace{0.5em} 
$F_x = F \cos\theta$ 
\\
\hspace{0.5em} 
$F_y = F \sin\theta$ 
\\
\hspace{0.5em} 
$\vec{F} = \vec{F_x} + \vec{F_y} = F\cos\theta_{\hat{i}} + F\sin\theta_{\hat{j}}$ 
\\
\hspace{0.5em} 
$F = \sqrt{F_x^2 + F_y^2}$ 
\\
\hspace{0.5em} 
$\theta = \tan^{-1} \frac{F_y}{F_x}$ 
\\
\\

\textbullet Distancia entre dos puntos en el espacio \\
\hspace{0.5em} 
$\left|P_1P_2\right| = $ 
\\
\hspace{0.8em} 
$\sqrt{(x_2-x_1)^2+(y_2-y_1)^2+(z_2-z_1)^2}$
\\
\\

\textbullet Vector unitario \\
\hspace{0.5em} 
$\hat{u} = \frac{\vec{QP}}{\left\|\vec{QP}\right\|} = \frac{P-Q}{\left\|P-Q\right\|}$ 
\\
\hspace{0.5em} 
$\vec{T} = T\hat{u}$ 
\\
\\


\textbullet Cosenos directores \\
\hspace{0.5em} 
$\theta_x = \cos^{-1}\frac{F_x}{F}$ 
\\
\\
\hspace{0.5em} 
$\theta_y = \cos^{-1}\frac{F_y}{F}$ 
\\
\\
\hspace{0.5em} 
$\theta_z = \cos^{-1}\frac{F_z}{F}$ 
\\
\\

\end{tabular}
\vspace{-1.5em}


\end{tabular}
\vspace{-1.5em}

\end{document}
