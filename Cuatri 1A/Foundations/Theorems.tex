\documentclass[12pt]{article}

\title{Important theorems}
\author{Abel Doñate}
\date{}

\usepackage{amsmath}
\usepackage{amssymb}
\usepackage{amsthm}
\usepackage{hyperref}
\usepackage{graphicx}
\graphicspath{ {./images/} }
\usepackage{wrapfig}

\setcounter{page}{0}

%Geometry
\usepackage{geometry}
\geometry{a4paper, margin=1in}



\begin{document}

\maketitle
\tableofcontents
\newpage

\section{Fermat's little theorem}
If $a$ and $p$ are coprimes, then:
\[a^{p-1}\equiv 1 \ (mod \ p)\]
\\
\textbf{Proof}\\
Consider the set $S=\{ a, 2a, \cdots, (p-1)a\}$.\\
Now we pick 2 elements of the set, for instance $ka, la$, with $1\leq k<l\leq p-1$. We are going to show that necessarily $ka \not\equiv la\ (mod \ p)$. Let's prove it by contradiction:\\
\\
If we suppose that it is true, then:
\[ka \equiv la\ (mod \ p) \ \ \implies p|(l-k)a \ \ \implies \begin{cases}
p|a \ \ \ \text{(impossible, they are coprimes)} \\
\text{or} \\
p|(l-k) \ \ \ \text{(impossible, it is positive and less than $p$)}
\end{cases} \]
So, it is proven by contradiction.\\
Thus, the residues of the elements of $S$ must be different from each other, so the set of the elements of $S$ in modulo $p$ is $S_p=\{1, 2, \cdots, p-1\}$, as it has to have the same number of elements that $S$ (the set does not has to be necessarily in order from $S$).\\

Now we multiply the elements of $S$ and the elements of $S_p$. If we consider the residues modulo $p$ of the results, they must be the same, because each element of $S_p$ is the residue of one element of $S$. Then:
\[a\cdot 2a \cdots (p-1)a = 1\cdot 2 \cdots (p-1) \ (mod \ p) \implies a^{p-1}(p-1)! = (p-1)! \ (mod \ p)  \]
Trivially $(p-1)!$ is coprime with p (they do not share any factor), so we can divide the expression by $(p-1)!$. As desired we end up with
\[a^{p-1}\equiv 1 \ (mod \ p)\]
\qed

\newpage

\section{Wilson's theorem}
Let $p$ any prime. Then it holds:
\[(p-1)!\equiv -1 \ (mod \ p)\]
\\
\textbf{Proof} \\
First we are going to proof two Lemmas.\\ \\
\textbf{Lemma 1.} If $a^2\equiv 1 \ (mod \ p)$, then $a \equiv 1 \ (mod \ p)$ or $a \equiv 1 \ (mod \ p)$. \\
The proof of this lemma is following:
\[a^2\equiv 1 \ (mod \ p) \implies p|a^2-1 \implies p|(a+1)(a-1) \implies \begin{cases}
p|(a+1) \\
\text{or} \\
p|(a-1) 
\end{cases}\]
so, a must be $1$ or $-1$ in modulo $p$. \\
\\
\textbf{Lemma 2.} Every number between $2$ and $p-2$ has a unique inverse that is not itself.\\
The proof is very simple. Using \textit{Lemma 1}, the only numbers that could be its own inverse are $1$ and $-1$. We know that, as every element must have an inverse ($\mathbf{Z_p}$ is a group), the inverse of the remaining elements must be different from themselves.\\
\\
Now we are ready for the proof. If we take $(p-1)!$, we can split it in this way
\[(p-1)!=1\cdot (2\cdot 3 \cdots (p-3) \cdot (p-2))\cdot (p-1)\]
Observe that in the middle remains the numbers between $2$ and $p-2$. Using \textit{Lemma 2} we can pair the elements in pairs formed by one element and its inverse (that is not itself) so that the product is $1 \ (mod \ p)$. Finally multiplying by $1$ and $(p-1)$ gives us \[(p-1)!\equiv -1 \ (mod \ p)\] as desired. \\
\qed



\end{document}
