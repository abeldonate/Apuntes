\documentclass[12pt]{article}

\title{Informática 1A}
\author{Abel Doñate}
\date{}

\usepackage{amsmath}
\usepackage{xcolor}

%Define the comand \code and \codecolor for programming font
\definecolor{light-gray}{gray}{0.95}
\newcommand{\code}[1]{\colorbox{light-gray}{\texttt{\textcolor{blue}{#1}}}}
\newcommand{\codeorange}[1]{\colorbox{light-gray}{\texttt{\textcolor{orange}{#1}}}}

\begin{document}

\maketitle
\tableofcontents
\newpage

\section{Console Shortcuts}
\begin{itemize}
	\item \codeorange{g++ program.cpp} to compile the program
	\item \codeorange{./a.out} to run the program
	\item \codeorange{python3 program.py}
	\item \codeorange{Ctr+c} breaks a \code{while()} loop
	\item \codeorange{Ctr+shift+c} copies information
	\item \codeorange{Ctr+shift+v} pastes information
\end{itemize}

\section{Include and Using}
\subsection{Using}
\code{using namespace std}
\subsection{Include}

To include all the standard libraries \code{\#include$<$bits/stdc++.h$>$} (DO NOT DO THAT)
\begin{itemize}
  \item \code{\#include$<$iostream$>$}
  \item \code{\#include$<$cmath$>$}
  	\begin{itemize}
  		\item \code{sqrt()}
  		\item \code{pow()}
  	\end{itemize}
  
  \item \code{\#include$<$algorithm$>$}
  	\begin{itemize}
  		\item \code{sort()}
  		\item \code{reverse(v.begin(), v.end())} reverses the string or vector
  	\end{itemize}
  	
  \item \code{\#include$<$iomanip$>$}
  	\begin{itemize}
  		\item \code{cout$<<$setprecision(n)$<<$fixed} fixes n decimal places in doubles.
  	\end{itemize}
  	
  \item \code{\#include$<$string$>$}
  	\begin{itemize}
  		\item \code{push\_back()} adds a char to the string.
  		\item \code{reverse(b.begin(), b.end())} reverses the string or vector (algorithm)
  	\end{itemize}
  	
\end{itemize}
  	
\section{Conversions}
\begin{itemize}
	\item From string to int \\
	\code{(int)s} s is the string
	\item From char to int \\
	\code{(int)c -'0'} c is the char
	\item From int to string \\
	\code{to\_string(n)} n is the int
	\item From int to char \\
	\code{'0' + n} n is the int
	\item From string to char \\
	\code{s[0]} s is the string
\end{itemize}

\section{Short	cuts}
\begin{itemize}
	\item \code{i++}
	\item \code{sum += v[i]}
\end{itemize}

\section{Vectors and matrices}
To define a vector \code{vector<int> v(size, default inputs)} \\
Similar with a matrix \code{vector<vector<int>> v(size x, vector<int>(size y))} \\
Usefull functions:
\begin{itemize}
	\item \code{v.size()} returns the size of the vector
	\item \code{v.push\_back()} adds a term at the end of the vector
	\item \code{v.pop\_back()}  removes the last term of the vector
	\item \code{v.clear()} cleans the vector
	\item \code{sort(v.begin, v.end)} sorts the vector
	\item \code{reverse(v.begin, v.end)} reverses the vector
	\item \code{swap(v[i], v[j])} swaps i and j term
\end{itemize}
To define a new type of valiable \code{typedef vector<int> vec} \\ or \code{using Row = vector<int>; using Matrix = vector<Row>;}

\section{Fundamental algorithms}
	\subsection{MergeSort}
	
	\subsection{structs}
	We can define our struct of Person as follows:\\
	\code{struct Person\{ string name; int age \};} \\ \\
	\code{bool comp(Person a, Person b)\{return a.age < b.age\} }\\
	\code{vector<Person> p;}\\
	\\
	Now we can sort \code{p} by ages: \\
	\code{sort(p1.begin(), p1.end(), comp)}
	


\section{Python}
Remember to construct the function \code{def main()} and call it later.
\subsection{Input and output}
\subsubsection{Output}
For the output \code{print()} 
\subsubsection{Input}
For the input first \code{from easyinput import read}. Then \code{read(\textit{type})} reads the input of the given type.\\ 
If you want to stop if no parameter is read use \code{while input != None}

\subsection{Fix Decimals}
To format the float a with n decimals
\code{a.formatted = "\{:.nf\}".format(a)}

\subsection{Conversions}
\begin{itemize}
	\item From int to string \\
	\code{str(n)} n is the int
	\item From string to int \\
	\code{int(s)} s is the string
\end{itemize}

\section{Tips for the Exam}
Here are some tips for the exam
\begin{itemize}
	\item Local valiables
	\item Minimum posible cases
	\item Do not use \code{break}. Instead, use a \code{bool}
	\item One line \code{if()}
\end{itemize}


\end{document}
