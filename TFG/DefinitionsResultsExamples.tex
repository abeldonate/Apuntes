\documentclass[leqno]{article}
\usepackage{verbatim}
\usepackage{array}
\usepackage{listings}
\usepackage{fancyvrb}
\usepackage{enumitem}

\usepackage[utf8]{inputenc}
\usepackage[T1]{fontenc}
\usepackage{textcomp}
\usepackage{multicol} \usepackage{mathtools}
\usepackage{amsmath}
\usepackage{wrapfig}
\usepackage{amssymb}
\usepackage{amsmath,amsfonts,amssymb,amsthm,epsfig,epstopdf,titling,url,array}
\usepackage{hyperref}
\usepackage{eso-pic}
\usepackage{pgf}
\usepackage{tikz}
\usepackage{tikz-cd}
\usepackage{graphicx}

% figure support
\usepackage{import}
\usepackage{xifthen}
\pdfminorversion=7
\usepackage{pdfpages}
\usepackage{transparent}
\usepackage{xcolor}

% geometry
\usepackage{geometry}
\geometry{a4paper, margin=1in}

% paragraph length
\setlength{\parindent}{0em}
\setlength{\parskip}{1em}

\newtheorem{theorem}{Theorem}
\newtheorem{lemma}{Lemma}
\newtheorem{proposition}{Proposition}
\newtheorem{definition}{Definition}
\newtheorem{observation}{Observation}
\newtheorem{example}{Example}

\newcommand{\incfig}[1]{%
\center
\def\svgwidth{0.9\columnwidth}
\import{./figures/}{#1.pdf_tex}
}
\newcommand{\incimg}[1]{%
\center
\includegraphics[width=0.9\columnwidth]{images/#1}
}

\DeclareMathOperator{\Hom}{\mathnormal{Hom}}
\DeclareMathOperator{\ord}{\mathnormal{ord}}

\pdfsuppresswarningpagegroup=1

\title{Definitions, results and examples}
\author{Abel Doñate Muñoz}
\date{}

\begin{document}
\maketitle
\tableofcontents
\newpage

\section{Rings}
\begin{definition}[Krull dimension] Supremum of the lengths of all chains of prime ideals. \[
\mathfrak{p}_0 \subsetneq \mathfrak{p}_1 \subsetneq \ldots \subsetneq \mathfrak{p}_n  \Rightarrow \dim R = n
\] 

\end{definition}

\begin{definition}[Regular ring] The minimal number of generators of its maximal ideal is the Krull dimension.

\end{definition}

\begin{definition}[Simple ring] Ring with no two-sided ideal besides zero and itself.
\end{definition}


\section{Modules}
\begin{definition}[Projective module] $P$ is projective if and only if for every surjective homomorphism $f:N\to M$ and every homomorphism $g:P\to M$, there exists a lifting $h:P\to N$ with the diagram commuting:
  % https://q.uiver.app/#q=WzAsMyxbMCwxLCJQIl0sWzEsMSwiTSJdLFsxLDAsIk4iXSxbMiwxLCJmIiwwLHsic3R5bGUiOnsiaGVhZCI6eyJuYW1lIjoiZXBpIn19fV0sWzAsMSwiZyJdLFswLDIsImgiLDAseyJzdHlsZSI6eyJib2R5Ijp7Im5hbWUiOiJkYXNoZWQifX19XV0=
\[\begin{tikzcd}
	& N \\
	P & M
	\arrow["f", two heads, from=1-2, to=2-2]
	\arrow["g", from=2-1, to=2-2]
	\arrow["h", dashed, from=2-1, to=1-2]
\end{tikzcd}\]
\end{definition}

\begin{proposition}[Characterizations of projective modules] The following are equivalent:
\begin{enumerate}[topsep=-6pt, itemsep=0pt]
  \item $P$ is projective.
  \item The SES $0\to A \to B\to P\to 0$ splits.
  \item $Hom(P, -)$ is an exact functor.
  \item $P$ is the direct sum of free modules.
\end{enumerate}
\end{proposition}

\begin{definition}[Flat module] $M$ is flat if and only if for every injective homomorphism $f:K\to L$, the map $f \otimes _R id : K\otimes _{R} M \to L\otimes _R M$ is injective, that is:
  % https://q.uiver.app/#q=WzAsNixbMCwwLCJLIl0sWzAsMSwiTCJdLFsyLDAsIktcXG90aW1lc19STSJdLFsxLDAsIlxcUmlnaHRhcnJvdyJdLFsxLDEsIlxcUmlnaHRhcnJvdyJdLFsyLDEsIkxcXG90aW1lcyBfUiBNIl0sWzAsMSwiZiIsMCx7InN0eWxlIjp7InRhaWwiOnsibmFtZSI6Imhvb2siLCJzaWRlIjoidG9wIn19fV0sWzIsNSwiZlxcb3RpbWVzIGlkIiwwLHsic3R5bGUiOnsidGFpbCI6eyJuYW1lIjoiaG9vayIsInNpZGUiOiJ0b3AifX19XV0=
\[\begin{tikzcd}
	K & \Rightarrow & {K\otimes_RM} \\
	L & \Rightarrow & {L\otimes _R M}
	\arrow["f", hook, from=1-1, to=2-1]
	\arrow["{f\otimes id}", hook, from=1-3, to=2-3]
\end{tikzcd}\]
\end{definition}

\begin{proposition}[Characterizations of flat modules] The following are equivalent:
\begin{enumerate}[topsep=-6pt, itemsep=0pt]
  \item $M$ is flat.
  \item $\otimes _RM$ is an exact functor.
\end{enumerate}
\end{proposition}

\begin{definition}[Torsion-free module] $M$ is torsion free if and only if its torsion submodule (the module with all the zero-divisors) is \{0\}:

\end{definition}

\begin{proposition}
In general we have the following implications of modules
\[
\text{Free} \Rightarrow \text{Projective} \Rightarrow \text{Flat} \Rightarrow \text{Torsion-free}
\] 
\end{proposition}

\begin{example}[Counterexamples of implications] Some counterexamples
\begin{itemize}[topsep=-6pt, itemsep=0pt]
  \item Projective $\nRightarrow$ Free. $\mathbb{Z} / 2\mathbb{Z}$ as  $\mathbb{Z} / 6\mathbb{Z}$-module.
  \item Flat $\nRightarrow$ Projective. $\mathbb{Q}$ as $\mathbb{Z}-$module.
  \item Torsion-free $\nRightarrow$ Flat. The ideal $I=(x,y)$ as $K[x, y]-$module.
\end{itemize}
\end{example}




\section{D-modules}

\begin{definition}[Ring / Module/ Weyl algebra] $A_n = \{\mathbb{C} \langle x_1, \ldots, x_n, \partial_1, \ldots, \partial_n \rangle \}$ that has the structure of a ring a module or an algebra.

\end{definition}

\begin{proposition} $A_n$ is:
  \begin{itemize}[topsep=-6pt, itemsep=0pt]
    \item A simple ring
	\item Noetherian
  \end{itemize}
\end{proposition}

\begin{proposition}
  Set of monomials $\mathcal{B}= \{x^\alpha \partial^\beta: \alpha , \beta \in \mathbb{N}^n\}$  is a basis of $A_n$. Then we can write every element as
  \[
	P = \sum_{\alpha , \beta } p_{\alpha \beta }x^\alpha \partial^\beta = \sum_{\beta } p_\beta (x) \partial^\beta 
  \] 
\end{proposition}

We denote $|\beta |=\sum \beta _i$.

\begin{definition}[Order and total order].
	\[
  \begin{tabular}{ccc}
	\text{Order} & \ord(P) = \max \beta  & \sigma (P) = \sum_{|\beta| = \ord(P) }p_{\beta }(x) \xi ^\beta \\
	\text{Total Order} & \ord^T = \max |\alpha +\beta | & \sigma (P)=\sum _{|\alpha +\beta |=\ord^T(P)} p_{\alpha \beta } x^\alpha \xi^\beta  
  \end{tabular}
	\] 
\end{definition}

\begin{definition}[Filtrations] Respectively Order and Total order filtrations:
  \[
	F_k(A_n)=\{P\in A_n : \ord(P)\le k\} , \qquad
	B_k(A_n)=\{P\in A_n : \ord^T(P)\le k\}
  \] 
\end{definition}

\section{F-modules}
For all this section $R$ is a commutative Noetherian ring with prime characteristic $p$.
\begin{definition}[Frobenius endomorphism] homomorphism $f:R\to R $ where $f(r)=r^p$
\end{definition}

\begin{definition}[Frobenius module] $M^{(e)}$ is the $R$-module $M$ endowed with the action  $r\cdot m= f^{e}(r)m$. We denote $M':=M^{(1)}$.
\end{definition}

\begin{definition}[Frobenius functor] The application $F$ that sends $M\to R'\otimes _RM$ and $\varphi :M\to N$ to $Id\otimes \varphi : R'\otimes _R M \to R'\otimes _RN $ is a functor.
\end{definition}

\begin{definition}[Frobenius powers] If $I=(x_1, \ldots, x_n)$ is an ideal or $R$. We define
   \[
	 I^{[p^e]}:= (x_1^{p^e}, \ldots, x_n^{p^e})R
  \] 
\end{definition}

\begin{proposition} $F^e(R / I)\cong  R / I^{[p^e]}$ 
\end{proposition}







\end{document}
