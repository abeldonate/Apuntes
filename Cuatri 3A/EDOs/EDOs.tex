\documentclass[leqno]{article}
\usepackage{verbatim}
\usepackage{array}
\usepackage{listings}
\usepackage{fancyvrb}
\usepackage{enumitem}

\usepackage[utf8]{inputenc}
\usepackage[T1]{fontenc}
\usepackage{textcomp}
\usepackage{multicol}
\usepackage{mathtools}
\usepackage{amsmath}
\usepackage{wrapfig}
\usepackage{amssymb}
\usepackage{amsmath,amsfonts,amssymb,amsthm,epsfig,epstopdf,titling,url,array}
\usepackage{hyperref}
\usepackage{eso-pic}
\usepackage{pgf}
\usepackage{tikz}
\usepackage{graphicx}

% figure support
\usepackage{import}
\usepackage{xifthen}
\pdfminorversion=7
\usepackage{pdfpages}
\usepackage{transparent}
\usepackage{xcolor}

% geometry
\usepackage{geometry}
\geometry{a4paper, margin=1in}

% paragraph length
\setlength{\parindent}{0em}
\setlength{\parskip}{1em}

\newtheorem*{theorem}{Theorem}
\newtheorem*{lemma}{Lemma}
\newtheorem*{proposition}{Proposition}
\newtheorem*{definition}{Definition}
\newtheorem*{observation}{Observation}

\newcommand{\incfig}[1]{%
\center
\def\svgwidth{0.9\columnwidth}
\import{./figures/}{#1.pdf_tex}
}
\newcommand{\incimg}[1]{%
\center
\includegraphics[width=0.9\columnwidth]{images/#1}
}
\pdfsuppresswarningpagegroup=1

\title{a}
\author{Abel Doñate Muñoz}
\date{}

\begin{document}
\maketitle
\tableofcontents
\newpage

\section{Ecuaciones diferenciales lineales}
\begin{definition}[EDO lineal]. Decimos que una  EDO es lineal si es de la forma $\dot{x} = A(t)x + b(t)$ \\
  Además se llama
  \begin{itemize}[topsep=-6pt, itemsep=0pt]
    \item Homogénea si $b(t) = 0$
    \item De coeficientes constantes si $A$ es constante
  \end{itemize}

\end{definition}

\begin{theorem}[Existencia y unicidad] Sea  $A$ una matriz con coeficientes continuos. Entonces hay una única solución $\varphi$ a
  \[
  \dot{x} = A(t)x + b(t), \qquad x(t_0) = x_0
  \] 
\end{theorem}

\begin{definition}[Flujo] Por el teorema anterior está bien definida la siguiente aplicación
  \[
  \varphi : I \times I \times \mathbb{R}^n \to  \mathbb{R}^n \quad \text{tal que} \qquad (t, t_0, x_0) \mapsto \varphi (t, t_0, x_0)
  \] 

\end{definition}

\begin{definition}[Matriz exponencial] $\displaystyle e^{tA}:= \sum \frac{A^kt^k}{k!}$
\end{definition}

\begin{theorem}[Forma canónica de Jordan] Sea $A \in \mathcal{M}_{n\times n}$. $\ \exists C\in \mathcal{M}_{n\times n}$ tal que
  \[
	J = C^{-1}AC = \begin{pmatrix} J_1 & & \\ & \ddots & \\ & & J_k \end{pmatrix}   \qquad \text{con} \qquad J_i =  \begin{pmatrix} \lambda_i & 1 & \cdots & 0 \\ & \lambda_i & 1 & \vdots \\ & & \lambda_i & 1 \\ & & & \lambda_i \end{pmatrix}  j
  \] 
\end{theorem}

\begin{proposition}
  A partir de la EDO $\dot{x} = Ax$ la podemos resolver con  $A= C^{-1}JC$ en
  \[
	x = e^{tA} = e^{tC^{-1}JC} = C^{-1}e^{tJ}C = C^{-1} \begin{pmatrix} e^{tJ_1} & & \\ & \ddots & \\ & & e^{tJ_k} \end{pmatrix} C
  \] 
\end{proposition}

\begin{proposition}
Sea $J$ un bloque de Jordan con  $\lambda$ en la diagonal. Entonces
\[
  e^{tJ} = \begin{pmatrix} 1 & t & \cdots & \frac{1}{(k-1)!}t^{k-1} \\ & 1 & \cdots & \frac{1}{(k-2)!}t^{k-2} \\ & & \ddots & \vdots \\ & & & 1  \end{pmatrix}  e^{t\lambda}
\] 
\end{proposition}

\subsection{Órbitas y Retratos de fase}

\begin{definition}[Órbita] Sea el sistema $\dot{x}=Ax, x_0\in \mathbb{R}^n$. Llamamos órbita de $x_0$ a
   \[
  \mathcal{O}(x_0) = \{\varphi (t, t_0, x_0)\in \mathbb{R}^n : t \in \mathbb{R}\}
  \] 
  Intuitivamente es el recorrido que hacen todas las soluciones de $\dot{x}=Ax$ que pasan en algún instante por $x_0$

\end{definition}

\subsection{Estabilidad}

\begin{definition}[Estabilidad] Sea $\dot{x}=A(t)x+b(t)$ con $\gamma(t), \tilde{\gamma}$ decimos que el sistema es:
  \begin{itemize}[topsep=-6pt, itemsep=0pt]
    \item \textbf{Estable.} $\ \forall t_0, \varepsilon >0 \ \exists \varepsilon : \|\gamma(t_0)-\tilde{\gamma}(t_0)\|<\delta \Rightarrow \|\gamma(t)-\tilde{\gamma}(t)\|<\varepsilon \ \forall t\ge t_0$
 \item \textbf{Asintóticamente estable.} Estable y $\ \forall t_0 \ \exists \varepsilon >0: \|\gamma(t_0)-\tilde{\gamma}(t_0)\|<\varepsilon$ para algún $t_0 \Rightarrow \lim \|\gamma(t)-\tilde{\gamma}(t)\| =0$
   \item \textbf{Inestable.} Si no es estable
  \end{itemize}
\end{definition}

\begin{theorem}[Estabilidad según espectro] Sea el sistema $\dot{x}=Ax$ estudiamos la estabilidad de la solución $x=0$
  \begin{enumerate}[topsep=-6pt, itemsep=0pt] 
    \item Si $Spec(A)\subseteq \{\Re(z)<0\} \Rightarrow $ Asintóticamente estable
    \item Si $\ \exists \lambda\in Spec(A): \Re(\lambda)>0 \Rightarrow $ Inestable
	\item Si $Spec(a)\subseteq \{\Re (z)\le 0\}$ y $Spec(A)\cap \{\Re(z)=0\}\neq \emptyset$ y las cajas de Jordan de $A$ con VAPs $\Re(\lambda)=0$ tienen tamaño $1 \Rightarrow $ Estable pero no Asintóticamente estable
	\item El resto de casos $\Rightarrow$ Inestable
  \end{enumerate}
\end{theorem}



\end{document}
