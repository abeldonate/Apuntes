\documentclass[leqno]{article}
\usepackage{verbatim}
\usepackage{array}
\usepackage{listings}
\usepackage{fancyvrb}
\usepackage{enumitem}
\usepackage{multicol}

\usepackage[utf8]{inputenc}
\usepackage[T1]{fontenc}
\usepackage{textcomp}
\usepackage{multicol}
\usepackage{mathtools}
\usepackage{amsmath}
\usepackage{wrapfig}
\usepackage{amssymb}
\usepackage{amsmath,amsfonts,amssymb,amsthm,epsfig,epstopdf,titling,url,array}
\usepackage{hyperref}
\usepackage{eso-pic}
\usepackage{pgf}
\usepackage{tikz}
\usepackage{graphicx}

% figure support
\usepackage{import}
\usepackage{xifthen}
\pdfminorversion=7
\usepackage{pdfpages}
\usepackage{transparent}
\usepackage{xcolor}

\setlength{\parindent}{0em}
\setlength{\parskip}{1em}

\newtheorem*{definition}{Definition}
\newtheorem*{theorem}{Theorem}
\newtheorem*{proposition}{Proposition}

\newcommand{\incfig}[1]{%
\center
\def\svgwidth{0.9\columnwidth}
\import{./figures/}{#1.pdf_tex}
}


% geometry
\usepackage{geometry}
\geometry{a4paper, margin=0.5in}


\newcommand{\incimg}[1]{%
\center
\includegraphics[width=0.9\columnwidth]{images/#1}
}
\pdfsuppresswarningpagegroup=1


\begin{document}

\begin{multicols}{2}[\columnsep2em]

\textbf{Sistemas lineales homogéneos} $\dot{x}=A(t)x$ 
\[
x = e^{At}c = Pe^{Dt}P^{-1}c = M(t)c = M(t)M(t_0)^{-1}x(t_0)
\] 

\textbf{Sistemas lineales no homogéneos} $\dot{x}=A(t)x + b(t)$ 
\[
x = x_h + x_p, \qquad x_p = M\int_{t_0}^t M(s)^{-1}b(s)ds
\] 

VAP real $\lambda_i \Rightarrow x = e^{\lambda_i t}v_i$

VAP complejo $\lambda = \alpha \pm \beta, \ v = u\pm wi$
\[
\begin{cases}
  x_1 = e^{\alpha t}(u\cos(\beta t)-w\sin(\beta t))\\
  x_2 = e^{\alpha t}(u\sin(\beta t)+w\cos(\beta t))
\end{cases}
\] 

Matriz exponencial del bloque de Jordan
\[
  e^{tJ} = \begin{pmatrix} 1 & t & \cdots & \frac{1}{(k-1)!}t^{k-1} \\ & 1 & \cdots & \frac{1}{(k-2)!}t^{k-2} \\ & & \ddots & \vdots \\ & & & 1  \end{pmatrix}  e^{t\lambda}
\] 

Cálculo en caso de Jordan
\begin{align*}
  &u \in Ker(A-\lambda I)^2 \Rightarrow (A-\lambda I)u = v\\
  &v \in Ker(A-\lambda I) \Rightarrow Av = \lambda v\\
  &x_1 = ue^{\lambda t} + vte^{\lambda t}, \quad x_2 = ve^{\lambda t}
\end{align*}

\begin{definition}[Estabilidad] Sea $\dot{x}=A(t)x+b(t)$ con $\gamma(t), \tilde{\gamma}$ decimos que el sistema es:
  \begin{itemize}[topsep=-6pt, itemsep=0pt]
    \item \textbf{Estable.} $\ \forall t_0, \varepsilon >0 \ \exists \varepsilon : \|\gamma(t_0)-\tilde{\gamma}(t_0)\|<\delta \Rightarrow \|\gamma(t)-\tilde{\gamma}(t)\|<\varepsilon \ \forall t\ge t_0$
 \item \textbf{Asintóticamente estable.} Estable y $\ \forall t_0 \ \exists \varepsilon >0: \|\gamma(t_0)-\tilde{\gamma}(t_0)\|<\varepsilon$ para algún $t_0 \Rightarrow \lim \|\gamma(t)-\tilde{\gamma}(t)\| =0$
   \item \textbf{Inestable.} Si no es estable
  \end{itemize}
\end{definition}

\begin{theorem}[Estabilidad según espectro] Sea el sistema $\dot{x}=Ax$ estudiamos la estabilidad de la solución $x=0$
  \begin{enumerate}[topsep=-6pt, itemsep=0pt] 
    \item Si $Spec(A)\subseteq \{\Re(z)<0\} \Rightarrow $ Asintóticamente estable
    \item Si $\ \exists \lambda\in Spec(A): \Re(\lambda)>0 \Rightarrow $ Inestable
	\item Si $Spec(a)\subseteq \{\Re (z)\le 0\}$ y $Spec(A)\cap \{\Re(z)=0\}\neq \emptyset$ y las cajas de Jordan de $A$ con VAPs $\Re(\lambda)=0$ tienen tamaño $1 \Rightarrow $ Estable pero no Asintóticamente estable
	\item El resto de casos $\Rightarrow$ Inestable
  \end{enumerate}
\end{theorem}
\hrule
Ecuaciones diferenciales de orden $n$
 \[
a_n(t)x^{(n)} + \cdots + a_1(t)x'(t) + a_0x(t) = f(x)
\] 

Wronskiano. Sean $y_1, \ldots , y_n$ soluciones de EDO
\begin{align*}
  & W(t) = \det \begin{pmatrix} y_1 & \cdots & y_n \\ \vdots & & \vdots \\ y^{(n-1)}_1 & \cdots & y^{(n-1)}_n  \end{pmatrix}  \\
  & W'(t) = - \frac{a_{n-1}(t)}{a_{n}(t)}W(t)
\end{align*}

Si $x_1$ es solución de la homogénea, otra solución es
 \[
x_2(t) =  x_1(t) \int \frac{e^{-\int a_1(t)dt}}{x_1(t)^2}dt
\] 

Fórmula de Liouville $\dot{\Phi}(t) = A(t)\Phi(t)$, $\Phi$ matriz
\[
\det (\Phi(t)) = \det(\Phi(t_0))e^{\int _{t_0}^t tr(A(s))ds}
\]

Variación de constantes $x_p = \sum u_jx_j, \ x_j$ cada homogénea
\begin{align*}
  u_j = \int_{t_0}^t \frac{W_j(s)}{W(s)}ds, \quad W_j: \text{ columna } j \leftrightarrow \begin{pmatrix} 0 \\ \vdots \\ f(x) \end{pmatrix}  
\end{align*}

\hrule

\textbf{Sistemas periódicos}

$A(t)$ T-peródica, $\Phi (t)$ matriz fundamental
\begin{align*}
  & \Phi (t + T) \text{ también fundamental} \\
  & M = \Phi (0)^{-1}\Phi (T) \text{ matriz monodromía}\\
  & \Phi (t+T)=\Phi (t)M
\end{align*}

\begin{theorem}[Estabilidad de sistema $T-peródico$] Sea $M$ la matriz de monodromía de  $\dot{x} = A(t)x$. Entonces
  \begin{enumerate}[topsep=-6pt, itemsep=0pt]
    \item Si $Spec(M)\subseteq D \Rightarrow $ asintóticamente estable
	\item Si  $Spec(M)\subseteq S$ y $M$ diagonaliza  $\Rightarrow$ estables, pero no asintóticamente estables
	\item Si $Spec(M)\not\subseteq \overline{D} $ o no diagonaliza $\Rightarrow$ inestables
  \end{enumerate}
\end{theorem}

\begin{theorem}[Floquet]
Sea $A$  $T-$periódica y  $M$ la matriz de monodromía de  $\dot{x}=Ax$. Sea $B : e^{TB}=M$. Entonces existe $P(t)$  $T-$periódica tal que  $P(t)e^{tB}$ es matriz fundamental del sistema
\[
x = P(t)y \text{ transforma } \dot{x}=Ax \text{ en } \dot{y} = By
\] 
\end{theorem}

VAPs de la matriz de monodromía
\[
\Pi \lambda_i = e^{\int_0^T tr(A(t))dt}
\] 

\textbf{Retratos de fase} $\displaystyle \begin{pmatrix} \dot{x}\\\dot{y} \end{pmatrix} = A \begin{pmatrix} x\\y \end{pmatrix}  $

Si $\lambda_1, \lambda_2 \neq 0$
\begin{enumerate}[topsep=-6pt, itemsep=0pt]
  \item $\lambda_1, \lambda_2 \in \mathbb{R}$
	\begin{itemize}[topsep=-6pt, itemsep=0pt]
	  \item Si $\lambda_1, \lambda_2>0 \Rightarrow $ Nodo repulsor
	  \item Si $\lambda_1, \lambda_2<0 \Rightarrow $ Nodo atractor
	  \item $\lambda_1<0<\lambda_2 \Rightarrow $ Sella
	\end{itemize}
  \item $\lambda = a\pm bi$
	\begin{itemize}[topsep=-6pt, itemsep=0pt]
	  \item Si $a>0 \Rightarrow$ Foco repulsor 
	  \item Si $a<0 \Rightarrow$ Foco atractor
	  \item Si $a=0 \Rightarrow$ Foco
	\end{itemize}
\end{enumerate}

Si $\lambda_1 = 0, \lambda_2 \neq 0 \Rightarrow$ Recta horizontal
	
Si $\lambda_1, \lambda_2 = 0\Rightarrow $ Recta 

\section{Teoremas fundamentales}
En toda la sección problema de Cauchy
\[
\begin{cases}
  \dot{x}=f(t, x)\\
  x(t_0) = x_0
\end{cases} \Rightarrow
x(t) = \mathcal{F}(x)(t)= x_0 + \int_{t_0}^t f(s, x(s))ds
\] 

\begin{theorem}[Arzela-Ascoli] Son equivalentes:
\begin{enumerate}[topsep=-6pt, itemsep=0pt]
  \item La familia $\mathcal{F}$ es puntualmente acotada y equicontinua en $K$
  \item De cada sucesión de elementos de $\mathcal{F}$ se puede extraer una parcial uniformemente convergente
\end{enumerate}
\end{theorem}

\begin{theorem}[Picard] Dados 
\begin{itemize}[topsep=-6pt, itemsep=0pt]
  \item $t_0\in \mathbb{R}, x_0 \in \mathbb{R}^n, a, b>0$
  \item $V_{a, b}=[t_0-a, t_0+a]\times \overline{B}_b(x_0)$ compacto
  \item $f:V_{a, b} \to  \mathbb{R}^n$ continua y $x-$Lipschitz
  \item $M\ge \|f\|=\max_{(t,x)\in V}\|f(t,x)\|$.
\end{itemize}
El pvi tiene solución única $\varphi : [t_0-\alpha , t_0+\alpha ] \to  \mathbb{R}^n$ con $\alpha = \min\{a, \frac{b}{M}\}$
\end{theorem}

\begin{proposition}
Sean  $I = [a,b]$ compacto, $f: I\times \mathbb{R}^n \to \mathbb{R}^n$ continua y $x-$Lipschitz. Entonces para cualquier $x_0, t_0$ el pvi tiene una única solución.
\end{proposition}

\begin{proposition}
Sea $f:(a,b)\times \mathbb{R}^n\to \mathbb{R}^n$ continua, $x-$Lipschitz. Entonces para cualquier  $t_0\in (a,b), x_0\in \mathbb{R}^n$ el pvi
\[
  \begin{cases}
\dot{x}=f(t,x)\\
x(t_0)=x_0
  \end{cases}
\] 
tiene una única solución $\varphi :(a,b)\to \mathbb{R}^n$
\end{proposition}

\begin{theorem}[Peano] Dados 
\begin{itemize}[topsep=-6pt, itemsep=0pt]
  \item $t_0\in \mathbb{R}, x_0 \in \mathbb{R}^n, a, b>0$
  \item $V_{a, b}=[t_0-a, t_0+a]\times \overline{B}_b(x_0)$ compacto
  \item $f:V_{a, b} \to  \mathbb{R}^n$ continua
  \item $M\ge \|f\|=\max_{(t,x)\in V}\|f(t,x)\|$.
\end{itemize}
El pvi tiene al menos una solución $\varphi : [t_0-\alpha , t_0+\alpha ] \to  \mathbb{R}^n$ con $\alpha = \min\{a, \frac{b}{M}\}$
\end{theorem}

\begin{definition}[Solución maximal] La solución $\varphi $ es maximal en $I$ si para cualquier otra solución $\tilde{\varphi }$ definida en $\tilde{I}$ tal que $I\subseteq \tilde{I}$, $\varphi = \tilde{\varphi }|_{I}$ 
\end{definition}

\begin{theorem} Sean $U\subseteq \mathbb{R}\times \mathbb{R}^n$ abierto, $f:U\to \mathbb{R}^n$ continua. Si $\varphi $ es maximal única de $\dot{x}=f(t,x)$ en el intervalo maximal $(\omega _-, \omega _+)$, entonces 
  \[
	(t,\varphi (t)) \xrightarrow[t \to \omega _{\pm}] \partial U
  \] 
  $\ \forall K\subseteq U$ compacto existen entornos $V_{\pm}$ de $\omega _{\pm}$ tal que $\varphi (t)\not\in K$ si $t\in V_{\pm}\cap (\omega _-,\omega _{+})$
\end{theorem}

\begin{definition}[Flujo de la EDO] Sea $\tilde{U} = \{(t, t_0, x_0, \lambda)\in \mathbb{R}^{1+1+n+p}: (t_0,x_0, \lambda)\in U, t\in (\omega_- (t_0, x_0, \lambda), \omega _+(t_0, x_0, \lambda)\}$
La aplicación $\varphi :\tilde{U}\to \mathbb{R}^n$ es el flujo
\end{definition}

\begin{theorem}[]
Sea $U \subseteq \mathbb{R}^{1+1+n+p}$ abierto y $f: U\to \mathbb{R}^n$ continua. Si el pvi tiene solución maximal $\varphi (t, t_0, x_0, \lambda)$ en el intervalo $I = (\omega _i(t_0,x_0,\lambda), \omega _+(t_0,x_0,\lambda))$ entonces $\tilde{U}$ es abierto y $\varphi :\tilde{U} \to  \mathbb{R}^n$ es continua.
\end{theorem}

\begin{theorem}[Lema de Gröwnwall]
Sean $u, v: [a,b]\to [0, \infty)$ continuas. Suponemos que existe $\alpha $ tal que
\[
u(t) \le \alpha + \int_{a}^t v(s)u(s)ds \quad \Rightarrow \quad u(t) \le \alpha e^{\int_a^tv(s)ds}
\] 
\end{theorem}

\begin{proposition} Sea $U$ abierto, $f$ continua y $L-$Lipschitz en  $x$. Entonces
  \[
  \|\varphi (t, t_0, x_0)-\varphi (t, t_0, \tilde{x}_0)\| \le e^{L|t-t_0|}\|x_0-\tilde{x}_0\|
  \] 
\end{proposition}

\begin{theorem}[] $M(t) = (\frac{d \varphi }{d x} )|_{(t,s,x,\lambda)}$ es la única solución del pvi
\[
  \begin{cases}
\frac{d }{d t} M(t) = \frac{\partial f}{\partial x}(t, \varphi (t,s,x,\lambda), \lambda) M(t)\\
M(s)=Id
  \end{cases}
\] 
\end{theorem}


\section{Teoría cualitativa}
\begin{definition}[Punto singular]
$p$ es un punto singular de $\dot{x}=f(x)$ si $f(p) = 0$
\end{definition}

\begin{proposition}
$\phi(t) = p$ es solución de $\dot{x}=f(x) \iff p$ punto singular de $f$. Además $\mathcal{O}(p)=\{p\}$
\end{proposition}

\begin{definition}[Órbita periódica]
$\mathcal{O}(p)$ de $\dot{x} = f(x)$ es periódica si la solución por $p$ e periódica.
\end{definition}

\begin{proposition}
Sea $\dot{x} = f(x)+ \varepsilon  g(t, x, \varepsilon )$ con $g$ T-periódica respecto  $t$. Suponemos  $\ \exists p: f(p)=0$. Entonces si $ \frac{2k\pi i}{T} \not\in spec(Df(p)) \Rightarrow \ \exists \varepsilon_0 >0 : \ \forall |\varepsilon |<\varepsilon _0$ el sistema tiene solución T-periódica $\gamma (t, \varepsilon )$ y $\|\gamma (t, \varepsilon )-p\|\le K|\varepsilon |$
\end{proposition}

\begin{definition}[Flujo]
\[
\begin{cases}
  \frac{d }{d t} \varphi (t, \tau , x) = f(t, \varphi (t, \tau , x))\\
  \varphi (\tau, \tau , x) = x
\end{cases}
\] 
\end{definition}

\begin{theorem}[Flujo con volumen fijo] Si consideramos $\nabla_{x}\cdot f(t, x) = Tr(\frac{d f}{d x} (t, x)) = 0$, entonces $\varphi $ preserva el volumen, es decir
  \[
  mesura(A) = mesura(\varphi_{t, \tau , A})
  \] 
\end{theorem}

\begin{definition}[Estabilidad] Liapunov
  \begin{itemize}[topsep=-6pt, itemsep=0pt]
    \item $\varphi $ estable. $\ \forall \varepsilon >0 \ \exists \delta>0: $ si $\psi $ solución también y  $\|\varphi (0)-\psi (0)\|<\delta \Rightarrow \|\varphi (t)-\psi (t)\|<\varepsilon $ 
	\item $\varphi $ asintóticamente estable. Estable y $\ \exists \delta>0: $ si $\|\varphi (0)-\psi (0)\|<\delta \Rightarrow \lim \|\varphi (t)-\psi (t)\| = 0$
  \end{itemize}
\end{definition}

\begin{theorem}[] Suponemos que $\ \forall \delta \ \exists \rho : $ si $ x\in B_{\rho } \Rightarrow \|g(t, x)\|<\delta\|x\| $ Consideramos
  \[
  \dot{x} = Ax + g(t, x)
  \] 
  Si todos los VAPS de $A$ tienen parte real negativa, la solución $\psi (t) = 0$ es asintóticamente estable.
\end{theorem}

\begin{definition}[Función de Liapunov] $p$ punto singular de $f$. $\tilde{U}\subseteq U$ entorno de $p$. La función $V: \tilde{U}\to \mathbb{R}$ diferenciable es de Liapunov si
  \begin{enumerate}[topsep=-6pt, itemsep=0pt]
    \item $V(p)=0, V(x)>0$ si $x\neq p$
	\item $\dot{V}(x)\le 0 \ \forall x\in \tilde{U}$ (estricta si $\le $)
  \end{enumerate}
\end{definition}

\begin{theorem}[] $p$ punto singular de  $\dot{x}=f(x)$. Si existe función de Liapunov, entonces $p$ es estable. Si la función es estricta, entonces  $p$ es asintóticamente estable.
\end{theorem}

\begin{definition}[Integral primera] Sea  $\varphi (t, x)$ el flujo de $\dot{x}=f(x)$. La función $I:U\to \mathbb{R}$ es integral primera del sistema si
  \[
  \frac{d }{d t} (I \circ \varphi )(t,x) = 0
  \] 
\end{definition}

\begin{proposition}
$I$ es integral primera de $\dot{x}=f(x) \iff$
\[
DI(x)f(x) = 0 \ \forall x\in U
\] 
\end{proposition}

\begin{definition}[Sistema Hamiltoniano]
$x=(q,p), f:U\to \mathbb{R}^n\times \mathbb{R}^n$. $\dot{x} = f(x)$ es Hamiltoniano si existe $H:U\to \mathbb{R}$ tal que
\[
\begin{cases}
  \dot{q}_j = \frac{\partial H}{\partial p_j}(q,p) \\
  \dot{p}_j = -\frac{\partial H}{\partial q_j}(q,p)
\end{cases}
\] 
\end{definition}

\begin{proposition}
$H$ hamiltoniano de  $\dot{x}=f(x) \Rightarrow H$ integral primera del sistema 
\end{proposition}


\section{Casuística}
\begin{definition}[Cambio a polares]
\[
\begin{cases}
  x = r\cos\theta \\
  y = r\sin\theta 
\end{cases} \Rightarrow 
\begin{cases}
  r' = \cos\theta x' + \sin \theta y'\\
  r\theta ' = \cos \theta y' - \sin \theta x'
\end{cases}
\] 
\end{definition}

\begin{definition}[Bernoulli]
\[
y' = a(x) + b(x)y^{r} \xRightarrow{z=y^{1-r}} lineal
\] 
\end{definition}

\begin{definition}[Ricatti] $y_1(x)$ sol conocida
\[
y' = a_0(x) + a_1(x)y + a_2(x)y^2 \begin{cases}
  \xRightarrow{y = y_1+z} Bernoulli\\
  \xRightarrow{y=y_1+\frac{1}{u}} lineal
\end{cases}
\] 
\end{definition}

\begin{definition}[Homogénea] $F(x,y)$ grad $0 \iff F = f(\frac{y}{x})$
  \[
  \frac{d y}{d x} = f(\frac{y}{x}) \xRightarrow{u(x) = \frac{y(x)}{x}} separable
  \] 
\end{definition}

\begin{definition}[Exactas] $Pdx + Qdy = 0$ con  $P_y=Q_x$. Entonces  $\ \exists U: U_x = P, U_y = Q$ tal que la solución es $U(x, Y) = c$
\end{definition}

\begin{definition}[Factores integrantes] buscamos $\mu(x, y)$ para que $\mu Pdx + \mu Qdy =0$ sea exacta.
\[
Q\mu_x-P\mu_y = \mu(P_y-Q_x)
\] 
\end{definition}

\begin{definition}[Lagrange] $y=xf(y')+g(y')$. Hacer cambio $p=y'$ y derivar con respecto a $x$ para encontrar
   \[
  \frac{d x}{d p}(p-f(p)) - f'(p)x=g'(p)  
  \] 
\end{definition}



\end{multicols}
\end{document}
