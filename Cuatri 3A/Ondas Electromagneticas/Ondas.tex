\documentclass[leqno]{article}
\usepackage{verbatim}
\usepackage{array}
\usepackage{listings}
\usepackage{fancyvrb}
\usepackage{enumitem}
\usepackage{esint}

\usepackage[utf8]{inputenc}
\usepackage[T1]{fontenc}
\usepackage{textcomp}
\usepackage{multicol}
\usepackage{mathtools}
\usepackage{amsmath}
\usepackage{wrapfig}
\usepackage{amssymb}
\usepackage{amsmath,amsfonts,amssymb,amsthm,epsfig,epstopdf,titling,url,array}
\usepackage{hyperref}
\usepackage{eso-pic}
\usepackage{pgf}
\usepackage{tikz}
\usepackage{graphicx}

% figure support
\usepackage{import}
\usepackage{xifthen}
\pdfminorversion=7
\usepackage{pdfpages}
\usepackage{transparent}
\usepackage{xcolor}

% geometry
\usepackage{geometry}
\geometry{a4paper, margin=1in}

% paragraph length
\setlength{\parindent}{0em}
\setlength{\parskip}{1em}

\newtheorem{theorem}{Theorem}
\newtheorem{lemma}[theorem]{Lemma}
\newtheorem{proposition}[theorem]{Proposition}
\newtheorem{definition}[theorem]{Definition}
\newtheorem{observation}[theorem]{Observation}

\newcommand{\incfig}[1]{%
\center
\def\svgwidth{0.9\columnwidth}
\import{./figures/}{#1.pdf_tex}
}
\newcommand{\incimg}[1]{%
\center
\includegraphics[width=0.9\columnwidth]{images/#1}
}
\pdfsuppresswarningpagegroup=1

\title{Ondas Electromagnéticas}
\author{Abel Doñate Muñoz}
\date{}

\begin{document}
\maketitle
\tableofcontents
\newpage

\section{Ondas en el vacío (densidades de carga y corriente nulas)}
\subsection{Ecuaciones de Maxwell}
\begin{center}
\begin{tabular}{|c|c|c|}
\hline
Ley de Gauss & $\nabla\cdot \mathcal{E} = 0$  & $\displaystyle\oiint \mathcal{E} \cdot  dS = 0$ \\
\hline
Ley de Gauss & $\nabla\cdot \mathcal{H} = 0$ & $\displaystyle \oiint \mathcal{H} \cdot  dS = 0$ \\
\hline
Ley de Maxwell-Faraday & $\displaystyle\nabla\times \mathcal{E} = -\mu_0 \frac{\partial \mathcal{H}}{\partial t}$ & $\displaystyle\oint \mathcal{E} \cdot  dl = -\mu_0 \frac{d}{dt} \iint \mathcal{H}\cdot dS$ \\
\hline
Ley de Ampere & $\displaystyle\nabla\times \mathcal{H} = \varepsilon _0 \frac{\partial \mathcal{E}}{\partial t}$ & $\displaystyle\oint \mathcal{H}\cdot dl = \varepsilon _0 \frac{d}{dt} \iint \mathcal{E} \cdot dS$ \\
\hline
\end{tabular}
\end{center}

De las ecuaciones de maxwell se pueden derivar las ecuaciones de onda donde $\displaystyle c = \frac{1}{\sqrt{\mu_0 \varepsilon _0} }$
\[
\nabla^2\mathcal{E} - \frac{1}{c^2} \frac{\partial \mathcal{E}}{\partial^2 t} = 0, \qquad 
\nabla^2\mathcal{H} - \frac{1}{c^2} \frac{\partial \mathcal{H}}{\partial^2 t} = 0
\] 

Si hacemos la asunción $\mathcal{E}(r, t) = \mathcal{E}_{x} (z, t)$ tenemos la ecuación de d'Alembert
\[
\frac{\partial}{}
\] 































\end{document}
