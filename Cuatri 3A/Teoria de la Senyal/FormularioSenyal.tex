\documentclass{myclass}
\usepackage{verbatim}
\usepackage{array}
\usepackage{listings}
\usepackage{fancyvrb}
\usepackage{enumitem}
\usepackage{multicol}
\usepackage{physics}

\usepackage[utf8]{inputenc}
\usepackage[T1]{fontenc}
\usepackage{textcomp}
\usepackage{multicol}
\usepackage{mathtools}
\usepackage{amsmath}
\usepackage{wrapfig}
\usepackage{amssymb}
\usepackage{amsmath,amsfonts,amssymb,amsthm,epsfig,epstopdf,titling,url,array}
\usepackage{hyperref}
\usepackage{eso-pic}
\usepackage{pgf}
\usepackage{tikz}
\usepackage{graphicx}

% figure support
\usepackage{import}
\usepackage{xifthen}
\pdfminorversion=7
\usepackage{pdfpages}
\usepackage{transparent}
\usepackage{xcolor}

\setlength{\parindent}{0em}
\setlength{\parskip}{1em}

\newtheorem*{definition}{Definition}
\newtheorem*{theorem}{Theorem}
\newtheorem*{proposition}{Proposition}

\newcommand{\incfig}[1]{%
{\center
\def\svgwidth{0.9\columnwidth}
\import{./figures/}{#1.pdf_tex}
\par}
}


% geometry
\usepackage{geometry}
\geometry{a4paper, margin=0.5in}


\newcommand{\incimg}[1]{%
\includegraphics[width=0.9\columnwidth]{images/#1}
}
\pdfsuppresswarningpagegroup=1


\begin{document}

\begin{multicols}{2}[\columnsep2em]
Author: Abel Doñate Muñoz
\section{Tema 1: Señales}

Types of signals
\begin{align*}
  \Pi(t) &= 1 \text{ in } t\in (-\frac{1}{2}, \frac{1}{2}) \\
  \Delta(t)&= 1-|t| \text{ in } |t|<1 \\
  sinc(t) &= \frac{\sin(\pi t)}{\pi t}
\end{align*}

Types of systems
\begin{itemize}[topsep=-6pt, itemsep=0pt]
  \item Linear: $ax_1+bx_2 \to ay_1+ay_2$
  \item Invariant: $x(t-\Delta ) \to y(t-\Delta)$
  \item Casual: depende de las anteriores 
  \item Stable: $|x(t)|<M_x \Rightarrow |y(t)|<M_y$
\end{itemize}

Energy and power
\[
E = \int |x(t)|^2dt, \qquad P = \lim \frac{1}{T} \int_{-\frac{T}{2}}^{\frac{T}{2}} |x(t)|^2dt
\] 

Linear time-invariant (LTI)
\[
y(t) = x(t)\ast h(t) = \int x(\tau )h(t-\tau )d\tau = \sum x[i]h[n-i]
\] 

Differences equation LTI
\[
y[n] = \sum_0^M b_jx[n-j] - \sum_1^N a_iy[n-i] \Rightarrow H(z) = \frac{\sum_{k=0}^M b_kz^{-k}}{\sum_{k=0}^{N}a_kz^{-k}}
\] 
Forma directa 1
\incfig{DirectForm1}

Forma directa 2 ($N=M$)
\incfig{DirectForm2}

\section{Tema 2: Transformadas}

Transforms
\begin{align*}
  x(t) &\to X(\omega )= \int x(t)e^{-j\omega t}dt \qquad x(t)= \frac{1}{2\pi}\int X(\omega )e^{j\omega t} \\
  x[n] &\to X(z) = \sum x[n]z^{-n}
\end{align*}

Fourier series
\begin{align*}
	x(t) &= \frac{a_0}{2} + \sum \left( a_n\cos(n\omega _0t) + b_n \sin(n\omega _0t )\right)\\
		 &= A_0 + \sum A_n\cos(n\omega _0t + \varphi_n) = \sum c_n e^{jn\omega_0 t}\\
	a_n &= \frac{2}{T}\int_T x(t)\cos(n\omega _0t)dt,   b_n = \frac{2}{T}\int_T x(t)\sin(n\omega _0t)dt\\
	A_n &= \sqrt{a_n^2 + b_n^2}, \varphi_n = \arctan(\frac{-b_n}{a_n}) \\
	c_n &= \frac{a_n-jb_n}{2}, \quad c_{-n} = c_n^*\\
	P_m &= \frac{1}{T}\int_T|x|^2dt= \sum |c_n|^2= A_0^2 + \sum A_n^2
\end{align*}

LTI response to $x(t)=\sum A_n \cos(n\omega _0t + \varphi _n)$
\[
y(t) = \sum A_n|H(n\omega _0)|\cos(n\omega_0t+ \varphi + \angle H(n\omega _0))
\] 

Fourier transform properties
\begin{align*}
&x(at)\to  \frac{1}{|a|}X(\frac{\omega}{a}), \qquad x(t) \to  \to  2\pi x(\omega) \\
&x(t-\Delta) \to X(\omega )e^{-j\omega \Delta}, \qquad x(t)e^{j\omega_0 t} \to X(\omega -\omega _0)\\
&\frac{d x(t)}{d t} \to j\omega X(\omega), \qquad \int_{-\infty}^tx(\tau)d\tau \to \frac{X(\omega )}{j\omega  }+ \pi X(0)\delta(\omega ) \\
&f(t)\ast g(t) \to F(\omega )G(\omega ), \qquad f(t)g(t) \to \frac{1}{2\pi}F(\omega )\ast G(\omega )
\end{align*}

\begin{center}
\begin{tabular}{cc}
  \textbf{Signal} & \textbf{Fourier transform}\\
  $\delta(t)$ & $1$ \\
  $1$ & $2\pi \delta(\omega)$\\
  $u(t)$ &  $\pi \delta(\omega ) + \frac{1}{j\omega }$ \\
  $\Pi (\frac{t}{T})$ & $Tsinc(\frac{\omega T}{2\pi})$ \\
  $\Delta  (\frac{t}{T})$ & $Tsinc^2(\frac{\omega T}{2\pi})$ \\
  $\cos(\omega _0t)$ & $\pi\delta(\omega -\omega _0)+ \pi\delta(\omega +\omega _0)$\\
  $\sin(\omega _0t)$ & $-j\pi\delta(\omega -\omega _0)+ j\pi\delta(\omega +\omega _0)$\\
  $sinc(\frac{\omega _0 t}{\pi})$ & $\frac{\pi}{\omega _0}\Pi(\frac{\omega }{2\omega _0})$\\
  $sign(t)$ &   $\frac{2}{j\omega }$\\
  $e^{-at}u(t)$ & $\frac{1}{a+j\omega }$ \\
  $\frac{1}{t}$ & $-j\pi sign(\omega)$
\end{tabular}
\end{center}

Z-transform properties
\begin{align*}
  &x[n-n_0] \to z^{-n_0}X(z), \quad x[-n] \to X(\frac{1}{z})\\
  &a^nx[n] = X(\frac{z}{a}), \quad nx[n] \to -z\frac{d X(z)}{d z} 
\end{align*}

\begin{center}
\begin{tabular}{cc}
  \textbf{Signal} & \textbf{Z-transform}\\
  $\delta[n]$ & $1$ \\
  $p_N[n]$ &  $\frac{1-z^{-N}}{1-z^{-1}}$ \\
  $u[n]$ &  $\frac{1}{1-z^{-1}}$ \\
  $\cos(\Omega_0n)u[n]$ & $\frac{1-\cos(\Omega_0)z^{-1}}{1-2\cos(\Omega _0)z^{-1}+z^{-2}}$\\
  $\sin(\Omega_0n)u[n]$ & $\frac{\sin(\Omega_0)z^{-1}}{1-2\cos(\Omega _0)z^{-1}+z^{-2}}$
\end{tabular}
\end{center}

Compute inverse Z-transform
\begin{align*}
  & X(z) = \frac{d_0+d_1z^{-1}+\cdots d_Mz^{-M}}{c_0 + c_1z^{-1} + \cdots c_Nz^{-N}} \\
  & X(z) = \sum_{r=0}^{M-N} B_rz^{-r} + \sum_{k=1}^{N} \frac{A_k}{1-p_kz^{-1}}\\
  & x[n] = \sum_{r=0}^{M-N} B_r \delta(n-r) + \sum_{k=1}^{N}A_kp_k^n u[n]
\end{align*}

Connections

Serie: $H = H_1H_2$ \quad Parallel:  $H = H_1+H_2$

Estabilidad
\begin{itemize}[topsep=-6pt, itemsep=0pt]
  \item Estable si $p_k \in D \ \forall k$
  \item Marginalmente estable si $p_k \in \overline{D} \ \forall k$ 
  \item Inestable si $\ \exists k: p_k \not\in \overline{D}$ 
\end{itemize}

DFT
\begin{align*}
  & X_N[k]  = DFT_{N}[x[n]] = X(e^{j\Omega})|_{\Omega = k \frac{2\pi}{N}} =\\
  &= \sum_{n=0}^{N-1}x[n]e^{-j\Omega n} |_{\Omega =k \frac{2\pi}{N}} = \sum_{n=0}^{N-1}x[n]e^{-j \frac{2\pi}{N}kn} \\
  & x_N[n] = \frac{1}{N}\sum_{k=0}^{N-1} X_N[k]e^{j \frac{2\pi}{N}kn}
\end{align*}

Properties
\begin{align*}
  x[n]e^{j \frac{2\pi}{N}k_0n} &\to \tilde{X}_{N}[k-k_0]p_N[k]\\
  x[n]\cdot y[n] &\to \frac{1}{N} X_N[k]\circ Y_N[k]\\
  x[n] \circ y[n] & \to  X[n]Y[n]\\
  \tilde{x}_N[n-m]\cdot p_N[n] &\to X_N[k]e^{-j \frac{2\pi}{N}km}
\end{align*}

DFT of $A\cos(2\pi \frac{k_0}{N})$ :
\[
  X(k) = 
\begin{cases}
  \frac{AN}{2} &\text{ si } k = k_0, N-k_0 \\
  0 &\text{ otherwise}
\end{cases}
\] 

\section{Tema 3: Conversores A/D y D/A }
\begin{center}
\begin{tabular}{cc}
  $f_s=\frac{1}{T}$ & frecuencia de sampleo (Hz)\\
  $\omega _2 = \frac{2\pi}{T}=2\pi f_s$ & frecuencia de sampleo (rad/s)\\
  $\omega _m$ & frecuencia máxima
\end{tabular}
\end{center}

Criterio de Nyquist $\omega_s>2\omega _m$

Antialiasing filter $H(\omega )=\Pi(\frac{\omega }{\omega _s}), \ \Omega  = \omega T$

Tipos de interpolación
\begin{center}
\begin{tabular}{cc}
  Linear & $h(t)=\Lambda(\frac{t}{T}), \ $ \\
  ZOH & $h(t)=\Pi(\frac{t-T /2}{T}), \ H(\omega ) = Tsinc( \frac{\omega}{\omega _s})e^{-j\omega \frac{T}{2}}$\\
  Ideal & $h(t)=sinc\left(\frac{t}{T}\right), \ H(\omega )=\Pi (\frac{\omega }{\omega _s})$
\end{tabular}
\end{center}

Transformada señal digital
\begin{align*}
  p(t) = \sum \delta(t-nT) \Rightarrow P(\omega ) = \frac{2\pi}{T}\sum \delta (\omega -\frac{2\pi}{T}k) \\
  y(t)=x(t)p(t) \Rightarrow Y(\omega ) = \frac{1}{T}\sum X(\omega -\frac{2\pi}{T}k)
\end{align*}

Diezmado
\incfig{Diezmado}

Interpolacion
\incfig{Interpolacion}

\section{Tema 4: Random signals}

Cumulative distrubution function (cdf) and probability density function (pdf)
\[
F_X(x_1;t_1) = P(X(t_1)\le x_1), \quad f_X(x_1;t_1) = \frac{\partial F_X(x_1; t_1)}{\partial x_1}
\] 

\begin{definition}[Mean]
\[
m_X(t) = E[X(t)] = \int_{-\infty}^\infty xf_X(x, t)dx
\] 
\end{definition}

\begin{definition}[Auto-correlation]
\begin{align*}
  & r_X(t_1, t_2) = E[X(t_1)X^*(t_2)] = \int \int x_1x_2^* f_X(x_1, x_2; t_1, t_2)dx_1dx_2 \\
  & \overline{r}_{X}(\tau ) = \frac{1}{T} \int_{0}^T r_X(t+\tau ,t) dt
\end{align*}
\end{definition}


\begin{definition}[Instantaneous power]
\[
P_X(t) = E[|X(t)|^2] = r_X(t,t)
\] 
\end{definition}

\begin{definition}[Auto-covariance]
\[
c_X(t_1, t_2) = r_X(t_1, t_2) - m_X(t_1)m_X(t_2)
\] 
\end{definition}

\begin{definition}[Variance]
\[
\sigma^2_{X}(t) = c_X(t,t) = r_X(t, t) - m_X(t)m_X^*(t)
\] 
\end{definition}

\begin{definition}[Cross-correlation] Orthogonal $ r_{XY} = 0$
\[
r_{XY}(t_1, t_2) = E[X(t_1)Y^*(t_2)]= \int\int xy^*f_{XY}(x, y; t_1, t_2)dxdy
\] 
\end{definition}

\begin{definition}[Cross-Covariance] Uncorrelated $ c_{XY} = 0$
\[
c_{XY}(t_1,t_2) = r_{XY}(t_1,t_2) - m_X(t_1)m_Y^*(t_2)
\] 
\end{definition}

Independence $\Rightarrow$ Uncorrelation
\begin{align*}
  & \text{Independence} \iff f_{XY}(x, y) = f_X(x)f_Y(y) \\
  & \text{Uncorrelation} \iff E[XY] = E[X]E[Y]
\end{align*}

\begin{definition}[Stationary] .
  \begin{itemize}[topsep=-6pt, itemsep=0pt]
    \item 1st order:  $f_X(x;t)=f(x; t+\Delta ) \ \forall t, \Delta $
    \item 2nd order:  $f_X(x_1, x_2;t_1,t_2)=f(x_1,x_2; t_1+\Delta, t_2+\Delta ) \ \forall t_1, t_2, \Delta $
	\item WSS  $m_X(t) = cte, \quad r_X(t_1, t_2) = r_X(\tau )$
	\item jointly WSS  $X, Y $ WSS$, \quad r_{XY}(t_1, t_2) = r_X(\tau )$
	\item Cyclostationary $m_x(t), r_X(t, t+\tau $ periodic
  \end{itemize}
\end{definition}

Properties
\begin{align*}
& r_{X}(\tau ) = r^*_{X}(-\tau ), \quad r_{XY}(\tau ) = r^*_{XY}(-\tau )\\
& |r_{XY}(\tau )| \le \sqrt{r_X(0)r_Y(0)}\\
& r_{X+Y} = r_X(\tau ) + r_Y(\tau ) + r_{XY}(\tau ) + r_{YX}(\tau )
\end{align*}

\begin{definition}[Power spectral density (PSD)]
\[
S_X(\omega ) = \lim_{T\to  \infty} \frac{E[|X_T(\omega , S)|^2]}{2T}
\]
\end{definition}

\begin{theorem}[Wiener-Khinchin]
  $S_X(\omega )=F[\overline{r}_X(\tau )]$
\end{theorem}

Properties of PSD
\begin{align*}
  & S_X(\omega ) \ge 0, \quad S_X(\omega ) = S_X(-\omega ) \\
  & P_X = \frac{1}{2\pi} \int_{-\infty}^{\infty} S_X(\omega ) d\omega 
\end{align*}

WSS process before filter $H(\omega )$
\begin{itemize}[topsep=-6pt, itemsep=0pt]
  \item Mean $m_Y = m_xH(0)$
  \item  Cross-correlation  $r_{YX}(\tau ) = r_X(\tau )\ast h(\tau )$ 
  \item Auto-correlation $r_Y(\tau )=r_X(\tau )\ast h(\tau )\ast h^*(-\tau )$
  \item Spectral density $S_Y(\omega ) = S_X(\omega )|H(\omega )|^2$
\end{itemize}

\begin{definition}[Auto-correlation matrix]
\[
  R_X=  \begin{pmatrix} r_X[0] & r_X[1] & \cdots & r_X[L-1] \\
	r^*_X[1] & r_X[0] & \cdots & r_X[L-2] \\
	\vdots & \vdots & \ddots & \vdots \\
	r^*_X[L-1] & r_X^*[L-2] & \cdots & r_X[0]
\end{pmatrix} 
\] 
\end{definition}

Properties of $R_X$
 \begin{itemize}[topsep=-6pt, itemsep=0pt]
  \item $T_X = Q\Lambda Q^H$
  \item  $P_Y=h^HR_Xh = \sum \lambda_i|h^Hq_i|^2$
\end{itemize}

\begin{definition}[Gaussian random variable]
\begin{align*}
  &f_X(x)=\frac{1}{\sqrt{2\pi\sigma _X^2}} e^{-\frac{(x-m_x)^2}{2\sigma ^2_{X}}}\\
  &f_X(x)=\frac{1}{\pi\sigma _X^2} e^{-\frac{|x-m_x|^2}{\sigma ^2_{X}}} \\
  &f_X(x)=\frac{1}{(2\pi)^{L /2}|C_X|^{1 /2}} e^{-\frac{1}{2}(x-\mu_X)^TC_X^{-1}(x-\mu_X)}\\
  &f_X(x)=\frac{1}{(\pi)^{L}|C_X|} e^{-(x-\mu_X)^HC_X^{-1}(x-\mu_X)}
\end{align*}
\end{definition}

\section{Tema 5: Estimation theory}

\begin{definition}[Bias] $b_{\hat{\theta}} = E[\hat{\theta} (x)-\theta ] = \mu_{\hat{\theta }} -\theta $
\end{definition}

\begin{definition}[Covariance] 
  \[
    C_{\hat{\theta }} = E[(\hat{\theta }(x)-\mu_{\hat{\theta }})(\hat{\theta }(x)-\mu_{\hat{\theta }})^H], \qquad \sigma_{\hat{\theta}_i}^2 = (C_{\hat{\theta }})_{i,i}
  \] 
\end{definition}

\begin{definition}[Mean square error (MSE)]
  \[
    M_{\hat{\theta }} = E[(\hat{\theta }(x)-\theta )(\hat{\theta }(x)-\theta )^H] = C_{\hat{\theta }} + b_{\hat{\theta }}b_{\hat{\theta }}^H
  \] 
\end{definition}

\begin{definition}[Minimum Variance Unbiased Estimator (MVUE)]
$b_{\hat{\theta }}=0, \quad \sigma_{\hat{\theta }}^2|_{min}$
\end{definition}

\begin{definition}[Sharpness] $\displaystyle S = -E\left[\frac{\partial^2 \ln (f_{\theta }(x))}{\partial \theta ^2}\right]$
\end{definition}

\begin{definition}[Cramer-Rao Lower Bound (CRLB)]
\[
\sigma ^2_{\hat{\theta }} \ge \frac{1}{E\left[ \left| \frac{\partial ln (f_\theta (x)}{\partial \theta^* } \right|^2  \right]} = \frac{1}{-E\left[ \frac{\partial^2 ln (f_\theta (x)}{\partial \theta^* \partial\theta }  \right]}
\] 
\end{definition}

\begin{definition}[Efficient estimator]
\[
\hat{\theta (x)}-\theta = CRLB(\theta )\cdot \frac{\partial \ln (f_\theta (x))}{\partial \theta } 
\] 
\end{definition}

\begin{definition}[Fisher information matrix]
\[
  (J(\theta ))_{i, j} = E \left[ \frac{\partial^2 \ln f_\theta (x)}{\partial \theta^* _i \partial\theta _j} \right] = -E \left[ \frac{\partial \ln f_\theta (x)}{\partial \theta^* _i} \frac{\partial \ln f_\theta (x)}{\partial\theta _j} \right]
\] 
\end{definition}

\begin{definition}[CRLB for multiple parameters] 
\[
  \sigma ^2_{\hat{\theta}_{i}}\ge (J^{-1}(\theta ))_{i, i}
\] 
\end{definition}

\begin{definition}[Maximum Likelihood Estimator (MLE)]
\[
\hat{\theta }_{ML}(x) = \arg (\max_{\theta }f_\theta (x))
\] 
\end{definition}

\begin{definition}[Maximum A Posteriori (MAP)]
\[
\hat{\theta }_{MAP}(x) = \arg (\max_{\theta }f_{\theta |x}(\theta |x))= \arg (\max_{\theta } f_{x|\theta }(x|\theta )f_\theta (\theta ))
\] 
\end{definition}

\begin{definition}[Minimum Mean Square Error (MMSE)]
\[
\hat{\theta }_{MMSE}(x) = \int \theta f_\theta (\theta ) f_{x|\theta }(x|\theta )
\] 
\end{definition}




\section{Tema 6: Spectral estimation}
\begin{definition}[Windowed sequence] 
  \[
  x_v[n]=x[n]v[n]=x[n]p_N[n], \quad |V(e^{j\Omega})|=\frac{\sin(N\Omega / 2)}{\sin(\Omega  / 2)}
  \] 
\end{definition}

\begin{definition}[Triangular window] 
  \[
    w[m]=\frac{1}{N}v[m]*v[-m] , \quad W(e^{j\Omega})=\frac{1}{N}\left|V(e^{j\Omega})\right|^2 
  \] 
\end{definition}

\begin{definition}[Biased estimation of auto-correlation]
  \begin{align*}
  \hat{r}_X[m] = \frac{1}{N} x_v[m]\ast x_v^*[-m] = \frac{1}{N}\sum x[n+m]x^*[n]v[n+m]v[n]\\
  E[\hat{r}_X[m]] = \left( 1-\frac{|m|}{N} \right) r_X[m],\ |m|\le N-1; \ 0 \text{ otherwise} 
  \end{align*}
\end{definition}



\begin{definition}[Periodogram] Biased
  \begin{align*}
	\hat{S}_p(e^{j\Omega })&=F[\hat{r}[m]]=\frac{1}{N}|X_v(e^{j\Omega })|^2 \\
	E[\hat{S}_p(e^{j\Omega })] &= \frac{1}{2\pi}\int_{-\pi}^{\pi}S_X(e^{j(\Omega-\theta )})\frac{1}{N}W(e^{j\theta })d\theta 
  \end{align*}
\end{definition}

\begin{definition}[Variance] (Very complicated, approximation)
\[
Var[\hat{S}_p(e^{j\Omega})]\approx (S_x(e^{j\Omega}))^2
\] 
\end{definition}

\begin{definition}[Estimation of the power] Unbiased
  \begin{align*}
    \hat{P}_X = \frac{1}{2\pi}\int_{-\pi}^{\pi} \hat{S}_p(e^{j\Omega})d\Omega = \frac{1}{N}\sum|x[n]|^2 \\
	E[\hat{P}_{X}] = E[\hat{r}_X[0]]=P_X
  \end{align*}
\end{definition}

\begin{definition}[Unbiased estimator of auto-correlation]
\begin{align*}
  &\breve{r}_X[m]= \begin{cases}
    \frac{1}{N-|m|}\sum_0^{N-|m|-1}x[n+|m|]x^*[n],\ |m|\le N-1\\
	0 \text{ otherwise}
  \end{cases}\\
  &E[\breve{r}_X[m]]=r_X[m] \text{ si } |m|\le N-1
\end{align*}
\end{definition}

\begin{definition}[Modified Periodogram] (Any window)
\[
  \hat{P}_X=\frac{1}{N}\sum_{0}^{N-1}|v[n]|^2|x[n]|^2
\] 
\end{definition}

\begin{definition}[Blackman-Tukey]
Biased estimate of the auto-correlation ($\hat{r}_X[m], |m|\le N-1$) $\Rightarrow$ Windowing $\hat{r}_X[m]w_a[m], |m|\le L-1$ $\Rightarrow$ Fourier transform  $\hat{S}_{BT}(e^{j\Omega})=F[\hat{r}_X[m]w_a[m]]$ 

The estimator is asymptotically unbiased if $N, L\to \infty , w_a[0]=1$
\end{definition}

\begin{definition}[Variance BT] $E_{W_a}$ energy of the window
\[
Var[\hat{S}_{BT}(e^{j\Omega})] \approx \frac{E_{W_a}}{N}S^2_X(e^{j\Omega})
\] 
\end{definition}

(BARLETT-WELCH 36)
\begin{definition}[Models] $w[n] \rightarrow \boxed{h(z)} \rightarrow x[n], r_w[0]=\sigma ^2$
\begin{align*}
  AR(p) \Rightarrow H(z)=\frac{1}{1+\sum_{1}^{p}a_kz^{-k}}\\
  MA(q) \Rightarrow H(z) = 1+\sum_1^q b_kz^{-k}\\
  ARMA(p,q) \Rightarrow H(z) = \frac{1+\sum_1^q b_kz^{-k}}{1+\sum _1^pa_kz^{-k}}
\end{align*}
\end{definition}

\begin{definition}[Yule-Walker equations]
\[
R_x  \begin{pmatrix} a_1 \\ \vdots \\ a_p \end{pmatrix} = - \begin{pmatrix} r_x[1] \\ \vdots \\ r_x[p] \end{pmatrix} , \quad \sigma ^2 = r_x[0]+\sum_1^pa_kr_x[-k]
\] 
\end{definition}


\section{Tema 7: Wiener Filtering}
\incfig{wiener}
\[
g = \vb{h}^H = [h^*[0], \ldots, h^*[Q-1]], \quad \vb{x}[n] = [x[n], \ldots, x[n-Q+1]]^T
\] 

\begin{definition}[Linear MSE estimator] 
\[
y[n]=\vb{h}^H\vb{x}[n], \quad e[n] = d[n]-y[n]
\] 
\end{definition}

\begin{theorem}[MSE]
 \[
\xi = E[|e[n]|^2] = P_d + \vb{h}^H\vb{R}_x\vb{h} - \vb{h}^H\vb{r}_{xd} -\vb{r}_{xd}^H\vb{h}
\] 
\end{theorem}

\begin{definition}[Wiener-Hopf equations](minimize MSE)
\[
\vb{R}_x\vb{h}_{opt}= \vb{r}_{xd} \Rightarrow \vb{h}_{opt}=\vb{R}_x ^{-1}\vb{r}_{xd}
\] 
\end{definition}

\begin{proposition}MSE not using Wiener filter 
\begin{align*}
&\xi = \xi _{min} + (\vb{h}-\vb{h}_{opt})^H\vb{R}_x(\vb{h}-\vb{h}_{opt}) \\
&E[\xi ]= \xi_{min} + trace(\vb{R}_xE[(\vb{h}-\vb{h}_{opt})(\vb{h}-\vb{h}_{opt})^H])
\end{align*}
\end{proposition}


\textbf{Linear prediction}
\incfig{predictor}

\begin{align*}
  & y[n] = \hat{x}[n] = \vb{h}^H\vb{x}[n-\Delta], \quad \xi _{min} = r_x[0]-\vb{h}_{opt}^H\vb{R}_x\vb{h}_{opt} \\
  & \vb{r}_{xd}=[r_x[-\Delta ], \ldots, r_x[-\Delta -Q+1]]^T,\quad \vb{h}_{opt} = \vb{R}_x ^{-1}\vb{r}_{x}[-1]
\end{align*}

Gradient method. Given a filter $h^{(k)}$ we want to find a new one $h^{(k+1)}$ with lower MSE. (Minimize $(h^{(k)}-h_{opt})^HR_x(h^{(k)}-h_{opt})$)
\begin{align*}
  & h^{(k+1)} = h^{(k)} - \mu \nabla_{h^*}\xi (h^{(k)}) = h^{(k)} - \mu(R_xh^{(k)} - r_{xd}) \\
  & z^{(k)} := Q^H(h^{(k)}-h_{opt}) \Rightarrow z^{(k+1)} = (I-\mu \Lambda)z^{(k)}
\end{align*}

Condiciones de convergencia
\begin{align*}
& 0<\mu<\frac{2}{trace(R_x)}\le \frac{2}{\lambda_{max}}, \quad \mu_{opt}= \frac{2}{\lambda_{min}+\lambda_{max}}\\
& N_{ite} = \frac{\ln \varepsilon }{\ln|1-\mu\lambda_i|}, \quad \chi := \frac{\lambda_{max}}{\lambda_{min}}
\end{align*}

\begin{definition}[Instantaneus estimate of gradient (LMS)]
\[
\nabla \xi (h^{(n)}) = -x[n]e^*[n] \Rightarrow h^{(n+1)} = h^{(n)} + \mu x[n]e^*[n]
\] 
\end{definition}

\end{multicols}
\end{document}
